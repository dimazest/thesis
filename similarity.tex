\chapter{Background}
\label{cha:background}

\section{The notion of similarity}
\label{sec:similarity}

Similarity is the degree of resemblance between two objects or events \cite{WCS:WCS1282} and plays a crucial role in psychological theories of knowledge and behaviour, where it is used to explain such phenomena as classification and conceptualisation \cite{Tversky1977,1986-13502-00119860101,medin1993respects,Markman1996,hahn1997concepts}. \textit{Fruit} is a \emph{category} because it is a practical generalisation. Fruits are sweet and constitute deserts, so when one is presented with an unseen fruit, one can hypothesise that it is served toward the end of a dinner.

Generalisations are extremely powerful in describing a language as well. The verb \textit{runs} requires its subject to be singular. \textit{Verb}, \textit{subject} and \textit{singular} are categories that are used to describe English grammar. When one encounters an unknown word and is told that it is a verb, one will immediately have an idea about how to use it assuming that it is used similarly to other English verbs.

From a computational perspective, this motivates and guides development of similarity components that are embedded into natural language processing systems that deal with tasks such as
word sense disambiguation \cite{Schutze:1998:AWS:972719.972724},
information retrieval \cite{Salton:1975:VSM:361219.361220,Milajevs:2015:IMN:2808194.2809448},
machine translation \cite{Dagan:1993:CWS:981574.981596},
dependency parsing \cite{hermann-blunsom:2013:ACL2013,andreas-klein:2014:P14-2},
dialogue act tagging \cite{kalchbrenner-blunsom:2013:CVSC,milajevs-purver:2014:CVSC},
reasoning over knowledge bases \cite{NIPS2013_5028},
and language modelling \cite{bengio2006}.

Concretely, a parser might benefit from a generalisation about the part of speech tag of a word which did not occur in the training data based on its occurrence pattern in a large corpus of documents from the web. A dialogue act tagging system, contrastly, might require to classify a whole sentence based on its role in a dialogue. To be useful, the similarity component has to be able to measure similarity between \emph{words} and \emph{sentences}.

According to \newcite{WCS:WCS1282} ``similarity is an essentially psychological notion, based on the way we represent objects, that is, the way they appear to us.'' An important consequence of this observation is that before measuring similarity of two
\todo{What's the general term for events and objects? Entity?}%
entities,\footnotemark{} their representation has to be obtained.

\footnotetext{Here and later \emph{entity} is a common term for \textit{events} and \textit{objects}.}

% \cite{Huth2016}

\section{Similarity of words}
\label{sec:word-meaning}

Section \ref{sec:similarity} established that before measuring similarity, the representation of entities has to be agreed. One needs to be extremely careful when the representation of lexical items is decided, as it is unavoidably connected to the \emph{meaning words in isolation}.

% The semantic formalisation of similarity is based on two ideas. The occurrence pattern of a word \emph{defines} its meaning \cite{firth1957lingtheory}, while the difference in occurrence between two words \emph{quantifies} the difference in their meaning \cite{harris1954distributional}.

Frege discusses two conflicting principles of meaning \cite{Janssen2001}. Isolated words meanings are the building blocks of sentence meanings, according to \emph{the principle of compositionality}:
\begin{displayquote}[\cite{Janssen2001}]
The meaning of a compound expression is a function of the meaning of its parts and the syntactic rule by which they are combined.
\end{displayquote}
But the word meaning in isolation is not defined, according to \emph{the principle of contextuality}:
\begin{displayquote}[\cite{Janssen2001}]
Never ask for the meaning of a word in isolation, but only in the context of a sentence.
\end{displayquote}

It worth noting here, that similarity in isolation is also problematic because the number of features an entity has is infinite and its easy to show that two entities will always have an infinite amount of common features \cite{goodman1972problems,hahn1997concepts}. To make similarity measurement possible, it has to be measured \emph{under a given description} \cite{WCS:WCS1282,medin1993respects,Markman1996}, thus similarity is always contextualised. Also, \newcite{Huth2016} were able to comprehensively map individual words across cortex, meaning that there are word representations.

Frege's principle of contextuality allows to define the meaning of a word by identifying its contribution to the meaning of a sentence. Firth's \citeyearpar{firth1957lingtheory} famous quote that ``you shall know a word by the company it keeps''  suggests that the word meaning can be \emph{modelled} as the combination of the meanings of its occurrences in sentences of a corpus. Note, that this does not provide the absolute word meaning, but only its meaning relative to the corpus. This assumption is also supported by the hypothesis of \newcite{harris1954distributional} that the differences of occurrences of two words \emph{quantify} the difference in their relative meanings.

Once relative word meaning is accepted, compositionality can be used to obtain representations of phrases and sentences \cite{THEO:THEO373,Dowty1980,sep-montague-semantics,DBLP:journals/corr/abs-1003-4394,baroni2014frege}.

\subsection{Distributional hypothesis}
\label{sec:distr-hypoth}

We would like to capture the intuition that while \textit{John} and \textit{Mary} are distinct, they are rather similar to each other (both of them are humans) and dissimilar to \textit{dog}, \textit{pavement} or \textit{idea}.

% The same applies at the phrase and sentence level: \textit{dogs chase cats} is similar in meaning to \textit{hounds pursue kittens}, but less so to \textit{cats chase dogs} (despite the lexical overlap).

Distributional methods provide a way to approach this problem. By representing words and phrases as vectors in a vector space, we can express similarity in meaning via a suitable distance metric within that space.

\subsubsection{Co-occurrence based word representation}
\label{sec:distr-repr}

One way to produce such representations is to directly exploit Harris' \citeyearpar{harris1954distributional} intuition by counting the contexts a word appears in.

% that semantically similar words tend to appear in similar contexts.

For example, one can construct a vector space in which the dimensions correspond to contexts, that are usually other words. The word vector components can then be calculated by taking the frequency with which the word co-occurred with the corresponding contexts within a predefined window in a corpus of interest.

\begin{wraptable}[10]{O}{7cm}
  \centering
  \begin{tabular}{lrrr}
    \toprule
    & philosophy & book & school \\
    \midrule
    John & 4  & 60 & 59  \\
    Mary & 0  & 10 & 22  \\
    girl & 0  & 19 & 93  \\
    boy  & 0  & 12 & 146 \\
    idea & 10 & 47 & 39  \\
    \bottomrule
  \end{tabular}
  \caption{Word co-occurrence frequencies extracted from the BNC.}
  \label{tab:comparison}
\end{wraptable}

Table~\ref{tab:comparison} shows 5 3-dimensional vectors for the words \textit{Mary}, \textit{John}, \textit{girl}, \textit{boy} and \textit{idea}. The words \textit{philosophy}, \textit{book} and \textit{school} label vector space dimensions.

As the vector for \textit{Mary} is \todo{Is it accurate to talk about vector closeness?}closer to \textit{girl} than it is to \textit{boy} in the vector space, we can say that \textit{Mary}'s contexts are similar to \textit{girls}'s (and less similar to \textit{boy}'s), therefore \textit{Mary} is semantically more similar to \textit{girl} than to \textit{boy}.

Mathematically the similarity can be expressed using, for instance, the cosine of the angle between two vectors:
%
\begin{align*}
\cos(\theta) &=
\frac{\ov{\mathit{Mary}}\cdot\ov{\mathit{girl}}}
{||\ov{\mathit{Mary}}||\ov{\mathit{girl}}||} =
%
\frac{0\times0 + 10\times19 + 22\times93}
{\sqrt{0^2 + 10^2 + 22^2}\sqrt{0^2 + 19^2 + 93^2}} \approx
\frac{2236}{2294} \approx 0.975
 \\
\cos(\phi) &=
\frac{\ov{\mathit{Mary}}\cdot\ov{\mathit{boy}}}
{||\ov{\mathit{Mary}}||\ov{\mathit{boy}}||} =
%
\frac{0\times0 + 10\times12 + 22\times146}
{\sqrt{0^2 + 10^2 + 22^2}\sqrt{0^2 + 12^2 + 146^2}} \approx
\frac{3332}{3540} \approx 0.941
\end{align*}
%
where $\theta$ is the angle between the vectors of \textit{Mary} and \textit{girl}; and $\phi$ is the angle between the vectors of \textit{Mary} and \textit{boy}.

In the current example of a na{\"\i}ve vector space, \textit{John} is also closer to \textit{girl} than to \textit{boy}, which is counter-intuitive. This might be because of the small number of dimensions used, the poor selection of the context words, or the usage of raw co-occurrence numbers. Refer to \newcite{Turney:2010:FMV:1861751.1861756} and \newcite{TACL570} for the discussion of vector space parameters, and see \newcite{kiela-clark:2014:CVSC}, \newcite{lapesa2014large} and
%
\todo{reference the section about parameters} below
%
for a detailed comparison of their tuning and performance.

\subsubsection{Neural word embedding}
\label{sec:neural-embedding}

Deep learning techniques use this distributional hypothesis
differently. Instead of relying on observed co-occurrence frequencies,
a neural model is trained to maximise some objective function related
to e.g.~the probability of observing the surrounding words in some
context \cite{mikolov2013distributed}:
%
\begin{align}
 \frac{1}{T}\sum^{T}_{t=1}\sum_{-c \leq j \leq c, j\neq0} \log p(w_{t+j}|w_t)
  \label{eq:objective-func}
\end{align}
%
\noindent
Maximising this function produces vectors which maximise the
conditional probability of observing words in a context around the
target word $w_t$, where $c$ is the size of the training context, and
$w_1 w_2, \cdots w_T$ is a sequence of training words. They therefore
capture the distributional intuition and can express degrees of
lexical similarity.

% Explain why is this better than counting

They have also proved successful at other tasks \cite{mikolov2013linguistic}; the vectors obtained encode not only attributional similarity (similar words are close to each other), but also relational similarities \cite{Turney:2010:FMV:1861751.1861756}. For example, it is possible to extract the \texttt{singular:plural} relation (\textit{apple}:\textit{apples}, \textit{car}:\textit{cars}) using vector subtraction:
%
\begin{align*}
  \overrightarrow{\mathit{apple}} - \overrightarrow{\mathit{apples}}
  \approx
  \overrightarrow{\mathit{car}} - \overrightarrow{\mathit{cars}}
\end{align*}
%
also semantic relationships are preserved:
%
\begin{align*}
  \overrightarrow{\mathit{king}} - \overrightarrow{\mathit{man}}
  \approx
  \overrightarrow{\mathit{queen}} - \overrightarrow{\mathit{woman}}
\end{align*}
%
allowing the formation of analogy queries similar to
$\overrightarrow{\mathit{king}} - \overrightarrow{\mathit{man}} +
\overrightarrow{\mathit{woman}} = \mathtt{?}$, obtaining
$\overrightarrow{\mathit{queen}}$ as the
result.\footnote{\newcite{levy2014linguistic} improved
  \newcite{mikolov2013linguistic}'s method of retrieving relational similarities
  by changing the objective function and improved the state-of-the-art results
  both for neural embeddings and co-occurrence based vectors.}

% Both neural and co-occurrence-based approaches have advantages over
% classical formal approaches in their ability to capture lexical
% semantics and degrees of similarity; their success at extending this
% to the sentence level, and to more complex semantic phenomena, depends
% on their applicability within compositional models.

\subsection{Similarity metrics}
\label{sec:similarity-metrics}


\subsection{Aggregation of word similarities for similarity of compaunds}

\cite{turney2012domain}

\section{Similarity of compounds}
\label{sec:similarity-compounds}

\subsection{The principle of compositionality}
\label{sec:formal-semantics}

Formal approaches to the semantics of natural language have long built upon the
classical idea of compositionality---that the meaning of a sentence is a
function of its parts \cite{Janssen2001}. In compositional type-logical
approaches, predicate-argument structures representing phrases and sentences are
built from their constituent parts by general operations such as beta-reduction
within the lambda calculus \cite{THEO:THEO373}: for example, given a semantic
representation of \emph{John} as $\mathit{john}'$ and \emph{loves} as
$\lambda y.\lambda x.\mathit{loves}'(x, y)$, the sentence \emph{John loves Mary}
can be constructed as
$$
\lambda y.\lambda
x.\mathit{loves}'(x, y)(\mathit{mary}')(\mathit{john}') =
\mathit{loves}'(\mathit{john}', \mathit{mary}')
$$

To get the semantic representation of the sentence \textit{John loves Mary} we
need to do the following. Syntactic rules define how constituents are combined
to form other constituents (and finally a sentence). Translation rules define
how semantic representations of the constituents are combined to get a semantic
representation of the whole.

Categorial grammars are widely used to obtain syntactic structure of a
sentence. Given a set of basic categories $\texttt{ATOM}$, for example
$\{\mathit{n}, \mathit{s}, \mathit{np}\}$ complex categories
$\mathtt{CAT} \backslash \mathtt{CAT}$ and $\mathtt{CAT}/\mathtt{CAT}$ can be
constructed, where $\mathtt{CAT}$ is either an element of \texttt{\texttt{ATOM}}
or a complex category. So the transitive verb category is
$\mathtt{np}\backslash\mathtt{s}/\mathtt{np}$. Intuitively we want to say that
obtaining a sentence with a transitive verb there must be a noun phrase before
and after it.

Parsing is done by composing categories together according to two rules:
%
\begin{enumerate}
\item \textbf{Backward application}: If $\alpha$ is a string of category $A$ and
  $\beta$ is a string of category $A\backslash{}B$, then $\alpha\beta$ is of
  category $B$.
\item \textbf{Forward application}: If $\alpha$ is a string of category $A$ and
  $\beta$ is a string of category $B/A$, then $\beta\alpha$ is of category $B$.
\end{enumerate}

\begin{figure}
  \centering
  \Tree [
    .$s$
    [
      .$\mathit{np}$
      John
    ]
    [
      .$\mathit{np}\backslash{}s$
      [
        .$\mathit{np}\backslash{}\mathit{s}/\mathit{np}$
        loves
      ]
      [
        .$\mathit{np}$
        Mary
      ]
    ]
  ]
  \caption{A syntactic tree for \textit{John loves Mary}. Lexicon assigns
    categories to words: \textit{John} is $\mathit{np}$, loves is
    $\mathit{np}\backslash{}\mathit{s}/\mathit{np}$ and Mary is
    $\mathit{np}$. Backward and forward composition rules derive the syntactic
    tree.}
\label{fig:cg}
\end{figure}

Figure~\ref{fig:cg} illustrates the parse tree for \textit{John loves Mary}
obtained using the category composition rules.

The last step is to map syntactic categories with semantic terms. Again, there
are base types ($e$ for entities and $t$ for sentences) and complex types of the
form $(a \to b)$ where $a$ and $b$ are types. The mapping between syntactic
categories and semantic types is defined as a function $\mathit{type}$:
%
\begin{align*}
  &\mathit{type}(np) = e \\
  &\mathit{type}(s) = t \\
  &\mathit{type}(A/B) = (\mathit{type}(B) \to \mathit{type}(A)) \\
  &\mathit{type}(B\backslash{}A) = (\mathit{type}(B) \to \mathit{type}(A)) \\
\end{align*}

Syntactic backward and forward application corresponds to functional
application. Figure \ref{fig:syn} shows the final parse tree.

\begin{figure}
  \centering
  \Tree [
    .$s$~:~$\mathit{loves}'(\mathit{john}',\mathit{mary}')$
    [
      .$\mathit{np}$~:~$\mathit{john}'$
      John
    ]
    [
      .$\mathit{np}\backslash{}s$~:~$\lambda~x.\mathit{loves}'(x,~\mathit{mary}')$
      [
        .$\mathit{np}\backslash{}\mathit{s}/\mathit{np}$~:~$\lambda{}y.\lambda{}x.\mathit{loves}'(x,y)$
        loves
      ]
      [
        .$\mathit{np}$~:~$\mathit{mary}'$
        Mary
      ]
    ]
  ]
  \caption{The final parse tree.}
\label{fig:syn}
\end{figure}

Given a suitable pairing between a syntactic grammar, semantic representations
and corresponding general combinatory operators, this can produce structured
sentential representations with broad coverage and good generalisability \cite{step2008:2222}. This logical approach is extremely powerful because
it can capture complex aspects of meaning such as quantifiers and their
interaction \cite{Copestake2005}, and enables inference using
well studied and developed logical methods \cite{bos2000first}.

\subsection{Compositional distributional semantics}
\label{sec:composition}

Methods based on this distributional hypothesis have recently been applied to many tasks, but mostly at the word level, for instance, word sense disambiguation \cite{ZhitomirskyGeffet2009} and lexical substitution \cite{Thater:2010:CSR:1858681.1858778}. They exploit the notion of similarity which correlates with the angle between word vectors \cite{Turney:2010:FMV:1861751.1861756}.

\emph{Compositional} distributional semantics goes beyond the word level and models the meaning of phrases or sentences based on their parts. \newcite{mitchell-lapata:2008:ACLMain} perform composition of word vectors using vector addition and multiplication operations. The limitation of this approach is the operator associativity, which ignores the argument order, and thus word order. As a result, ``\textit{John loves Mary}'' and ``\textit{Mary loves John}'' get assigned the same meaning.

Concretely, if \textit{John}, \textit{Mary} and \textit{loves} meaning is
represented as vectors $\ov{\mathit{john}}$, $\ov{\mathit{mary}}$ and
$\ov{\mathit{loves}}$, the meaning of the sentence \textit{John loves Mary} is
$\ov{\mathit{john}} + \ov{\mathit{loves}} + \ov{\mathit{mary}}$.

To capture word order, various approaches have been proposed. \newcite{Grefenstette:2011:ESC:2145432.2145580} extend the compositional approach by using non-associative linear algebra operators as proposed in the theoretical work of \cite{DBLP:journals/corr/abs-1003-4394}.

The functional application of semantic term can be replaced with tensors \cite{Bourbaki1998commutative}. Then, a transitive verb is represented by matrix, which can be obtained from a corpus using the formula $\sum_i\ov{s_i} \otimes \ov{o_i}$ (the relation method of \newcite{Grefenstette:2011:ESC:2145432.2145580}), where $\ov{s_i}$ and $\ov{i_i}$ are the subject--object pairs of the verb. The vector of the whole sentence is $\overline{\mathit{loves}} \odot(\ov{\mathit{john}} \otimes \ov{\mathit{mary}})$.

\subsection{Similarity of heads in context}
\label{sec:similarity-context}

A noun phrase can be similar to a noun as in \textit{female lion} and \textit{lioness}, and to another noun phrases as in \textit{yellow car} and \textit{cheap taxi}. The same similarity principle can be applied to phrases as to words. In this case, similarity is measured in \emph{context}, but is still comparison of the phrases' head words which meaning is modified by arguments they appear with \cite{Kintsch2001173,mitchell-lapata:2008:ACLMain,mitchell2010composition,Dinu:2010:MDS:1870658.1870771,Baroni2010nouns,thater-furstenau-pinkal:2011:IJCNLP-2011,Seaghdha:2011:PMS:2145432.2145545}. With verbs this idea can be applied to compare transitive verbs with intransitive. For example, \textit{to cycle} is similar to \textit{ride a bicycle}.

% TODO: mention \cite{wieting2015paraphrase}

Sentential similarity might be treated as the similarity of the heads in the contexts. That is, the similarity between \textit{sees} and \textit{notices} in \textit{John \textbf{sees} Mary} and \textit{John \textbf{notices} a woman}. This approach abstracts away grammatical difference between the sentences and concentrates on semantics and fits the proposed model as the respect for the head, which is a lexical entity, has to be found.


\section{Intrinsic evaluation}
\label{sec:intrinsic-evaluation}

As we have established, the semantic formalisation of similarity is based on two ideas. The occurrence pattern of a word \emph{defines} its meaning \cite{firth1957lingtheory}, while the difference in occurrence between two words \emph{quantifies} the difference in their meaning \cite{harris1954distributional}. From a computational perspective, this motivates and guides development of similarity components that are embedded into natural language processing systems that deal with tasks such as
word sense disambiguation \cite{Schutze:1998:AWS:972719.972724},
information retrieval \cite{Salton:1975:VSM:361219.361220,Milajevs:2015:IMN:2808194.2809448},
machine translation \cite{Dagan:1993:CWS:981574.981596},
dependency parsing \cite{hermann-blunsom:2013:ACL2013,andreas-klein:2014:P14-2},
and dialogue act tagging \cite{kalchbrenner-blunsom:2013:CVSC,milajevs-purver:2014:CVSC}.

% TODO: Mention \cite{bengio2006}?

Because it is difficult to measure performance of a single (similarity) component in a pipeline, datasets that focus on similarity are popular among computational linguists. Apart from a pragmatic attempt to alleviate the problems of evaluating similarity components, these datasets serve as an empirical test of the hypotheses of Firth and Harris, bringing together our understanding of human mind, language and technology.


\subsection{Word similarity}
\label{sec:lexical-similarity}

Two datasets, namely MEN \cite{Bruni:2012:DST:2390524.2390544} and SimLex-999 \cite{hill2014simlex}, are currently widely used. They are designed especially for meaning representation evaluation and surpass datasets stemming from psychology \cite{1986-13502-00119860101}, information retrieval \cite{2002:PSC:503104.503110} and computational linguistics \cite{Rubenstein:1965:CCS:365628.365657} in quantity by having more entries and, in case of SimLex-999, attention to the evaluated relation by distinguishing similarity from relatedness. The datasets provide similarity (relatedness) scores between word pairs.

\subsection{Disambiguation of verbs in context}
\label{sec:disamb}

The transitive verb disambiguation dataset\footnote{This and the sentence   similarity datasets are available at   \url{http://www.cs.ox.ac.uk/activities/compdistmeaning/}} described in \cite{Grefenstette:2011:ETV:2140490.2140497}. The dataset consists of ambiguous transitive verbs together with their arguments; landmark verbs, which identify one of the verb senses; and human judgements which specify the similarity to the landmarks of the disambiguated sense of the verb in the context given. This is similar to the intransitive dataset described in \cite{mitchell-lapata:2008:ACLMain}.

Consider the sentence \textit{``system meets specification''}; here,
\textit{meets} is the ambiguous transitive verb, and \textit{system}
and \textit{specification} are the arguments in this context. Possible
landmarks for \emph{meet} are \textit{satisfy} and \textit{visit}; for
this sentence, the human judgements show that the disambiguated verb
meaning is similar to the landmark \textit{satisfy}, and less similar
to \textit{visit}.

The task is to estimate the similarity of the sense of a verb in a context with
a given landmark. To provide our similarity measures, we compose the verb with
its arguments using one of our compositional operators, do the same for the
landmark and the arguments, and compute the cosine similarity of the two
vectors. To evaluate performance, we average the human judgements for the same
verb, argument and landmark entries, and use the average values to calculate the
correlation. As a baseline, we compare to the correlation achieved using only
the verb vector, without composing with its arguments.

\subsection{Sentence similarity}
\label{sec:sentence-similarity}

The transitive sentence similarity dataset described in \newcite{kartsaklis-sadrzadeh-pulman:2013:CoNLL-2013}. The dataset consists of transitive sentence pairs and a human similarity judgement. The task is to estimate similarity between two sentences.

\section{Tasks that require similarity}
\label{sec:nlp-tasks}

\subsection{Dialogue act tagging}
\label{sec:dialogue-act-tagging}

There are many ways to approach the task of dialogue act tagging
\cite{Stolcke:2000:DAM:971869.971872}. The most successful approaches combine
\emph{intra-}utterance features, such as the (sequences of) words and
intonational contours used, together with \emph{inter-}utterance
features, such as the sequence of utterance tags being used
previously.
%
To capture both of these aspects, sequence models such as Hidden Markov Models are widely used \cite{Stolcke:2000:DAM:971869.971872,DBLP:conf/interspeech/SurendranL06}. The sequence of words is an observable variable, while the sequence of dialogue act tags is a hidden variable.

However, some approaches have shown competitive results without exploiting features of inter-utterance context. \newcite{webb2005dialogue} concentrate only on features found inside an utterance, identifying ngrams that correlate strongly with particular utterance tags, and propose a statistical model for prediction which produces close to the state of the art results.

The current state of the art \cite{kalchbrenner-blunsom:2013:CVSC} uses Recurrent Convolutional Neural Networks to achieve high accuracy. This model includes information about word identity, intra-utterance word sequence, and inter-utterance tag sequence, by using a vector space model of words with a compositional approach. The words vectors are not based on distributional frequencies in this case, however, but on randomly initialised vectors, with the model trained on a specific corpus. This raises several questions: what is the contribution of word sequence and/or utterance (tag) sequence; and might further gains be made by exploiting the distributional hypothesis? What is the contribution of utterance meaning to its tag?

\subsection{Paraphrase detection}
\label{sec:paraphrase}

Microsoft paraphrase corpus \cite{dolan2005par} is a collection of sentences labeled whether one is a paraphrase of another.

%%% Local Variables:
%%% mode: latex
%%% TeX-master: "thesis.tex"
%%% End:
