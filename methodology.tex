\chapter{A methodology for robust parameter selection}
\label{sec:methodology}

Experiments with distributional vector space models can be divided to two related categories: one that aims to achieve the highest score on a task, and another that studies the difference between certain parameters. The difference between the two categories can be expressed in the following questions:

\begin{compactitem}
\item What parameter combination gives the highest result? \newcite{baroni-dinu-kruszewski:2014:P14-1} is a representative study of this kind.
\item What parameter instance is superior? For example, the comparison of neighbour rank and distance measure in predicting semantic priming \cite{lapesa-evert:2013:CMCL}.
\end{compactitem}

The first question is applicable in a situation when conceptually different methods are compared, for example ``count'' and  ``predict'' methods \cite{baroni-dinu-kruszewski:2014:P14-1}. The second question is applicable to a study of the difference of parameters within a conceptual method as in \newcite{lapesa-evert:2013:CMCL}.

The co-occurrence information can be used in different ways to build distributional models of meaning \cite{Turney:2010:FMV:1861751.1861756}. This has led to a series of systematic parameter studies \cite{Bullinaria2007,BullinariaLevy2012,kiela-clark:2014:CVSC,lapesa2014large,TACL570}. All of them explore numerous parameter combinations, report the best scores, compare and derive recommendations of the parameter choices.

\newcite{lapesa2014large} make one step further in studying parameter behaviour by identifying the most influential parameters and their two-way interactions with a linear model, which is fit in such a way that parameters of a vector space model predict its performance. This approach solves two problems. First of all, it is designed to avoid overfitting. Secondly, it avoids noise in parameter selection revealing the regularity in the parameter choices.

The goal of this work is to provide the representative performance numbers of count-based distributional models of meaning---so they could be compared to other semantic models---and to study general behaviour of vector space parameters and compositional operators---so they could be compared. The study is performed systematically using recently developed evaluation datasets for lexical and phrasal meaning representations.

\section{Strategies of avoiding overfitting}
\label{sec:avoiding-overfitting}

This work adopts the strategy of \newcite{lapesa2014large} to avoid overfitting and reduce noise in parameter selection. Several evaluation datasets are used individually and in combination.

The models are tested on two lexical datasets: SimLex-999 \cite{hill2014simlex} and MEN \cite{Bruni:2014:MDS:2655713.2655714}. These two datasets are chosen because they are larger than other previously used datasets. It has been argued that the score variance is strongly dependent on the size of the evaluation dataset \cite{W16-2502}. SimLex-999 consists of 999 word pairs and MEN consists of 3000. Compare this with the sizes of other popular datasets: 353 \cite{2002:PSC:503104.503110} and 65 \cite{Rubenstein:1965:CCS:365628.365657}.

The three compositional datasets that are employed in this study are KS14 \cite{kartsadrqpl2014}, GS11 \cite{Grefenstette:2011:ESC:2145432.2145580} and PhraseRel (Section~\ref{sec:phraserel}). They consist of phrases with controlled syntax (they are subject-verb-object phrases) and cover several aspects: similarity, disambiguation and relevance.

We combine the lexical datasets to see whether there are models that perform well on both lexical datasets simultaneously and thus to supposingly avoid individual dataset idiosyncrasies. Similarly, we combine the phrasal datasets.

Lexical and phrasal datasets are combined to look for a universal vector space. The model selection procedure is performed in two ways. First, we take the compositional operator into account, so we are able to recommend models that perform well on lexical dataset and phrasal datasets with, for example, addition. Finally, we abstract over the compositional method and seek for parameters that are universally good for lexical and phrasal tasks for all operator.

Because we test the models on several datasets, transfer selected models to the unseen datasets and perform model selection on their combinations (which is another way of avoiding overfitting) we also report the models that performed best in our exhaustive evaluation. This allows us to see whether overfitting actually happens, as we expect that during transfer the models with the highest score will degrade their performance more than the models selected using heuristics.

Concretely, we perform 3 different parameter selections that are discussed below.

\subsection{Best model}

This parameter selection technique chooses the parameters that yield the best result. This method is widely adopted \cite{mitchell-lapata:2008:ACLMain,Grefenstette:2011:ESC:2145432.2145580,milajevs-purver:2014:CVSC,milajevs-EtAl:2014:EMNLP2014}, but as previously discussed is prone to overfitting.

\subsection{Cross-validation}

Cross-validation is similar model selection method, but the parameter selection is done on the average performance of the training splits over 5 folds. We report the average performance over the testing splits.

Even though cross-validation avoids overfitting, its performance results are not compatible with the best model selection, because they are based on averages over the folds. Also, the existing datasets are not made with such evaluation in mind \cite{W16-2506}.

\subsection{Heuristics}

This parameter selection is based on parameters' average behaviour. We look for the average parameter performance for the each dimensionality (for lexical experiments) or operator-dimensionality combination (for compositional experiments) and choose the parameter value with the highest upper bound of the 0.95 confidence interval.

The parameters processed in order of their ablation. Parameter's ablation is proportional to the reduction of the adjusted $R^2$ score if the parameter is left out. Refer to \newcite{lapesa2014large} for the details.

This method not only avoids overfitting but also yields evaluation results that are comparable with the best-model reports.

\section{Hypotheses}
\label{sec:hypotheses}

\begin{hyp}[H\ref{hyp:overfitting}]
\label{hyp:overfitting}
Heuristic-based model selection avoids overfitting.
\end{hyp}

In other words, models that are chosen using heuristic-based selection achieve better results than max-based selected models on the datasets they were not instantiated on.

\begin{hyp}[H\ref{hyp:10percent}]
\label{hyp:10percent}
The performance difference between the best behaving model and a model selected using heuristics is 10\%.
\end{hyp}

The optimal results reported in \citep[\textcolor{citecolor}{Table~5}]{lapesa2014large} are within 10\% (with an exception of the ESSLLI dataset, where the difference is 21\%). We expect similar relative difference.

\begin{hyp}[H\ref{hyp:dimen}]
\label{hyp:dimen}
The optimal parameter choice depends on dimensionality.
\end{hyp}

\newcite{milajevs-sadrzadeh-purver:2016:ACL-SRW} showed that the parameter recommendations provided by \newcite{TACL570} are not applicable to the vector spaces with dimensionality of few thousand.

The difference could be because the counts of the most frequent words do not contain noise. The counts (and therefore the probability estimates) of less frequent words are noisy and require special treatment such as smoothing to patch PMI's Achilles heel when small counts receive a high value \cite{TACL570}.

\begin{hyp}[H\ref{hyp:var}]
\label{hyp:var}
Highly dimensional models are more likely to perform better than their low dimensional counterparts.
\end{hyp}

As the vector space dimensionality increases, performance stabilises \cite{kiela-clark:2014:CVSC,BullinariaLevy2012,lapesa2014large}. We speculate that in a highly dimensional case the difference between parameter choices matter less and thus higher results are reported more often for highly-dimensional spaces.

\begin{hyp}[H\ref{hyp:lextocomp}]
\label{hyp:lextocomp}
Models that perform exceptionally well on lexical tasks not necessarily perform well on compositional tasks.
\end{hyp}

\begin{hyp}[H\ref{hyp:order}]
\label{hyp:order}
In compositional tasks, models that take word order into account perform better than the models that ignore it.
\end{hyp}

Addition perform very well in compositional tasks \cite{milajevs-EtAl:2014:EMNLP2014}, but that could be because optimal parameters for other operators were not chosen.

\begin{hyp}[H\ref{hyp:universal}]
\label{hyp:universal}
There is a universal model that performs well on a broad range of tasks.
\end{hyp}

\chapter{Parameters and experimental setup}

\todo[inline]{Intro}

\section{Co-occurrence quantification}
\label{sec:quantification}

\subsection{PMI variants (discr)}
\label{sec:pmi-variants}

Most co-occurrence weighting schemes in distributional semantics are based on \emph{point-wise mutual information} (PMI, Equation~\ref{eq:pmi}, \newcite{J90-1003,Turney:2010:FMV:1861751.1861756,NIPS2014_5477}).
%
\begin{equation}
  \label{eq:pmi}
  \operatorname{PMI}(x, y) = \log\frac{P(x,y)}{P(x)P(y)}
\end{equation}
%
PMI in its raw form is problematic: non-observed co-occurrences lead to infinite PMI values, making it impossible to compute similarity. A common ``fix'' to this problem is to replace all infinities with zeroes, and we use PMI hereafter to refer to a weighting with this fix, refer to Section~\ref{sec:quantification-measures} for a discussion.

An alternative solution is to increment the probability ratio by 1; we refer to this as \textit{compressed PMI} (CPMI):
%
\begin{equation}
  \label{eq:cpmi}
  \operatorname{CPMI}(x, y) = \log\Big( 1 + \frac{P(x,y)}{P(x)P(y)} \Big)
\end{equation}

\subsection{Shifted PMI (neg)}
\label{sec:shifted-pmi}

Many approaches use only \emph{positive} PMI values, as  negative PMI values may not positively contribute to model performance \cite{Turney:2010:FMV:1861751.1861756}. This can be generalised to an additional cutoff parameter $k$ (\texttt{neg}) following \newcite{TACL570}, giving our third PMI variant (abbreviated as SPMI):
%
\begin{equation}
  \label{eq:ppmi}
  \operatorname{SPMI_k} = \max (0, \operatorname{PMI}(x, y) - \log k)
\end{equation}
%
We can apply the same idea to CPMI:
%
\begin{equation}
  \label{eq:pcpmi}
  \operatorname{SCPMI_k} = \max (0, \operatorname{CPMI}(x, y) - \log 2k)
\end{equation}

\subsection{Frequency weighting (freq)}
\label{sec:frequency-weighting}

Another issue with PMI is its bias towards rare events \cite{TACL570}; one way of solving this issue is to weight the value by the co-occurrence frequency \cite{Evert05}:
%
\begin{equation}
  \label{eq:lmi}
  \operatorname{LMI}(x, y) = n(x, y)\operatorname{PMI}(x, y)
\end{equation}
%
where $n(x, y)$ is the number of times $x$ was seen together with $y$. For clarity, we refer to $n$-weighted PMIs as \NPMI/, \NSPMI/, etc. When this weighting component is set to 1, it has no effect; we can  explicitly label it as \PMI/, \SPMI/, etc.

In addition to the extreme $1$ and $n$ weightings, we also experiment with a \todo{Look into the ``Inspired IR'' paper for the discussion of $\log n$}$\log n$ weighting.

\subsection{Context distribution smoothing (cds)}
\label{sec:cont-distr-smooth}

\newcite{TACL570} show that performance is affected by smoothing the context distribution $P(x)$:
%
\begin{equation}
  \label{eq:cds}
  P_{\alpha}(x) = \frac{n(x)^{\alpha}}{\sum_{c}n(c)^{\alpha}}
\end{equation}
%
We experiment with $\alpha=1$ (no smoothing) and $\alpha = 0.75$. We call this estimation method \emph{local context probability}; we can also estimate a \emph{global context probability} based on the size of the corpus $C$:
%
\begin{equation}
  \label{eq:cds-nan}
  P(x) = \frac{n(x)}{|C|}
\end{equation}

\subsection{Quantification measure generalisation}
\label{sec:quantification-measures}

To systematically study the aforementioned quantification measures together with other variations we propose to view all these measures as instances of this general formula:
%
\begin{equation}
  \small
  \label{eq:association}
  \operatorname{Quantification}(x,y) = \operatorname{freq}(x, y)
                                       \operatorname{discr}(x, y)
\end{equation}
%
which consists of two components: $\operatorname{freq}(x, y)$ which quantifies the co-occurrence of two terms: a target term $x$ and a feature term $y$, and $\operatorname{discr}(x, y)$ which quantifies the ``surprise'' or ``informativeness'' of seeing (or not seeing) the two terms together,  labeled as discriminativeness.

In this framework, PMI can be seen as a quantification measure where the frequency component is the constant 1 and the discriminativeness is PMI itself. PosPMI is seen analogously. For LMI, $\operatorname{freq}(x, y) = n(x, y)$ and $\operatorname{discr}(x, y) = \operatorname{PMI}(x, y)$. It makes sense to set the frequency to 1 for PMI and PosPMI as both of them already have the co-occurrence information.

% ITTF is a counterpart of IDF in TF-IDF. We define ITTF as
% \begin{equation}
%   \small
%   \label{eq:ittf}
%   \operatorname{ITTF}(y) = \log\frac{|W|}{|\{w \in W: n(w, y) > 0\}|}
% \end{equation}
% where $W$ is the set of all words used in a corpus. Whereas PMI depends on both the target word and the feature word, ITTF depends only on the feature, therefore $\operatorname{discr}(x, y) = \operatorname{ITTF}(y)\ \text{for all targets}\ x$.

%  Table~\ref{fig:association-measures} shows the quantification measures discussed in this paper.

% \begin{table}[ht]
%   \centering
%   \begin{tabular}{rccc}
%     \toprule
%     \multirow{2}{*}{Frequency} &
%     \multicolumn{3}{c}{Discriminativeness} \\
%     \cmidrule(r){2-4}
%                   & PMI        & PosPMI     & ITTF       \\
%     \midrule
%     $1$                 & \checkmark & \checkmark &            \\
%     $\log(n)$           & \checkmark & \checkmark & \checkmark \\
%     $n$                 & \checkmark & \checkmark & \checkmark \\
%     \bottomrule
%   \end{tabular}
%   \caption{\textbf{Quantifications studied in this work.}
%     %Quantification is a combination of frequency and discriminativeness.
%   }
%   \label{fig:association-measures}
% \end{table}

% TODO: Needs attention!
From the probabilistic point of view, under the independence assumption of two words occurring together, \NPMI/ can be seen as measuring the logarithm of the ratio of the probabilities of groups of length $n$ (the group that contains only $(x,y)$s and another one that contains $x$s and $y$s):
%
\begin{equation*}
  n\log\frac{P(x, y)}{P(x)P(y)} = \log\frac{P(x, y)^{n}}{P(x)^{n}P(y)^{n}}
\end{equation*}
%
When the constant 1 is used as the frequency component, it is equivalent to the subsumption assumption, when the probability of a sequence depends only on what elements it contains, rather than how many of them.

From the geometric point of view, the transformation from \PMI/ to \NPMI/ changes the directions of vectors by pulling the vectors towards the dimensions for which $n(x, y)$ is higher. As a side effect, it also stretches the vectors. The importance of these two effects is discussed later in Section~\ref{sec:similarity-measure}.

From the linguistic perspective, \PMI/ captures a tendency of a word to co-occur with another word in general (captured by the direction of a vector), while \NPMI/ captures the expectation of seeing a particular co-occurrence in the source corpus. This is encoded in both the direction and the length of a vector.

The sublinear frequency\footnote{We increment $n$ by one to avoid an infinite logarithm value when $n$ is 0.} $\log(n)$ stands between the two extreme cases of independence and subsumption. However, it is still a strong assumption as it treats all word pairs equally, but in natural language there are some pairs that are closer to subsumption and others that are closer to independence.

In our case, the constant frequency is the only association measure that is able to distinguish from the case where a co-occurrence event was not observed (and so according to MLE, $P(x, y)= 0$) and the case where the events are independent ($P(x, y) = P(x)P(y)$). Because for other frequencies, when no co-occurrence is observed the frequency is 0 and the discriminativeness value does not matter unless it is finite. That is why the backing off strategy already mentioned in Section~\ref{sec:pmi-variants}, ``when $n(x, y) = 0$ assume that $P(x, y) = P(x)P(y)$'' is an appropriate way of avoiding $-\infty$. There are many other smoothing strategies, for example \newcite{kneser1995improved,bengio2006}.

\section{Other model parameters}
\label{sec:other-model-paramt}

\subsection{Vector dimensionality (D)}
\label{sec:vect-dimens}

As context words we select the 1K, 2K, 3K, 5K, 10K, 20K, 30K, 40K and 50K most frequent lemmatised nouns, verbs, adjectives and adverbs. All context words are part of speech tagged, but we do not distinguish between refined word types (e.g.~intransitive vs.~transitive versions of verbs).

\subsection{Similarity measure}
\label{sec:similarity-measure}

To be able to measure the similarity of two words, we need to be able to compare their vectors. A very high-level approach is to look how two words agree on the features. If two word vectors tend to have equal values for most of the components, then this is a good indication of the similarity of the words they represent.

The cosine of the angle between two vectors is a widely used similarity measure in distributional semantics \cite{Turney:2010:FMV:1861751.1861756,lapesa2014large}.
%
\begin{equation*}
  \label{eq:cos}
  \cos(\vec{x}, \vec{y}) = \frac{\vec{x} \cdot \vec{y}}
                                {\|\vec{x}\| \|\vec{y}\|}
\end{equation*}

However, the inner product $\vec{x} \cdot \vec{y}$ is preferred in IR and current state-of-the-art NLP systems \cite{mikolov2013distributed,mikolov2013linguistic,TACL570}. Cosine is equivalent to the inner product if the vectors are Euclidean ($L_2$) normalized.

Euclidean normalization removes the effect of vector lengths, emphasizing instead their directions. Thus, remembering that vector length depends on overall frequency, linguistically we have two measures, cosine that is concerned with a similarity, and inner product which also reflects word frequency or expectation factors.

An advantage of cosine in a lexical similarity task is that it does not depend on the word frequency. Imagine a case where the similarity of a frequent and a rare word is calculated. In this situation, the similarity judgement should not depend on the relative frequency of the words; instead, their tendency of agreement on features should take the dominant role.

For example, \NPMI/ makes ``feature selection'' by weighting PMI values with the co-occurrence frequency as discussed \todo{Make sure that the stretching effect is discussed}previously. When cosine is applied, the stretching effect of \NPMI/ is eliminated, but the rotation effect stays. On average, the rotational effect will be much more significant for rare words, while frequent words are more likely to be stretched.

\subsection{Compositional operator}
\label{sec:comp-oper}

\todo[inline]{Compositional operators}


% \begin{wraptable}[13]{O}{0.5\textwidth}
\begin{table}
  % \vspace{-2em}
  \centering
  \small
  \begin{tabular}{lcl}
    \toprule
    Parameter           & Abbreviation   & Values \\
    \midrule
    Dimensionality      & $D$            & 1K, 2K, 3K, 5K, 10K, 20K, 30K, 40K, 50K \\
    Discriminativeness  & \texttt{discr} & PMI, CPMI, SPMI, \textbf{SCPMI} \\
    Frequency weighting & \texttt{freq}  & $1$, $n$, $\boldsymbol{\log n}$ \\
    Shifting            & \texttt{neg}   & \textbf{0.2}, \textbf{0.5}, \textbf{0.7}, 1, 1.4, 2, 5, 7 \\
    Context distribution smoothing       & \texttt{cds} & \textbf{\textit{global}}, 1, 0.75 \\
    Similarity          &                & Cosine, Correlation and Inner product \\
    \addlinespace
    Window size         &                & 5 from both sides                     \\
    Corpus              &                & Concatenation of ukWaC and Wackypedia \\
    \bottomrule
  \end{tabular}
  \caption{Model parameters and their values used in the experiments. To our knowledge, values are bold have not been used previously.}
\label{tab:parameters}
\end{table}

%%% Local Variables:
%%% mode: latex
%%% TeX-master: "../thesis"
%%% End:


%%% Local Variables:
%%% mode: latex
%%% TeX-master: "thesis"
%%% End:
