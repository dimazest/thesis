\chapter{Reflection}
\label{cha:reflection}

\section{The notion of similarity}
\label{sec:notion-similarity}

In contrast to linguistic datasets which contain randomly paired words from a broad selection, datasets that come from psychology contain entries that belong to a single category such as \textit{verbs of judging} \cite{FILLENBAUM197454} or \textit{animal terms} \cite{HENLEY1969176}. The reason for category oriented similarity studies is that ``stimuli can only be compared in so far as they have already been categorised as identical, alike, or equivalent at some higher level of abstraction'' \cite{turner1987rediscovering}. Moreover, because of the \emph{extension effect} \cite{medin1993respects}, the similarity of two entries in a context is less than the similarity between the same entries when the context is extended. ``For example, \textit{black} and \textit{white} received a similarity rating of 2.2 when presented by themselves; this rating increased to 4.0 when \textit{black} was simultaneously compared with \textit{white} and \textit{red} (\textit{red} only increased 4.2 to 4.9)'' \cite{medin1993respects}. In the first case \textit{black} and \textit{white} are more dissimilar because they are located on the extremes of the greyscale, but in the presence of \textit{red} they become more similar because they are both monochromes.

Both MEN and SimLex-999 provide pairs that do not share any similarity to control for false positives, and they do not control for the comparison scale. This makes similarity judgements ambiguous as it is not clear what low similarity values mean: incompatible notions or contrast in meaning. SimLex-999 assigns low similarity scores to the incompatible pairs (0.48, \textit{trick} and \textit{size}) and to antonymy (0.55, \textit{smart} and \textit{dumb}), but \textit{smart} and \textit{dumb} have relatively much more in common than \textit{trick} and \textit{size}!

\section{3 ways of measuring similarity}
\label{sec:3-ways-measuring}

\subsection{Similarity in context}
\label{sec:similarity-context}

A noun phrase can be similar to a noun as in \textit{female lion} and \textit{lioness}, and to another noun phrases as in \textit{yellow car} and \textit{cheap taxi}. The same similarity principle can be applied to phrases as to words. In this case, similarity is measured in \emph{context}, but is still comparison of the phrases' head words which meaning is modified by arguments they appear with \cite{Kintsch2001173,mitchell-lapata:2008:ACLMain,mitchell2010composition,Dinu:2010:MDS:1870658.1870771,Baroni2010nouns,thater-furstenau-pinkal:2011:IJCNLP-2011,Seaghdha:2011:PMS:2145432.2145545}. With verbs this idea can be applied to compare transitive verbs with intransitive. For example, \textit{to cycle} is similar to \textit{ride a bicycle}.

% TODO: mention \cite{wieting2015paraphrase}

Sentential similarity might be treated as the similarity of the heads in the contexts. That is, the similarity between \textit{sees} and \textit{notices} in \textit{John \textbf{sees} Mary} and \textit{John \textbf{notices} a woman}. This approach abstracts away grammatical difference between the sentences and concentrates on semantics and fits the proposed model as the respect for the head, which is a lexical entity, has to be found.

\subsection{Composition of similarity scores}
\label{sec:comp-simil-scor}

\cite{turney2012domain}

\subsection{Similarity of compounds}
\label{sec:similarity-compounds}

\section{Relevance: another semantic measure}
\label{sec:meas-relev-vect}


\section{Problems with model's hyperparameter selection during evaluation}
\label{sec:param-comb-model}


%%% Local Variables:
%%% mode: latex
%%% TeX-master: "thesis"
%%% End:
