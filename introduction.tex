\chapter{Introduction}
\label{ch:introduction}


Computers require specially designed programming languages to be controlled, despite the fact that they play a crucial role in our lives. Ideally, it would be perfect if the interaction with computers was not different from the interaction between humans. Computational linguistics is one of the fields that addresses this problem.

In order to be controlled by people, or be able to assist people in language-related tasks, computers need to understand it. However, different tasks require various levels of language understanding. For instance, even if one does not recognize or know the language of a piece of text in Figure~\ref{fig:lv}, one can tell how many words there are, and that there is only one sentence. After a while, one can even say that this is probably a piece of poetry.
% * <sdyck@ualberta.ca> 2016-11-03T00:02:27.725Z:
%
% > After a while, 
%
% After a while of what? Analyzing it? reading it? 
%
% ^.
% * <sdyck@ualberta.ca> 2016-11-03T00:00:40.843Z:
%
% > nd it.
%
% I would be a little more specific than just "it". The computers must understand the language used?
%
% ^.

The conclusions above require neither the complete understanding of the language nor the meaning of the text. The knowledge that texts---at least in some languages---consist of words separated by a space and how poems are written is enough. Moreover, knowing the letter distribution across all human languages or having a list of words in them, one would conclude that the text is in Latvian. This information can be provided without knowing what the text is about.

\begin{figure}[t]
\tiny
\begin{subfigure}[t]{0.30\textwidth}
  Jaunkundze ar sunīti \\

  Un Vecrīgas šķērsielā, šaurā \\
  kā vēstuļu kastītes sprauga, \\
  kur troksnim un burzmai tik atbalss, \\
  kur smaržo pēc darvas, \\ ${}$\qquad dzelzs un pēc āboliem pagrabos sausos, \\
  es satiku jaunkundzi -- \\
  glītu un veiklu kā mēle, \\
  kā spēlējot vijoles lociņš.
  \vspace{0.74cm}
  \caption{}
  \label{fig:lv}
\end{subfigure}
~
\begin{subfigure}[t]{0.36\textwidth}
  Барышня с собачкой \\

  В Старой Риге, на улице поперечной, узкой, \\
  как щель в почтовый ящик, \\
  в который проникают только отголоски шума, гама, \\
  где запах дёгтя, ржавчины и яблок в сухих  подвалах, \\
  я встретил барышню -- \\
  красива и ловка - она - язык, \\
  смычок, играющий на скрипке.
  \vspace{0.99cm}
  \caption{}
  \label{fig:ru}
\end{subfigure}
~
\begin{subfigure}[t]{0.26\textwidth}
  Young Woman with a Dog \\

  On a narrow side-street in Riga’s old quarter, \\
  as though in a mailbox slot \\
  where noise and hustle only echo, \\
  and it smells of tar and steel \\
  and apples kept in dry basements, \\

  I met a young woman \\
  attractive and active \\
  as a tongue, \\
  as a violin-bow playing.
  \caption{}
  \label{fig:en}
\end{subfigure}
\caption[Three pieces of written natural language.]{Three pieces of written natural language. The text on
  Figure~\ref{fig:lv} is the beginning of the poem ``Jaunkundze ar sunīti'' by
  Aleksandrs Čaks, Figure~\ref{fig:ru} is a translation to Russian by Lora Trin,
  and Figure~\ref{fig:en} is an English translation by Inara Cedrins.}
\end{figure}


On the other hand, a task that asks for a list of  word associations for a text, essay or  painting inspired by it demands a much better understanding of the text that requires a deeper knowledge of the language and greater familiarity with the culture. Luckily, nowadays these kinds of tasks are not expected to be completed by computers in day-to-day life because people enjoy doing these things themselves.

However, it is reasonable to ask a computer the following questions regarding the text:
\begin{inparaenum}[a)]
\item What is the text of Figure~\ref{fig:lv} about?
\item What is the relationship between the texts in Figure~\ref{fig:lv} and
  Figure~\ref{fig:ru}?
% the *meanings are identical*
\item Is the content similar or identical?
\item Where did the meeting take place?
\item What poems are similar to this?
\end{inparaenum}

Text summarisation, machine translation, information extraction and information retrieval are branches of computational linguistics that provide methods for answering these questions. The questions above have a general property: all of them are about the meaning of the text. Natural language semantics is an area that studies meaning and thus, is necessary to solve the tasks.

The creativity of natural languages---the fact that humans are able to produce and understand sentences they have never come across---complicates meaning modeling. Even if we had a way to map each word to its meaning, it is impractical to apply the same procedure to sentences because as we process a piece of text, most of the sentences in it will be seen for the fist time. Therefore, we need to be able to build the meaning of a sentence from its constituent parts.
% * <sdyck@ualberta.ca> 2016-11-03T00:24:15.438Z:
%
% I would suggest adding here, ", rather than the sentence in its entirety."
%
% ^.

Syntax is the study of the structure of a sentence. Grammar defines the rules that describe what a sentence that belongs to a language should look like. For example, In an English sentence, a subject is written before a verb and an object is after. Having the constituent meaning representations, the meaning of a sentence is built by following the syntactic structure of it.

To be able to deal with the meaning of a text in a natural language, one needs to have meaning representation of constituents, the syntactic structure and a compositional procedure that outputs the meaning representation.

Recently, lexical distributional semantics \cite{BullinariaLevy2012,Bullinaria2007,Turney:2010:FMV:1861751.1861756} has advanced after the success of word2vec \cite{mikolov2013linguistic,mikolov2013distributed,mikolov2013efficient}. At the same time, developments in compositional distributional semantics \cite{mitchell2010composition,maillard-clark-grefenstette:2014:TTNLS,Grefenstette:2011:ESC:2145432.2145580,Grefenstette:2011:ETV:2140490.2140497,kartsadrqpl2014,fried-polajnar-clark:2015:ACL-IJCNLP} showed that Frege's principle of compositionality \cite{Janssen2001} is useful in obtaining representations of phrases and sentences.
% * <sdyck@ualberta.ca> 2016-11-03T00:28:29.470Z:
%
% > word2vec
%
% I would briefly explain what this is, otherwise the second point in this paragraph about Frege's principle doesn't have much context. 
%
% ^.

Even though the new generation of meaning representation models is shown to be superior in compositional tasks \cite{milajevs-EtAl:2014:EMNLP2014}, it has been criticised---most notably by \newcite{TACL570}. However, it is not yet known whether the critique of word2vec also applies to the compositional setting.
% * <sdyck@ualberta.ca> 2016-11-03T00:31:07.929Z:
%
% > compositional 
%
% should this be lexical?
%
% ^.

This work adopts the recommendations of \newcite{TACL570} for model parameter tuning and performs a large-scale model selection study on lexical and compositional tasks.
% * <sdyck@ualberta.ca> 2016-11-03T00:37:45.835Z:
%
% > This work adopts the recommendations of \newcite{TACL570} for model parameter tuning and performs a large-scale model selection study on lexical and compositional tasks.
%
% ^.

This work shows that after a careful selection of  vector space model parameters, the selected models are competitive with the current state-of-the-art results and on some datasets, outperform it.
% * <sdyck@ualberta.ca> 2016-11-03T00:32:14.807Z:
%
% > The experiments show that after a careful selection of  vector space model parameters, the selected models are competitive with the current state-of-the-art results and on some datasets outperform it.
%
% This sentence could use a bit of clarification. What kind of results are you talking about? Maybe use the word model(s) rather than results.
%
% ^.

%%% Local Variables:
%%% mode: latex
%%% TeX-master: "thesis"
%%% End:
