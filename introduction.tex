\chapter{Introduction}
\label{ch:introduction}

Computers, machines that play a more and more important role in our lives,
require specially designed programming languages to be controlled. This is
different from interactions between people where spoken or written natural
language is used. Ideally, it would be perfect if interactions with computers
was not different to interaction between humans. Computational linguistics is
one of the fields that aims to solve this problem.

In order to be controlled by people, or be able to assist people in language
related tasks, computers need to understand it. However, different tasks need
various level of language ``understanding''. For instance, even if one does not
recognize or know the language of a piece of text on Figure~\ref{fig:lv}, he or
she can tell how many words there are, and that there is only one
sentence. After a while, one can even say that this is probably a piece of
poetry.

The conclusions above require neither deep understanding of the language nor the
meaning of the text. Knowledge that texts (at least in some languages) consist
of words separated by a space and how poems are usually written is
enough. Moreover, knowing the letter combination distribution across the
languages or a list of words for all human languages, one would conclude that
the text on Figure~\ref{fig:lv} is in Latvian. We managed to get this answers
without knowing what the text is about.

\begin{figure}[t]
\tiny
\begin{subfigure}[t]{0.30\textwidth}
  Jaunkundze ar sunīti \\

  Un Vecrīgas šķērsielā, šaurā \\
  kā vēstuļu kastītes sprauga, \\
  kur troksnim un burzmai tik atbalss, \\
  kur smaržo pēc darvas, \\ ${}$\qquad dzelzs un pēc āboliem pagrabos sausos, \\
  es satiku jaunkundzi -- \\
  glītu un veiklu kā mēle, \\
  kā spēlējot vijoles lociņš.
  \vspace{0.74cm}
  \caption{}
  \label{fig:lv}
\end{subfigure}
~
\begin{subfigure}[t]{0.36\textwidth}
  Барышня с собачкой \\

  В Старой Риге, на улице поперечной, узкой, \\
  как щель в почтовый ящик, \\
  в который проникают только отголоски шума, гама, \\
  где запах дёгтя, ржавчины и яблок в сухих  подвалах, \\
  я встретил барышню -- \\
  красива и ловка - она - язык, \\
  смычок, играющий на скрипке.
  \vspace{0.99cm}
  \caption{}
  \label{fig:ru}
\end{subfigure}
~
\begin{subfigure}[t]{0.26\textwidth}
  Young Woman with a Dog \\

  On a narrow side-street in Riga’s old quarter, \\
  as though in a mailbox slot \\
  where noise and hustle only echo, \\
  and it smells of tar and steel \\
  and apples kept in dry basements, \\

  I met a young woman \\
  attractive and active \\
  as a tongue, \\
  as a violin-bow playing.
  \caption{}
  \label{fig:en}
\end{subfigure}
\caption[Three pieces of written natural language.]{Three pieces of written natural language. The text on
  Figure~\ref{fig:lv} is the beginning of the poem ``Jaunkundze ar sunīti'' by
  Aleksandrs Čaks, Figure~\ref{fig:ru} is a translation to Russian by Lora Trin,
  and Figure~\ref{fig:en} is an English translation by Inara Cedrins.}
\end{figure}


On the other hand, a task that provides a list of associations with the text, an
essay or a painting inspired by it demands understanding of the text, language
knowledge. Luckily, nowadays these kind of problems are not expected or demanded
to be done by computers. People enjoy doing these kind of tasks themselves.

However, it is reasonable to ask a computer the following questions regarding
text meaning:
\begin{inparaenum}[a)]
\item What is the text on Figure~\ref{fig:lv} about?
\item What is the relation between the texts on Figure~\ref{fig:lv} and
  Figure~\ref{fig:ru}?
\item Are these texts identical?
\item Where did the meeting took place?
\item What poems are similar to this?
\end{inparaenum}

Text summarizaiton, machine translation, information extraction and retrieval
are just a few of many branches of computational linguistics that provide
methods for answering these questions. The questions above have a general
property: all of them are about the meaning of the text. Natural language
semantics is an area that studies meaning representation.

Creativity of natural languages---the fact that humans are able to produce and
understand sentences they have never came across---complicates meaning
modeling. Even if we had a way to map each word to its meaning, it is
impractical to apply the same procedure to sentences, because as we process a
piece of text most of the sentences in it will be seen for the fist
time. Because of this, we need to be able to build the meaning representation of
a sentence from its constituent parts: words.

Syntax is a study about the structure of a sentence. Grammars define the rules
that are describe how a sentences that belong to a language should look
like. For example, a subject is in front of a verb and an object is after it in
an English sentence. Having the constituent meaning representation, the meaning
of a sentence is built guided by its syntax.

To a first, high level approximation, to be able to deal with the meaning of a
text in natural language one needs to have meaning representation of
constituents, a view to the (syntactic) structure of the text and a
compositional procedure that outputs the meaning representation of the whole
text.

\todo[inline]{Finish introduction}


% This work concentrates on, studies and proposes and improvement for two points:
% meaning representation and composition procedure. Distributional hypothesis
% \cite{harris1954distributional} is used to produce word meaning and categorical
% composition \cite{coecke2010} is the composition mechanism. The goal of this
% work is application of these techniques to the tasks which complex language such
% as dialogue act tagging or paraphrase detection.

%%% Local Variables:
%%% mode: latex
%%% TeX-master: "thesis"
%%% End:
