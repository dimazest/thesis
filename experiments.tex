\chapter{Intrinsic experiments}
\label{cha:experiments}

\section{Similarity and relatedness of words}
\label{sec:lexical}

Table~\ref{tab:parameters} lists parameters and their values. As the source corpus we use the concatenation of Wackypedia and ukWaC \cite{ukwac} with a symmetric 5-word window \cite{milajevs-EtAl:2014:EMNLP2014}; the evaluation metric is the correlation with human judgements as is standard with SimLex \cite{hill2014simlex}.

% We derive our parameter selection heuristics by greedily selecting parameters (\texttt{cds}, \texttt{neg}) that lead to the highest average performance for each combination of frequency weighting, PMI variant and dimensionality $D$. Figures~\ref{fig:interaction-cds} and \ref{fig:interaction-neg} show the interaction of \texttt{cds} and \texttt{neg} with other parameters. We also vary the similarity measure (cosine and correlation  \cite{kiela-clark:2014:CVSC}), but do not report results here due to space limits.

\subsection{SimLex-999}
\label{sec:simlex-999}

\begin{wrapfigure}[6]{O}{0.5\textwidth}
  \vspace{-1em}
  \includegraphics[width=0.5\textwidth]{supplement/figures/SimLex999-results}
  \caption{SimLex-999 results}
  \label{fig:SimLex999-results}
\end{wrapfigure}

%%% Local Variables:
%%% mode: latex
%%% TeX-master: "../thesis"
%%% End:


\subsubsection{Max selection}
\label{sec:max-selection-simlex}

Figure~\ref{fig:SimLex999-results} illustrates the results based on the best model selection and Table~\ref{tab:parameters} shows the results together with picked parameters. Note that maximum selection is identical with cross-validation: they pick the same models.

In general model performance increases as the dimensionality increases. However, the best result of 0.389 is achieved with a 2000 dimensional space, this could be an example of overfitting. Model performance becomes changes for dimensions grater than 20000.

For spaces with dimensionality less than 5000 \texttt{freq} 1 and inner product yield best results. Otherwise, cosine with \logNSCPMI/, smoothing $\alpha=0.75$ and shifting $k=0.7$ gives the best results.

\begin{table}
  \centering

  \begin{tabular}{rrlllrl}
\toprule
 dimensionality &  SimLex999 &  freq &  discr &     cds &  neg &     similarity \\
\midrule
           1\,000 &      0.369 &     1 &   spmi &       1 &  0.2 &  inner\_product \\
           2\,000 &      0.389 &     1 &  scpmi &  global &  0.7 &  inner\_product \\
           3\,000 &      0.376 &     1 &   spmi &    0.75 &  0.2 &  inner\_product \\
           5\,000 &      0.363 &  logn &  scpmi &  global &  1.0 &            cos \\
          10\,000 &      0.371 &  logn &  scpmi &       1 &  0.7 &            cos \\
          20\,000 &      0.381 &  logn &  scpmi &    0.75 &  0.7 &            cos \\
          30\,000 &      0.383 &  logn &  scpmi &    0.75 &  0.7 &            cos \\
          40\,000 &      0.384 &  logn &  scpmi &    0.75 &  0.7 &            cos \\
          \textbf{50\,000} &      \textbf{0.385} &  \textbf{logn} &  \textbf{scpmi} &    \textbf{0.75} &  \textbf{0.7} &            \textbf{cos} \\
\bottomrule
\end{tabular}


  \caption[SimLex-999 Max selection]{SimLex-999 Max selection. Correlation values that are not statistically
    significant from the highest value of 0.389 are in bold. For SimLex-999,
    $\sigma^{0.9}_{0.05} = 0.023$ making the values greater than 0.366
    indistinguishable from the highest result.}
  \label{tab:Simlex999-max-selection}
\end{table}


\subsubsection{Heuristics}
\label{sec:heuristics-simlex}

\begin{wraptable}[8]{O}{0.5\textwidth}
  \vspace{-60pt}
  \centering

  \begin{tabular}{lr}
\toprule
      parameter &  partial $R^2$ \\
\midrule
     similarity &      0.379 \\
           freq &      0.268 \\
            neg &      0.241 \\
 dimensionality &      0.084 \\
          discr &      0.077 \\
            cds &      0.064 \\
\bottomrule
\end{tabular}


  \caption{SimLex-999 feature ablation.}
  \label{tab:SimLex999-ablation}
\end{wraptable}


% \begin{figure}

  \centering

  \begin{subfigure}[t]{0.49\textwidth}
    \includegraphics[width=\textwidth]{supplement/figures/SimLex999-interaction-similarity}
    \caption{similarity}
    \label{fig:SimLex999-interaction-similarity}
  \end{subfigure}
  \begin{subfigure}[t]{0.49\textwidth}
    \includegraphics[width=\textwidth]{supplement/figures/SimLex999-interaction-freq}
    \caption{\texttt{freq}}
    \label{fig:SimLex999-interaction-freq}
  \end{subfigure}

  \begin{subfigure}[t]{0.49\textwidth}
    \includegraphics[width=\textwidth]{supplement/figures/SimLex999-interaction-neg}
    \caption{\texttt{neg}}
    \label{fig:SimLex999-interaction-neg}
  \end{subfigure}
  \begin{subfigure}[t]{0.49\textwidth}
    \includegraphics[width=\textwidth]{supplement/figures/SimLex999-interaction-discr}
    \caption{\texttt{discr}}
    \label{fig:SimLex999-interaction-discr}
  \end{subfigure}

  \begin{subfigure}[t]{0.49\textwidth}
    \includegraphics[width=\textwidth]{supplement/figures/SimLex999-interaction-cds}
    \caption{\texttt{cds}}
    \label{fig:SimLex999-interaction-cds}
  \end{subfigure}

  \caption{SimLex-999 parameter interaction. Parameters are shown in the order of their influence.}
  \label{fig:SimLex999-interaction}
\end{figure}

%%% Local Variables:
%%% mode: latex
%%% TeX-master: "../thesis"
%%% End:


The linear model achieves an adjusted $R^2$ of 0.867, indicating that the model is able to predict model performance based on parameter selection quite well. Table~\ref{tab:SimLex999-ablation} shows partial $R^2$ scores for parameters. The most influential parameters in decreasing order are similarity, \texttt{freq} and \texttt{neg}.

% \begin{wrapfigure}{O}{0.5\textwidth}
\begin{figure}[h]
  % \vspace{-30pt}
  \centering

  \includegraphics[width=0.5\textwidth]{supplement/figures/SimLex999-interaction-similarity}

  \caption{SimLex-999 influence of the similarity measure.}
  \label{fig:SimLex999-similarity}
\end{figure}

Figure~\ref{fig:SimLex999-similarity} shows the average performance of similarity measures. Correlation outperforms all other measures for all dimensions and peaks at the dimensionality of 20000.

\begin{figure}[h]
% \begin{wrapfigure}{O}{0.5\textwidth}
  % \vspace{-30pt}
  \centering

  \includegraphics[width=0.5\textwidth]{supplement/figures/SimLex999-interaction-freq}

  \caption{SimLex-999 influence of \texttt{freq}.}
  \label{fig:SimLex999-freq}
\end{figure}

The influence of \texttt{freq}, the second parameter, is shown on Figure~\ref{fig:SimLex999-freq}. $\log n$ frequency outperforms other choices for all dimensions. After 20000 dimensions $\log n$'s performance stabilises.

\begin{figure}[h]
% \begin{wrapfigure}{O}{0.5\textwidth}
  % \vspace{-30pt}
  \centering

  \includegraphics[width=0.5\textwidth]{supplement/figures/SimLex999-interaction-neg}

  \caption{SimLex-999 influence of \texttt{neg}.}
  \label{fig:SimLex999-neg}
\end{figure}

The third parameter \texttt{neg} of 0.7 shows the best performance (Figure~\ref{fig:SimLex999-neg}). However, there is little difference between models with dimensionality grater than 20000, apart from the models that do not perform shifting, whose performance peaks at 20000 dimensions.

\begin{figure}[h]
% \begin{wrapfigure}{O}{0.5\textwidth}
  % \vspace{-30pt}
  \centering

  \includegraphics[width=0.5\textwidth]{supplement/figures/SimLex999-interaction-discr}

  \caption{SimLex-999 influence of \texttt{discr}. PMI and CPMI are not shown because at the step before models with shifting were chosen.}
  \label{fig:SimLex999-discr}
\end{figure}

There is little difference between SPMI and SCPMI performance with a little advantage to SCPMI (Figure~\ref{fig:SimLex999-discr}).

% \begin{figure}[h]
\begin{wrapfigure}[5]{O}{0.5\textwidth}
  \vspace{-30pt}
  \centering

  \includegraphics[width=0.5\textwidth]{supplement/figures/SimLex999-interaction-cds}
  \caption{SimLex-999 influence of \texttt{cds}.}
  \label{fig:SimLex999-cds}
\end{wrapfigure}

Finally, models benefit from context distribution smoothing, spaces with less than 10000 dimensions produce the best results with $\alpha = 1$, for spaces with higher dimensionality $\alpha = 0.75$ is the most advantageous (Figure~\ref{fig:SimLex999-cds}).

\todo[noline]{Contrast or compare results with \cite{milajevs:2016:SRW1}}

\paragraph{Difference with "Max" selection}

\begin{table}
  \centering

  \begin{tabular}{rrlllll}
\toprule
 dimensionality &  SimLex999 &  freq &  discr &   cds &  neg &   similarity \\
\midrule
           1\,000 &       0.33 &  logn &  scpmi &     1 &  0.7 &  correlation \\
           2\,000 &       0.35 &  logn &  scpmi &     1 &  0.7 &  correlation \\
           3\,000 &       0.35 &  logn &  scpmi &     1 &  0.7 &  correlation \\
           5\,000 &       0.35 &  logn &  scpmi &     1 &  0.7 &  correlation \\
          10\,000 &       0.37 &  logn &  scpmi &  0.75 &  0.7 &  correlation \\
          20\,000 &       \textbf{0.38} &  logn &  scpmi &  0.75 &  0.7 &  correlation \\
          30\,000 &       \textbf{0.38} &  logn &  scpmi &  0.75 &  0.7 &  correlation \\
          40\,000 &       \textbf{0.38} &  logn &  scpmi &  0.75 &  0.7 &  correlation \\
          50\,000 &       \textbf{0.38} &  logn &  scpmi &  0.75 &  0.7 &  correlation \\
\bottomrule
\end{tabular}


  \caption[SimLex-999 selection based on heuristics]{SimLex-999 selection based on heuristics. The highest value is 0.384.
  The values that are grater than 0.361 are indistinguishable from the highest score.}
  \label{tab:Simlex999-heuristics-selection}
\end{table}


As expected manual parameter selection is more stable as Table~\ref{tab:Simlex999-heuristics-selection} shows. Both selection models agree on parameters for highly dimensional spaces ($D \geq 2000$), with an exception of similarity: Max selection prefers cosine, while manual prefers correlation based similarity measure. Because of this, manual selection does not pick the best result of the 2000 dimensional model, but at the 50000 dimensions  a model selected manually scores 0.001 lower: 0.384 versus 0.385 as also seen on Figure~\ref{fig:SimLex999-results}.

The average relative difference between Max selection and heuristics is 0.039.

\subsection{MEN}
\label{sec:men}

\subsubsection{Max selection}
\label{sec:max-selection-men}

% \begin{figure}
\begin{wrapfigure}[7]{O}{0.5\textwidth}
  \vspace{-1em}
  \includegraphics[width=0.5\textwidth]{supplement/figures/men-results}
  \caption{MEN results}
  \label{fig:men-results}
\end{wrapfigure}

%%% Local Variables:
%%% mode: latex
%%% TeX-master: "../thesis"
%%% End:


Figure~\ref{fig:SimLex999-results} shows the selection results. Again, cross-validation results are identical with Max selection. Table~\ref{tab:men-max-selection} shows the results together with the selected models.

\begin{table}[b]
  \centering

  \begin{tabular}{rrlllrl}
\toprule
 dimensionality &    men &  freq &  discr &     cds &  neg &   similarity \\
\midrule
           1000 &  0.686 &     1 &  scpmi &  global &  1.4 &  correlation \\
           2000 &  0.728 &  logn &  scpmi &       1 &  0.7 &          cos \\
           3000 &  0.737 &  logn &  scpmi &       1 &  0.7 &          cos \\
           5000 &  0.743 &  logn &  scpmi &    0.75 &  0.7 &          cos \\
          10000 &  0.753 &  logn &  scpmi &    0.75 &  1.0 &  correlation \\
          20000 &  0.763 &  logn &  scpmi &    0.75 &  1.0 &  correlation \\
          30000 &  0.765 &  logn &  scpmi &    0.75 &  1.0 &  correlation \\
          40000 &  0.765 &  logn &  scpmi &    0.75 &  1.0 &  correlation \\
          50000 &  0.765 &  logn &  scpmi &    0.75 &  1.0 &  correlation \\
\bottomrule
\end{tabular}


  \caption{MEN Max selection}
  \label{tab:men-max-selection}
\end{table}


Model performance monotonically increases as the dimensionality increases. The highest score of 0.765 is achieved by 3 spaces with $D \geq 30000$, \logNSCPMI/, smoothed contest distribution ($\alpha = 0.75$), shifted PMI values ($k = 1$) and the similarity measure based on correlation.

In comparison with SimLex-999, models with ``more extreme'' parameters give better results. For example, $\alpha = 0.75$ is the best for models tested on SimLex-999 with dimensionality starting with 20000, while for models tested on MEN, this parameter choice is the best starting with 5000. Similar behaviour is observed for \texttt{neg} and similarity. For highly dimensional spaces the switch from SimLex-999 to MEN changes the best \texttt{neg} choice from 0.7 to 1 and similarity from cosine to correlation. Such a switch of parameter choices might suggest the difference between \textit{relatedness} and \textit{similarity}.

\subsubsection{Heuristics}
\label{sec:heuristics-men}

\begin{wraptable}[18]{O}{0.5\textwidth}
  \vspace{50pt}
  \centering

  \begin{tabular}{lr}
\toprule
      parameter &  partial $R^2$ \\
\midrule
            neg &  0.31 \\
           freq &  0.21 \\
     similarity &  0.18 \\
          discr &  0.12 \\
 dimensionality &  0.11 \\
            cds &  0.09 \\
\bottomrule
\end{tabular}


  \caption{MEN feature ablation.}
  \label{tab:men-ablation}
\end{wraptable}

The linear model give an adjusted $R2$ of 0.733, which is lover than on SimLex-999, but is still high. Table~\ref{tab:men-ablation} shows partial $R^2$ scores for the explored parameters. The most influential parameter is \texttt{neg}, followed by \texttt{freq} and similarity. This is different in the case of SimLex-999 where these parameters influence ``order'' is reversed.

\begin{figure}[h]
% \begin{wrapfigure}{O}{0.5\textwidth}
  % \vspace{-30pt}
  \centering

  \includegraphics[width=0.5\textwidth]{supplement/figures/men-interaction-neg}

  \caption{MEN influence of \texttt{neg}.}
  \label{fig:men-neg}
\end{figure}

\texttt{neg} with $k = 2$ is preferable for spaces with dimensionality of less than 20000, for spaces with more dimensions, $k = 5$ is more beneficial (Figure~\ref{fig:men-neg}).  We, however, expect that for spaces with more than 500000 dimensions even higher values should be preferred. This contrasts with the heuristics derived from SimLex-999, where single \texttt{neg} value of 0.7 is chosen.

\begin{figure}[h]
% \begin{wrapfigure}{O}{0.5\textwidth}
  % \vspace{-30pt}
  \centering

  \includegraphics[width=0.5\textwidth]{supplement/figures/men-interaction-freq}

  \caption{MEN influence of \texttt{freq}.}
  \label{fig:men-freq}
\end{figure}

Regarding the frequency component, $\log n$ outperforms all other choices (Figure~\ref{fig:men-freq}). It is exactly same choice as for heuristics based on SimLex-999.

\begin{figure}[b]
% \begin{wrapfigure}{O}{0.5\textwidth}
  % \vspace{-30pt}
  \centering
  \begin{subfigure}[t]{0.49\textwidth}
    \hspace{-20pt}
    \includegraphics[width=1.1\textwidth]{supplement/figures/men-interaction-similarity}

  \caption{similarity}
  \label{fig:men-similarity}
  \end{subfigure}
  \begin{subfigure}[t]{0.49\textwidth}
    \includegraphics[width=\textwidth]{supplement/figures/men-interaction-discr}

  \caption{\texttt{discr}}
  \label{fig:men-discr}
  \end{subfigure}

  \caption{MEN.}
\end{figure}

Correlation is the preferred similarity measure (Figure~\ref{fig:men-similarity}), again this is inline with the choice based on SimLex-999.

\begin{figure}[h]
% \begin{wrapfigure}{O}{0.5\textwidth}
  % \vspace{-30pt}
  \centering

  \includegraphics[width=0.5\textwidth]{supplement/figures/men-interaction-discr}

  \caption{MEN influence of \texttt{discr}.}
  \label{fig:men-discr}
\end{figure}

SPMI is the preferred discriminativeness (Figure~\ref{fig:men-discr}), however it is closely followed by CPMI and SCPMI. This contrasts with SimLex-999, where SCPMI is preferred, however in both cases the difference between the two discriminativeness choices is minimal.

\begin{figure}[b]
%\begin{wrapfigure}[9]{O}{0.5\textwidth}
  % \vspace{-30pt}
  \centering

  \includegraphics[width=0.5\textwidth]{supplement/figures/men-interaction-cds}

  \caption{MEN influence of \texttt{cds}.}
  \label{fig:men-cds}
\end{figure}

Global context probability gives on average higher results (Figure~\ref{fig:men-cds}), while SimLex-999 prefers context distribution smoothing.

\paragraph{Difference with Max selection on MEN}

\begin{table}[b]
  \centering

  \begin{tabular}{rrlllll}
\toprule
 dimensionality &    men &  freq & discr &     cds & neg &   similarity \\
\midrule
           1\,000 &  0.684 &  logn &  spmi &  global &   2 &  correlation \\
           2\,000 &  0.721 &  logn &  spmi &  global &   2 &  correlation \\
           3\,000 &  0.730 &  logn &  spmi &  global &   2 &  correlation \\
           5\,000 &  0.735 &  logn &  spmi &  global &   2 &  correlation \\
          10\,000 &  0.745 &  logn &  spmi &  global &   2 &  correlation \\
          20\,000 &  0.757 &  logn &  spmi &  global &   5 &  correlation \\
          30\,000 &  0.759 &  logn &  spmi &  global &   5 &  correlation \\
          40\,000 &  0.759 &  logn &  spmi &  global &   5 &  correlation \\
          50\,000 &  0.758 &  logn &  spmi &  global &   5 &  correlation \\
\bottomrule
\end{tabular}


  \caption{MEN selection based on heuristics}
  \label{tab:men-heuristics-selection}
\end{table}


The two selection procedures agree on fewer parameters than the ones bases on SimLex-999. Both agree on discrimination ($\log n$) and similarity score for spaces with dimensionality greater than 10000 (correlation). While SCPMI is chosen by Max selection, SPMI is preferred by the selection based on heuristics. In contrast to the Max selection, which chooses the models with context distribution smoothing, heuristics prefer models with global context probabilities. Also, heuristics pick models with higher shifting values $\alpha$ (2 and 5), in contrast to Max selection, where 0.7 and 1 are picked. Table~\ref{tab:men-heuristics-selection} summarises the parameter selection based on heuristics.

The average difference between Max selection and heuristics is 0.008.

\paragraph{Difference with heuristics based on SimLex-999}

Heuristics based on MEN agree with ones bases on SimLex-999 on two parameters: frequency ($\log n$) and similarity (correlation). The methods disagree on \texttt{discr} (SCPMI versus SPMI), context distribution (smoothed versus global) and shifting parameter, for which higher values for MEN are preferred.

\subsection{Universal parameter selection for lexical datasets}
\label{sec:Universal-lexical-param-selection}

% SimLex --> Men
% Men --> SimLex
% Universal, normalized!

% Idea: find common parameters and take average over one for which there is not agreement. Normalize the result and repeat the heuristics selection?

\section{Similarity of sentences}
\label{sec:sentential}

\subsection{KS14}
\label{sec:ks14}

 
\clearpage
\KOMAoptions{paper=A3,pagesize}
\recalctypearea

% \newgeometry{margin=1.5cm}

\begin{landscape}
\thispagestyle{empty} %% Remove header and footer.

\begin{figure}
  \centering

  \begin{subfigure}[t]{0.6\textwidth}
    \includegraphics[width=\textwidth]{supplement/figures/KS14-max_-selection-freq}
    \caption{Max. Freq.}
    \label{fig:}
  \end{subfigure}
  \begin{subfigure}[t]{0.6\textwidth}
    \includegraphics[width=\textwidth]{supplement/figures/KS14-cross_validation-selection-freq}
    \caption{CV. Freq.}
    \label{fig:}
  \end{subfigure}

  \begin{subfigure}[t]{0.6\textwidth}
    \includegraphics[width=\textwidth]{supplement/figures/KS14-max_-selection-neg}
    \caption{Max. Neg.}
    \label{fig:}
  \end{subfigure}
  \begin{subfigure}[t]{0.6\textwidth}
    \includegraphics[width=\textwidth]{supplement/figures/KS14-cross_validation-selection-neg}
    \caption{CV. Neg.}
    \label{fig:}
  \end{subfigure}

  \begin{subfigure}[t]{0.6\textwidth}
    \includegraphics[width=\textwidth]{supplement/figures/KS14-max_-selection-cds}
    \caption{Max. CDS.}
    \label{fig:}
  \end{subfigure}
  \begin{subfigure}[t]{0.6\textwidth}
    \includegraphics[width=\textwidth]{supplement/figures/KS14-cross_validation-selection-cds}
    \caption{CV. CDS.}
    \label{fig:}
  \end{subfigure}

  \begin{subfigure}[t]{0.6\textwidth}
    \includegraphics[width=\textwidth]{supplement/figures/KS14-max_-selection-similarity}
    \caption{Max. Sim.}
    \label{fig:}
  \end{subfigure}
  \begin{subfigure}[t]{0.6\textwidth}
    \includegraphics[width=\textwidth]{supplement/figures/KS14-cross_validation-selection-similarity}
    \caption{CV. Sim.}
    \label{fig:}
  \end{subfigure}

  \begin{subfigure}[t]{0.6\textwidth}
    \includegraphics[width=\textwidth]{supplement/figures/KS14-max_-selection-discr}
    \caption{Max. Discr.}
    \label{fig:}
  \end{subfigure}
  \begin{subfigure}[t]{0.6\textwidth}
    \includegraphics[width=\textwidth]{supplement/figures/KS14-cross_validation-selection-discr}
    \caption{CV. Discr.}
    \label{fig:}

  \end{subfigure}


  \caption{KS14 selection.}
  \label{fig:selection_ks14}
\end{figure}

\end{landscape}
\restoregeometry

\clearpage
\KOMAoptions{paper=A4,pagesize}
\recalctypearea



\subsection{GS11}
\label{sec:gs11}

\input{figures/selection-gs11}

% \subsection{GS12}
% \label{sec:gs12}

\subsection{Universal parameter selection for compositional datasets}
\label{sec:robust-param-comp-selecion}

\section{Universal parameter selection for lexical and compositional datasets}
\label{sec:universal-param-selection}



% \subsubsection{Tweaked IR evaluation}
% \label{sec:tweak-ir-eval}

% \todo[inline]{Discuss \cite{Milajevs:2015:IMN:2808194.2809448} and link to \ref{sec:phraserel}}

% \section{PhraseRel: relevance of sentences}
% \label{sec:sentential-relevance}

%%% Local Variables:
%%% mode: latex
%%% TeX-master: "thesis"
%%% End:
