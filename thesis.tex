\documentclass[11pt,a4paper,english,oneside]{book}
% * <sdyck@ualberta.ca> 2016-11-02T01:22:15.568Z:
%
% ^.
\usepackage{url}
\usepackage{latexsym}

\usepackage[onehalfspacing]{setspace}
\usepackage{pdflscape}
\usepackage[top=1in,bottom=1in,left=1.5in,right=1in]{geometry}

\usepackage{fontspec}
\defaultfontfeatures{Ligatures=TeX}
\usepackage{sectsty}
\allsectionsfont{\normalfont\sffamily}
% \paragraphfont{\bfseries}
\setmainfont[
BoldFont=texgyrepagella-bold.otf,
ItalicFont=texgyrepagella-italic.otf,
BoldItalicFont=texgyrepagella-bolditalic.otf]
{texgyrepagella-regular.otf}

\usepackage{esvect}
%\usepackage{mathtools}
%\usepackage[libertine]{newtxmath}

\usepackage{titlesec}
\usepackage{titling}
\newfontfamily\headingfont[]{Transport-Heavy.ttf}
\renewcommand{\maketitlehooka}{\headingfont}

\setsansfont{Transport-Heavy.ttf}

\usepackage[round,sort]{natbib}

\usepackage[font=small,labelfont=bf]{caption}
\usepackage{subcaption}
\usepackage{booktabs}
\usepackage{amsmath}
\usepackage{multirow}
\usepackage{paralist}
\usepackage[parfill]{parskip}
\usepackage[colorinlistoftodos,prependcaption,textsize=footnotesize]{todonotes}
\usepackage{wrapfig}
\usepackage{rotating}
\usepackage{csquotes}

\renewcommand{\mkcitation}[1]{ #1}

\usepackage{tikz}
\usepackage{tikz-qtree}

\usepackage{listings}
\usepackage{color}
\usepackage{textcomp}
\definecolor{listinggray}{gray}{0.9}
\definecolor{lbcolor}{rgb}{0.9,0.9,0.9}
\lstset{
  captionpos=b,
  basicstyle=\ttfamily,
  keywordstyle=\color[RGB]{66, 54, 122},
}

% \clubpenalty=10000
% \widowpenalty = 10000

\usepackage[pdfusetitle]{hyperref}
\def\UrlBreaks{\do\/\do-}

\usepackage{ntheorem}
\theoremseparator{:}
\newtheorem{hyp}{Hypothesis}

\setlength\theorempreskipamount{\baselineskip}
% \setlength\theorempostskipamount{1ex}

% \title{Categorical Composition Models in Large Scale NLP Tasks}
% \title{Bringing Semantics to NLP Tasks Using Categorical Composition}
% \title{Scaling Up Distributional Semantics to Real World NLP Tasks Using Categorical Composition}
% \title{The Role of Computers in Understanding Natural Language}
% \title{What Computers Should Know about Texts}
% \title{Robust methodology for measuring word and sentence similarity and relevance with distributional semantic models}
% \title{Robust methodology for measuring word and sentence similarity with distributional semantic models}
% \title{Methodology for Robust Selection of Distributional Models of Word and Sentence  Similarity}
\title{Methodology for Selection of Distributional Models to Measure Relationships between Words and Sentences}
\author{Dmitrijs Milajevs}
% \affil{Queen Mary University of London}

% http://colorschemedesigner.com/#4o42FfqublRMS
\definecolor{linkcolor}{RGB}{66, 54, 122}
\definecolor{citecolor}{RGB}{84, 141, 100}
\definecolor{urlcolor}{RGB}{168, 70, 67}
\hypersetup{
  colorlinks=true,
  linkcolor=linkcolor,
  citecolor=citecolor,
  urlcolor=urlcolor,
}

\bibpunct{\textcolor{citecolor}{(}}{\textcolor{citecolor}{)}}{\textcolor{citecolor}{,}}{\textcolor{citecolor}{a}}{}{\textcolor{citecolor}{;}}

\sloppy
\widowpenalty=500
\clubpenalty=500

\DeclareMathOperator{\my-c}{count}
\newcommand{\ov}{\vv}

\newcommand\newcite\citet
\renewcommand\cite\citep

\def\PMI/{$\operatorname{1PMI}$}
\def\SPMI/{$\operatorname{1SPMI}$}
\def\CPMI/{$\operatorname{1CPMI}$}
\def\SCPMI/{$\operatorname{1SCPMI}$}

\def\NPMI/{$\operatorname{nPMI}$}
\def\NSPMI/{$\operatorname{nSPMI}$}
\def\NCPMI/{$\operatorname{nCPMI}$}
\def\NSCPMI/{$\operatorname{nSCPMI}$}

\def\logNPMI/{$\operatorname{lognPMI}$}
\def\logNSPMI/{$\operatorname{lognSPMI}$}
\def\logNCPMI/{$\operatorname{lognCPMI}$}
\def\logNSCPMI/{$\operatorname{lognSCPMI}$}

\newcommand{\BASEURL}{http://www.eecs.qmul.ac.uk/~dm303/thesis}
% \newcommand{\BASEURL}{http://localhost:8000/thesis}
\newcommand{\dataurl}[1]{\href{\BASEURL/#1}{\nolinkurl{#1}}}

\def\emnlp/{KS14}

\patchcmd{\titlepage}
{\thispagestyle{empty}}
{\thispagestyle{plain}}
{}
{}

\begin{document}
\maketitle
\addtocounter{page}{1}

\chapter*{Statement of originality}

I, Dmitrijs Milajevs, confirm that the research included within this thesis is my own work or that where it has been carried out in collaboration with or supported by others, that this is duly acknowledged below and my contribution indicated. Previously published material is also acknowledged below.

I attest that I have exercised reasonable care to ensure that the work is original and does not to the best of my knowledge break any UK law, infringe any third party’s copyright or other Intellectual Property Right, or contain any confidential material.

I accept that the College has the right to use plagiarism detection software to check the electronic version of this thesis.

I confirm that this thesis has not been previously submitted for the award of a degree by this or any other university.

The copyright of this thesis rests with the author and no quotation from it or information derived from it may be published without the prior written consent of the author.

% Signature: \includegraphics{signature.png}

\today

\chapter*{Details of collaboration and publications}

Content from the following publications appears in this thesis, which has been written with the guidance of my supervisors Mehrnoosh Sadrzadeh and Matthew Purver. Some of the publications were written in collaboration with Dimitri Kartsaklis, Thomas Roelleke and Sascha Griffiths. This thesis was proof read for the purposes of spelling and grammar by Sara Dyck.

\begin{enumerate}
\item \citet*{milajevs-purver:2014:CVSC}. Main author. Investigating the
  Contribution of Distributional Semantic Information for Dialogue Act
  Classification. In \textit{Proceedings of the 2nd Workshop on Continuous Vector Space Models and their Compositionality (CVSC)}.
\item \citet*{milajevs-EtAl:2014:EMNLP2014}. Main author. Evaluating Neural Word
  Representations in Tensor-Based Compositional Settings. \textit{In Proceedings of the 2014 Conference on Empirical Methods in Natural Language Processing (EMNLP)}.
\item \citet*{Milajevs:2015:IMN:2808194.2809448}. Main author. IR Meets NLP: On
  the Semantic Similarity Between Subject-Verb-Object Phrases. \textit{In Proceedings of the 2015 International Conference on Theory of Information Retrieval}.
\item \citet*{milajevs-sadrzadeh-purver:2016:ACL-SRW}. Main author. Robust Co-occurrence Quantification for Lexical Distributional Semantics. \textit{In Proceedings of the ACL 2016 Student Research Workshop}.
\item \citet*{milajevs-griffiths:2016:repeval}. Main author. A Proposal for Linguistic Similarity Datasets Based on Commonality Lists. \textit{In Proceedings of the 1st Workshop on Evaluating Vector-Space Representations for NLP}.
\end{enumerate}

\chapter*{Software}

The following software was developed as a part of this thesis.
\begin{itemize}
\item \href{http://fowlercorpora.readthedocs.io/en/latest/}{Fowler.corpora} is an implementation of distributional lexical and sentential similarity models.
\item \href{https://pypi.python.org/pypi/google-ngram-downloader}{Google-ngram-downloader} is an on-the-fly reader of the \href{http://storage.googleapis.com/books/ngrams/books/datasetsv2.html}{Google Books ngram dataset}.
\item \href{http://www.nltk.org/}{NLTK} was extended in several ways. The \href{https://github.com/nltk/nltk/blob/develop/nltk/corpus/reader/bnc.py}{BNC reader} was extended to support the full BNC edition. \href{https://github.com/nltk/nltk/blob/develop/nltk/wsd.py}{Lesk's word sense disambiguation algorithm} was extended to support part of speech tags. The \href{https://github.com/nltk/nltk/blob/develop/nltk/parse/dependencygraph.py}{Dependency Graph} datastructure was refactored to support custom dependency graph construction. An \href{https://github.com/nltk/nltk/pull/1249}{interface} to the \href{http://stanfordnlp.github.io/CoreNLP/}{Stanford CoreNLP} web API was added to the NLTK.
\end{itemize}

%%% Local Variables:
%%% mode: latex
%%% TeX-master: "thesis"
%%% End:

\cleardoublepage

\vspace*{\fill}
\begin{center}
% \begin{minipage}{\textwidth}
This work was supported by the EPSRC grant EP/J002607/1.
% \end{minipage}
\end{center}
\vfill


{\Large \headingfont \thetitle}

\vspace{1em}

{\large \headingfont \theauthor}

\vspace{1em}

{\headingfont Abstract}

Representation of sentences that captures semantics is an essential part of natural language processing systems, such as information retrieval or machine translation. The representation of a sentence is commonly built by combining together the representations of words that the sentence consists of. Similarity between words is widely used as a proxy to evaluate semantic representations. The word similarity models are well studied and are shown to correlate with human judgements.

Current evaluation of models of sentential similarity builds on the results obtained in lexical experiments. The main focus is how the lexical representations are used, rather than what they should be. It is often assumed that the optimal representations for word similarity are also optimal for sentence similarity. This work drops this assumption and systematically looks for lexical representations that are optimal for similarity measurement.

We find that the best representation for words similarity is not always the best for sentence similarity and vice versa. The best models in word similarity task perform best with additive composition. However, the best result on compositional tasks is achieved with Kronecker-based composition. There are representations that are equally good in both tasks when used with multiplicative composition.

The systematic study of the parameters of similarity models reveals that the more information lexical representations contain, the more attention should be paid to noise. In particular, the word vectors in models with the feature size at the magnitude of the vocabulary size should be sparse, but if a small number of context features is used then the vectors should be dense.

Given the right lexical representations, compositional operators achieve state-of-the-art performance improving over models that use neural word embeddings. To avoid oferfitting, either several test datasets should be used, or parameter selection should be based on parameters' average behaviour.

\vfill

%Submitted in partial fulfillment of the requirements of the Degree of Doctor of Philosophy

Submitted for the degree of Doctor of Philosophy

Queen Mary University of London

\today


%%% Local Variables:
%%% mode: latex
%%% TeX-master: "thesis"
%%% End:

\cleardoublepage

% \setcounter{tocdepth}{3}
\tableofcontents

\cleardoublepage
% \phantomsection
% \addcontentsline{toc}{chapter}{List of Figures}
\listoffigures

\cleardoublepage
% \phantomsection
% \addcontentsline{toc}{chapter}{List of Tables}
\listoftables

% \cleardoublepage
% \listoftodos

\chapter{Introduction}
\label{ch:introduction}


Computers require specially designed programming languages to be controlled, despite the fact that they play a crucial role in our lives. Ideally, it would be perfect if the interaction with computers was not different to the interaction between humans. Computational linguistics is one of the fields that addresses this problem.

In order to be controlled by people, or be able to assist people in language related tasks, computers need to understand it. However, different tasks need various level of language understanding. For instance, even if one does not recognize or know the language of a piece of text in Figure~\ref{fig:lv}, one can tell how many words there are, and that there is only one sentence. After a while, one can even say that this is probably a piece of poetry.

The conclusions above require neither the complete understanding of the language nor the meaning of the text. Knowledge that texts---at least in some languages---consist of words separated by a space and how poems are written is enough. Moreover, knowing the letter distribution across all human languages or having a list of words in them, one would conclude that the text is in Latvian. This information can be provided without knowing what the text is about.

\begin{figure}[t]
\tiny
\begin{subfigure}[t]{0.30\textwidth}
  Jaunkundze ar sunīti \\

  Un Vecrīgas šķērsielā, šaurā \\
  kā vēstuļu kastītes sprauga, \\
  kur troksnim un burzmai tik atbalss, \\
  kur smaržo pēc darvas, \\ ${}$\qquad dzelzs un pēc āboliem pagrabos sausos, \\
  es satiku jaunkundzi -- \\
  glītu un veiklu kā mēle, \\
  kā spēlējot vijoles lociņš.
  \vspace{0.74cm}
  \caption{}
  \label{fig:lv}
\end{subfigure}
~
\begin{subfigure}[t]{0.36\textwidth}
  Барышня с собачкой \\

  В Старой Риге, на улице поперечной, узкой, \\
  как щель в почтовый ящик, \\
  в который проникают только отголоски шума, гама, \\
  где запах дёгтя, ржавчины и яблок в сухих  подвалах, \\
  я встретил барышню -- \\
  красива и ловка - она - язык, \\
  смычок, играющий на скрипке.
  \vspace{0.99cm}
  \caption{}
  \label{fig:ru}
\end{subfigure}
~
\begin{subfigure}[t]{0.26\textwidth}
  Young Woman with a Dog \\

  On a narrow side-street in Riga’s old quarter, \\
  as though in a mailbox slot \\
  where noise and hustle only echo, \\
  and it smells of tar and steel \\
  and apples kept in dry basements, \\

  I met a young woman \\
  attractive and active \\
  as a tongue, \\
  as a violin-bow playing.
  \vspace{0.22cm}
  \caption{}
  \label{fig:en}
\end{subfigure}
\caption[Three pieces of written natural language]{Three pieces of written natural language. The text in
  Figure~\ref{fig:lv} is the beginning of the poem ``Jaunkundze ar sunīti'' by
  Aleksandrs Čaks \citeyearpar{čaks1996dzejas}, Figure~\ref{fig:ru} is a Russian
  translation by Lora Trin (the text is available at \url{http://grafomanam.net/works/34812}),
  and Figure~\ref{fig:en} is an English translation by Inara Cedrins \citeyearpar{cedrins}.}
\end{figure}


On the other hand, a task that asks for a list of associations with the text, essay or a painting inspired by it demands a much better understanding of the text that requires a deeper knowledge of the language and greater familiarity with the culture. Luckily, nowadays these kind of problems are not expected to be done by computers in day-to-day life, because people enjoy doing these kind of tasks themselves.

However, it is reasonable to ask a computer the following questions regarding the text:
\begin{inparaenum}[a)]
\item What is the text of Figure~\ref{fig:lv} about?
\item What is the relationship between the texts in Figure~\ref{fig:lv} and
  Figure~\ref{fig:ru}?
% the *meanings are identical*
\item Is the content similar or identical?
\item Where did the meeting take place?
\item What poems are similar to this?
\end{inparaenum}

Text summarizaiton, machine translation, information extraction and information retrieval are branches of computational linguistics that provide methods for answering these questions. The questions above have a general property: all of them are about the meaning of the text. Natural language semantics is an area that studies meaning and thus is necessary to solve the tasks.

Creativity of natural languages---the fact that humans are able to produce and understand sentences they have never came across---complicates meaning modeling. Even if we had a way to map each word to its meaning, it is impractical to apply the same procedure to sentences, because as we process a piece of text most of the sentences in it will be seen for the fist time. Because of this, we need to be able to build the meaning of a sentence from its constituent parts.

Syntax is a study about the structure of a sentence. Grammars define the rules that describe how a sentences that belong to a language should look like. For example, a subject is written before a verb and an object is after in an English sentence. Having the constituent meaning representations, the meaning of a sentence is built by following the syntactic structure of it.

To be able to deal with the meaning of a text in a natural language one needs to have meaning representation of constituents, the syntactic structure and a compositional procedure that outputs the meaning representation.

Recently, lexical distributional semantics \cite{BullinariaLevy2012,Bullinaria2007,Turney:2010:FMV:1861751.1861756} has been advanced after the success of word2vec \cite{mikolov2013linguistic,mikolov2013distributed,mikolov2013efficient}. At the same time, developments in compositional distributional semantics \cite{mitchell2010composition,maillard-clark-grefenstette:2014:TTNLS,Grefenstette:2011:ESC:2145432.2145580,Grefenstette:2011:ETV:2140490.2140497,kartsadrqpl2014,fried-polajnar-clark:2015:ACL-IJCNLP} showed that Frege's principle of compositionality \cite{Janssen2001} is useful in obtaining representations of phrases and sentences.

Even though the new generation of meaning representation models is shown to be superior in compositional tasks \cite{milajevs-EtAl:2014:EMNLP2014}, it has been criticised most notably by \newcite{TACL570}. However, it is not known whether the critique of word2vec also applies to the compositional setting.

This work adopts the recommendations of \newcite{TACL570} for model parameter tuning and performs a large-scale model selection study on lexical and compositional tasks.

The experiments show that after a careful selection of a vector space model parameters, the selected models are competitive with the current state of the art results and on some datasets outperform it.

%%% Local Variables:
%%% mode: latex
%%% TeX-master: "thesis"
%%% End:

\chapter{Background: distributional models of meaning}
\label{cha:background}

\section{The notion of similarity}
\label{sec:similarity}

Similarity is the degree of resemblance between two objects or events \cite{WCS:WCS1282} and plays a crucial role in psychological theories of knowledge and behaviour, where it is used to explain such phenomena as classification and conceptualisation \cite{Tversky1977,1986-13502-00119860101,medin1993respects,Markman1996,hahn1997concepts}. \textit{Fruit} is a \emph{category} because it is a practical generalisation. Fruits are sweet and generally constitute desserts, so when one is presented with an unseen fruit, one can hypothesise that it is served toward the end of a dinner.

Generalisations are extremely powerful in describing a language, as well. The verb \textit{runs} requires its subject to be singular. \textit{Verb}, \textit{subject} and \textit{singular} are categories that are used to describe English grammar. When one encounters an unknown word and is told that it is a verb, one will immediately have an idea about how to use it, assuming that it is used similarly to other English verbs.

From a computational perspective, this motivates and guides the development of \emph{semantic similarity} (in the sense that they measure the degree of resemblance of meaning) components that are embedded into natural language processing systems.

In Information Retrieval (IR), queries are expanded with related terms to increase the number of retrieved relevant documents. For example, if a user issues the query \textit{lakes in Sweden}, the system might add related words to the query such as \textit{lake}, \textit{reservoir}, \textit{river} or even \textit{swim} so that the documents that do not contain the word \textit{lakes} are retrieved \cite{Xu:1996:QEU:243199.243202}.
% \cite{Salton:1975:VSM:361219.361220}

A dependency parser might benefit from a generalisation about the part of speech tag of a word which did not occur in the training data, based on its occurrence pattern in a large corpus of documents from the web \cite{hermann-blunsom:2013:ACL2013,andreas-klein:2014:P14-2}.

The information that the word \textit{carpet} is similar in meaning to the word \textit{mat} might be exploited by a language model in estimating the probability of the sentence \textit{the cat set on the carpet}, even if it did not occur in the corpus, but \textit{cat sat on the mat} did \cite{bengio2006}.

A dialogue act tagging system might require classification of an utterance based on its role in a dialogue, such as a question or an acknowledgement \cite{kalchbrenner-blunsom:2013:CVSC}.

The examples show that similarity is a broad term that is task-dependent. An IR system needs to identify semantically similar (\textit{lake}, \textit{river}) and related (\textit{lake}, \textit{swim}) terms. A dependency parser benefits from the similarity of word usage. A language model exploits similarity in word meaning. A dialogue act tagging system relies on the role similarity of the utterances. Also different linguistic entities are compared, the similarity can be measured between pairs of words and pairs of sentences.

\section{Representation for similarity measurement}
\label{sec:word-meaning}

According to \newcite{WCS:WCS1282}, ``similarity is an essentially psychological notion, based on the way we represent objects, that is, the way they appear to us.'' Since it is not yet known how objects are represented in human mind, the computational way of representing meaning to measure semantic similarity has to be agreed upon. However, one needs to be extremely careful when the meaning representation is decided, as it is unavoidably connected to the \emph{meaning of words in isolation}.

% The semantic formalisation of similarity is based on two ideas. The occurrence pattern of a word \emph{defines} its meaning \cite{firth1957lingtheory}, while the difference in occurrence between two words \emph{quantifies} the difference in their meaning \cite{harris1954distributional}.

Frege discusses two conflicting principles of meaning \cite{Janssen2001}. According to \emph{the principle of compositionality}, isolated word meanings are the building blocks of sentence meanings:
\begin{displayquote}[\cite{Janssen2001}]
The meaning of a compound expression is a function of the meaning of its parts and the syntactic rule by which they are combined.
\end{displayquote}
%
However, according to \emph{the principle of contextuality}, the word meaning in isolation is not defined:
\begin{displayquote}[\cite{Janssen2001}]
Never ask for the meaning of a word in isolation, but only in the context of a sentence.
\end{displayquote}

It is worth noting here that similarity in isolation is also problematic because the number of features an entity has is infinite, and it is easy to show that two entities will always have an infinite amount of common features \cite{goodman1972problems,hahn1997concepts}. For example, the tree next to my house is similar to my house because both of them are less than one kilometre in height, both of them are less than two kilometre in height, both of them are less than three kilometre in height and so on. 

To make similarity measurement possible, it has to be measured \emph{under a given description} \cite{WCS:WCS1282,medin1993respects,Markman1996}, thus, similarity is always contextualised. In other words, similarity emerges only when the possible properties are weighted. In our example, the tree and the house are similar with respect to the colour: both of them are green (the height properties are assigned zero weight).
% On the other side, \newcite{Huth2016} were able to comprehensively map individual words across cortex, meaning that there are word representations supporting the idea of \newcite{WCS:WCS1282}.

Frege's principle of contextuality allows us to define the meaning of a word by identifying its contribution to the meaning of a sentence. Firth's \citeyearpar{firth1957lingtheory} famous quote that ``you shall know a word by the company it keeps,'' suggests that the word meaning can be \emph{modelled} as the combination of the meanings of its occurrences in sentences of a corpus. Note that this does not provide the absolute word meaning, but only its meaning relative to the corpus. This assumption is also supported by the distributional hypothesis of \newcite{harris1954distributional} that the differences of occurrences of two words quantify the difference in their relative meaning, but do not necessarily define the meaning.

Once the relative word meaning is accepted, compositionality can be used to obtain representations of phrases and sentences \cite{THEO:THEO373,Dowty1980,sep-montague-semantics,DBLP:journals/corr/abs-1003-4394,baroni2014frege}.

\section{Word meaning}
\label{sec:distr-hypoth}

% In principle, we would like to capture the intuition that while \textit{John} and \textit{Mary} are distinct entities, they are rather similar to each other (both of them are humans) and are dissimilar to \textit{dog}, \textit{pavement} or \textit{idea}. However, we start with the word meaning representation that captures the fact that entities are distinct, but does not provide the means to measure similarity.

% The same applies at the phrase and sentence level: \textit{dogs chase cats} is similar in meaning to \textit{hounds pursue kittens}, but less so to \textit{cats chase dogs} (despite the lexical overlap).

\subsection{Formal semantics}
\label{sec:classical-approaches}

Formal semantics provides the means to infer some piece of information from another. The main studied relation is the entailment of sentences, for example, \textit{John swims in Åresjön} entails \textit{John swims in a Swedish lake}. To evaluate entailment, the sentences are converted to formulas. The words correspond to symbols in formal logic.

The individual \textit{Åresjön} corresponds to the symbol \textit{Åresjön'}, which is mapped to the actual lake by the interpretation function $\mathcal{I}$.

One-place properties are seen as sets of individuals, so $\mathcal{I}(\mathit{Swedish'})$ is a set that contains $\mathcal{I}(\mathit{Åresjön'})$ and $\mathcal{I}(\mathit{Väsman'})$ among many other entities.

\textit{Swim'} is a two-place predicate that is represented as a set that contains the pairs between which the relation holds, so if John actually swims in Åresjön, then $\mathcal{I}(\mathit{swim'})$ will contain the pair $(\mathcal{I}(\mathit{John'}), \mathcal{I}(\mathit{Åresjön'}))$.

While such formalism is very powerful for entailment detection between sentences, similarity measurement is problematic, because there is no relation between atomic symbols: we only know that \textit{Åresjön} and \textit{Väsman} correspond to different entities in the universe, but know nothing about their properties.

\subsection{Distributional semantics}
\label{sec:distr-repr}

Distributional methods provide a way to measure similarity between symbols. The representations are produced by directly exploiting Harris' \citeyearpar{harris1954distributional} intuition that similar words occur in similar contexts.

For example, one can construct a vector space in which the dimensions correspond to contexts, which are usually other words. The word vector components can then be calculated by taking the frequency with which the word co-occurred with the corresponding contexts within a predefined window in a corpus of interest. The similarity in meaning can be expressed via a suitable distance metric within the space.

\begin{wraptable}[10]{O}{7cm}
  \centering
  %\vspace{-1em}
  \begin{tabular}{lrrr}
    \toprule
    & philosophy & book & school \\
    \midrule
    John & 4  & 60 & 59  \\
    Mary & 0  & 10 & 22  \\
    girl & 0  & 19 & 93  \\
    boy  & 0  & 12 & 146 \\
    idea & 10 & 47 & 39  \\
    \bottomrule
  \end{tabular}
  \caption{Word co-occurrence frequencies extracted from the BNC}
  \label{tab:comparison}
\end{wraptable}

Table~\ref{tab:comparison} shows five three-dimensional vectors for words \textit{Mary}, \textit{John}, \textit{girl}, \textit{boy} and \textit{idea}. These are \textit{target words}.  The words \textit{philosophy}, \textit{book} and \textit{school} label vector space dimensions and are referred to as \emph{context words}.

As the vector for \textit{Mary} is closer to \textit{girl} than it is to \textit{boy} in the vector space, we can say that \textit{Mary}'s contexts are similar to \textit{girl}'s (and less similar to \textit{boy}'s), therefore \textit{Mary} is semantically more similar to \textit{girl} than to \textit{boy}.

Mathematically, the similarity can be expressed using, for instance, the cosine of the angle between two vectors:
%
\begin{align*}
\cos(\theta) &=
\frac{\ov{\mathit{Mary}}\cdot\ov{\mathit{girl}}}
{||\ov{\mathit{Mary}}||||\ov{\mathit{girl}}||} =
%
\frac{(0\times0) + (10\times19) + (22\times93)}
% * <sdyck@ualberta.ca> 2016-11-03T02:13:08.483Z:
%
% > {(0\times0) + (10\times19) + (22\times93)
%
% adding parentheses makes it easier to tell where the numbers came from, but feel free to remove them if you want/disagree. 
%
% ^.
{\sqrt{0^2 + 10^2 + 22^2}\sqrt{0^2 + 19^2 + 93^2}} \approx
\frac{2236}{2294} \approx 0.975
 \\
\cos(\phi) &=
\frac{\ov{\mathit{Mary}}\cdot\ov{\mathit{boy}}}
{||\ov{\mathit{Mary}}||||\ov{\mathit{boy}}||} =
%
\frac{(0\times0) + (10\times12) + (22\times146)}
{\sqrt{0^2 + 10^2 + 22^2}\sqrt{0^2 + 12^2 + 146^2}} \approx
\frac{3332}{3540} \approx 0.941
\end{align*}
%
where $\theta$ is the angle between the vectors of \textit{Mary} and \textit{girl}; and $\phi$ is the angle between the vectors of \textit{Mary} and \textit{boy}.

In the current example of a na{\"\i}ve vector space, \textit{John} is also closer to \textit{girl} than to \textit{boy}, which is counter-intuitive. This might be because of the small number of dimensions used, the poor selection of the context words, or the usage of raw co-occurrence numbers. The importance of parameters is discussed in details in Section~\ref{sec:parameter-selection-intro}.

\subsection{Neural word embeddings}
\label{sec:neural-embedding}

Deep learning techniques use the distributional hypothesis differently. Instead of relying on observed co-occurrence frequencies, a neural model is trained to maximise some objective function related to, for example, the probability of observing the surrounding words in some context \cite{mikolov2013distributed}:
% * <sdyck@ualberta.ca> 2016-11-03T02:27:02.618Z:
%
% > differently
%
% differently than what? co-occurrence​ based vectors?
%
% ^.
%
\begin{align}
 \frac{1}{T}\sum^{T}_{t=1}\sum_{-c \leq j \leq c, j\neq0} \log p(w_{t+j}|w_t)
  \label{eq:objective-func}
\end{align}
%
\noindent
Maximising this function produces vectors which maximise the
conditional probability of observing words in a context around the
target word $w_t$, where $c$ is the size of the training context, and
$w_1 w_2, \cdots w_T$ is a sequence of training words. Therefore, they
capture the distributional intuition and can express degrees of
lexical similarity.

They have also been proved to be successful at other tasks \cite{mikolov2013linguistic}. The vectors obtained by this method encode not only attributional similarity (similar words are close to each other), but also relational similarities \cite{Turney:2010:FMV:1861751.1861756}. For example, it is possible to extract the \texttt{singular:plural} relation (\textit{apple}:\textit{apples}, \textit{car}:\textit{cars}) using vector subtraction:
%
\begin{align*}
  \overrightarrow{\mathit{apple}} - \overrightarrow{\mathit{apples}}
  \approx
  \overrightarrow{\mathit{car}} - \overrightarrow{\mathit{cars}}
\end{align*}
%
also, semantic relationships are preserved:
%
\begin{align*}
  \overrightarrow{\mathit{king}} - \overrightarrow{\mathit{man}}
  \approx
  \overrightarrow{\mathit{queen}} - \overrightarrow{\mathit{woman}}
\end{align*}
%
allowing the formation of analogy queries similar to
$\overrightarrow{\mathit{king}} - \overrightarrow{\mathit{man}} +
\overrightarrow{\mathit{woman}} = \mathtt{?}$, obtaining
$\overrightarrow{\mathit{queen}}$ as the
result.

\newcite{levy2014linguistic} refined the method of retrieving relational similarities by changing the objective function and improved the state-of-the-art results, both for neural embeddings and co-occurrence based vectors. However, recently the evaluation method based on analogies and vector offsets has been criticised---see \newcite{linzen:2016:RepEval}.

\section{Similarity of phrases and sentences}
\label{sec:similarity-compounds}

Both neural and co-occurrence-based approaches have advantages over the formal approaches in their ability to capture lexical semantics and degrees of similarity. However, their success at extending this to the sentence level, and to more complex semantic phenomena, depends on their applicability within compositional models.

\subsection{The principle of compositionality}
\label{sec:formal-semantics}

Formal approaches to the semantics of natural language have built upon the
classical idea of compositionality which states that the meaning of a sentence is a
function of its parts \cite{Janssen2001}. In compositional type-logical
approaches, predicate-argument structures representing phrases and sentences are
built from their constituent parts by general operations such as beta-reduction
within the lambda calculus \cite{THEO:THEO373}: for example, given a semantic
representation of \emph{John} as $\mathit{John}'$, \emph{loves} as
$\lambda y.\lambda x.\mathit{loves}'(x, y)$ and \emph{Mary} as $\mathit{Mary'}$, the sentence \emph{John loves Mary}
can be constructed as
$$
\lambda y.\lambda
x.\mathit{loves}'(x, y)(\mathit{mary}')(\mathit{john}') =
\mathit{loves}'(\mathit{john}', \mathit{mary}')
$$

To get the semantic representation of the sentence \textit{John loves Mary}, we
need to do the following. Syntactic rules define how constituents are combined
% * <sdyck@ualberta.ca> 2016-11-03T03:32:46.899Z:
%
% > need to do the following
%
% What needs to be done? This doesn't fit with how the rest of the paragraph is phrased.
%
% ^.
to form other constituents (and finally a sentence). Translation rules define
how semantic representations of the constituents are combined to get a semantic
representation of the whole.

Categorical grammars are widely used to obtain the syntactic structure of a sentence. Given a set of basic categories $\texttt{ATOM}$, for example $\{\mathit{n}, \mathit{s}, \mathit{np}\}$ complex categories
% * <sdyck@ualberta.ca> 2016-11-05T01:33:12.755Z:
%
% > xample $\{\mathit{n}, \mathit{s}, \mathit{np}\}$ complex categories
%
% what do these categories mean?
%
% ^.
$\mathtt{CAT} \backslash \mathtt{CAT}$ and $\mathtt{CAT}/\mathtt{CAT}$ can be constructed, where $\mathtt{CAT}$ is either an element of \texttt{\texttt{ATOM}} or a complex category. So the transitive verb category is $\mathtt{np}\backslash\mathtt{s}/\mathtt{np}$. Intuitively, we want to say that to obtain a sentence with a transitive verb there must be noun phrases before and after the verb.

Parsing is done by composing categories together according to two rules:
%
\begin{enumerate}
\item \textbf{Backward application}: If $\alpha$ is a string of category $A$ and
  $\beta$ is a string of category $A\backslash{}B$, then $\alpha\beta$ is of
  category $B$.
\item \textbf{Forward application}: If $\alpha$ is a string of category $A$ and
  $\beta$ is a string of category $B/A$, then $\beta\alpha$ is of category $B$.
\end{enumerate}

\begin{figure}
  \centering
  \Tree [
    .$s$
    [
      .$\mathit{np}$
      John
    ]
    [
      .$\mathit{np}\backslash{}s$
      [
        .$\mathit{np}\backslash{}\mathit{s}/\mathit{np}$
        loves
      ]
      [
        .$\mathit{np}$
        Mary
      ]
    ]
  ]
  \caption[A syntactic tree]{A syntactic tree for \textit{John loves Mary}. The lexicon assigns
    categories to words: \textit{John} is $\mathit{np}$, loves is
    $\mathit{np}\backslash{}\mathit{s}/\mathit{np}$ and Mary is
    $\mathit{np}$. Backward and forward composition rules derive the syntactic
    tree.}
\label{fig:cg}
\end{figure}

Figure~\ref{fig:cg} illustrates the parse tree for \textit{John loves Mary}
obtained using the category composition rules.

The last step is to map syntactic categories with semantic terms. Again, there
are base types ($e$ for entities and $t$ for sentences) and complex types of the
form $(a \to b)$ where $a$ and $b$ are types. The mapping between syntactic
categories and semantic types is defined as a function $\mathit{type}$:
%
\begin{align*}
  &\mathit{type}(np) = e \\
  &\mathit{type}(s) = t \\
  &\mathit{type}(A/B) = (\mathit{type}(B) \to \mathit{type}(A)) \\
  &\mathit{type}(B\backslash{}A) = (\mathit{type}(B) \to \mathit{type}(A)) \\
\end{align*}

Syntactic backward and forward application corresponds to functional
application. Figure~\ref{fig:syn} shows the final parse tree.

\begin{figure}
  \centering
  \Tree [
    .$s$~:~$\mathit{loves}'(\mathit{john}',\mathit{mary}')$
    [
      .$\mathit{np}$~:~$\mathit{john}'$
      John
    ]
    [
      .$\mathit{np}\backslash{}s$~:~$\lambda~x.\mathit{loves}'(x,~\mathit{mary}')$
      [
        .$\mathit{np}\backslash{}\mathit{s}/\mathit{np}$~:~$\lambda{}y.\lambda{}x.\mathit{loves}'(x,y)$
        loves
      ]
      [
        .$\mathit{np}$~:~$\mathit{mary}'$
        Mary
      ]
    ]
  ]
  \caption{The final parse tree}
\label{fig:syn}
\end{figure}

Given a suitable pairing between a syntactic grammar, semantic representations and corresponding general combinatory operators, this can produce structured sentential representations with a broad coverage and good generalisability \cite{step2008:2222}. This logical approach is extremely powerful because it can capture complex aspects of meaning such as quantifiers and their interactions \cite{Copestake2005}, and enables inference using well studied and developed logical methods \cite{bos2000first}.

\subsection{Compositional distributional semantics}
\label{sec:composition}

Methods based on the distributional hypothesis have recently been applied to many tasks, but mostly at the word level, for instance, word sense disambiguation \cite{ZhitomirskyGeffet2009} and lexical substitution \cite{Thater:2010:CSR:1858681.1858778}. They exploit the notion of similarity which correlates with the angle between word vectors \cite{Turney:2010:FMV:1861751.1861756}.

\emph{Compositional} distributional semantics goes beyond the word level and models the meaning of phrases or sentences based on their parts. \newcite{mitchell-lapata:2008:ACLMain} perform composition of word vectors using vector addition and multiplication operations. The limitation of this approach is the operator associativity, which ignores the argument order, and thus word order. As a result, ``\textit{John loves Mary}'' and ``\textit{Mary loves John}'' get assigned the same meaning.

Concretely, if \textit{John}'s, \textit{Mary}'s and \textit{loves}'s meanings are
represented as vectors $\ov{\mathit{john}}$, $\ov{\mathit{mary}}$ and
$\ov{\mathit{loves}}$, the meaning of the sentence \textit{John loves Mary} is
$\ov{\mathit{john}} + \ov{\mathit{loves}} + \ov{\mathit{mary}}$.

To capture word order and the syntactic structure of a sentence, various approaches have been proposed. \newcite{Grefenstette:2011:ESC:2145432.2145580} extend the compositional approach by using non-associative linear algebra operators as proposed in the theoretical work of \newcite{DBLP:journals/corr/abs-1003-4394}.

The functional applications of semantic terms can be replaced with tensors \cite{Bourbaki1998commutative}. Then, a transitive verb is represented by a matrix, which can be obtained from a corpus using the formula $\sum_i\ov{s_i} \otimes \ov{o_i}$ \cite{Grefenstette:2011:ESC:2145432.2145580}, where $\ov{s_i}$ and $\ov{o_i}$ are the subject-object pairs of the verb and $\otimes$ is the Kronecker product. The vector of the whole sentence is $\overline{\mathit{loves}} \odot(\ov{\mathit{john}} \otimes \ov{\mathit{mary}})$, where $\odot$ is the element-wise product.
% * <sdyck@ualberta.ca> 2016-11-03T03:46:50.769Z:
%
% > it{loves
%
% Should this have a vector symbol above it, rather than a straight line?
% DM: no, becausue it's a matrix, not a vector.
% ^.

\subsection{Similarity of heads in context}
\label{sec:similarity-context}

A noun phrase can be similar to a noun, as in \textit{female lion} and \textit{lioness}, and to other noun phrases as in \textit{yellow car} and \textit{cheap taxi}. The same similarity principle can be applied to phrases as well as to words. In this case, similarity is measured in \emph{context}, and most methods of calculating similarity of phrases still rely on comparisons of the phrases' head words, which meanings are modified by the arguments they appear with \cite{Kintsch2001173}.

\newcite{mitchell-lapata:2008:ACLMain,mitchell2010composition} use element-wise addition and multiplication to model argument interaction. \newcite{Baroni2010nouns} represents adjectives as matrices that modify nouns (vectors) using matrix multiplication.

\newcite{Dinu:2010:MDS:1870658.1870771} model word meaning as a distribution over senses; a context feature (other word in the context)  directly modulates word’s sense distribution using conditional probability. \newcite{thater-furstenau-pinkal:2011:IJCNLP-2011} contextualise vectors by assigning higher weight to features that correspond or are distributionally similar to the context words.

% \cite{Seaghdha:2011:PMS:2145432.2145545}.

With verbs, similarity in context can be applied to compare a transitive and intransitive verb. For example, \textit{cycle} is similar to \textit{ride a bicycle}. Here we see that \textit{a bicycle} disambiguates the verb \textit{ride} making the phrase similar to the verb \textit{cycle}. The connection between measuring similarity of a phrase and disambiguation has been noted in \newcite{kartsaklis-sadrzadeh-pulman:2013:CoNLL-2013}.

% TODO: mention \cite{wieting2015paraphrase}

Sentential similarity might be treated as the similarity of the heads in their contexts. That is, the similarity between \textit{sees} and \textit{notices} in \textit{John \textbf{sees} Mary} and \textit{John \textbf{notices} a woman}. This approach abstracts away the grammatical difference between the sentences and concentrates on their semantics.

\section{Evaluation}
\label{sec:intrinsic-evaluation}

The difference in occurrence between two words quantifies the difference in their meaning \cite{harris1954distributional} and there is a machinery to measure the difference in the meanings of two sentences \cite{DBLP:journals/corr/abs-1003-4394}. The final necessary part is the evaluation methodology to test this approach.

Because it is difficult to perform extrinsic evaluation (also called evaluation in use) to measure the performance of a similarity component in a pipeline of a complete natural language processing system (for example, a dialog system), intrinsic datasets that focus on similarity are popular among computational linguists.

Apart from a pragmatic attempt to alleviate the problems of evaluating
similarity components, these datasets serve as an empirical test of the
hypotheses of Firth and Harris, bringing together our understanding of the human mind, language and technology. The following sections introduce the main datasets used in this area.

\subsection{Word similarity}
\label{sec:lexical-similarity}

\newcite{Rubenstein:1965:CCS:365628.365657} performed an empirical study on the relationship between the similarity in word meaning and similarity of contexts they appear in. They build a list of 65 word pairs that range from highly synonymous to semantically unrelated.

To obtain the similarity judgements, the human subjects were given a shuffled deck of card. Each card contained a word pair from the list. They were asked to sort the cards by similarity, so that the most similar items appeared in the top of the deck. In addition to ranking, human subjects were asked to give similarity scores ranging from 4.0 to 0.0, where the higher value indicates the higher similarity level.

100 sentences for each distinct word in the list were written by 50 participants (they did not provide similarity judgements) to be used as the contexts. The words from the list had to be used as nouns and the sentences had to be at least 10 words long.

To measure similarity, the overlap was calculated over tokens and types. The token condition takes into account the appearance frequencies of context words, while the type condition only considers the word types. So if the word \textit{table} appeared 10 times as a context of a word of interest, in the case of the token condition all 10 occurrences are considered, in the case of the type condition only the fact that it appeared as a context is recorded.

Formally, the overlap $M_x$ over condition $x$ between the word context of words $A$ and $B$ was calculated as:
%
\begin{equation*}
  M_x = \frac{N(A_xB_x)}{\min(N(A_x), N(B_x))}
\end{equation*}
%
where $N(A_xB_x)$ is the number of context words shared between $A$ and $B$ under condition $x$. $N(A_x)$ and $N(B_x)$ are the number of context words under condition $x$ for the words $A$ and $B$ respectively.

They found that the more similar words in meaning are the more context they share for both token and type conditions. Moreover, the relationship is strongest for the highly synonymous pairs, namely the pairs with similairty greater than 3.0.

\newcite{1986-13502-00119860101} studied the similarity relation from the psychological perspective. They analysed the similarity judgements between the entries by their geometric structure. The geometric approach represents the entries as points in a multidimensional space so that the distance between them reflects the similarity.

They examined 100 datasets to identify common geometric patterns. The datasets contained entries that belonged to a single category such as \textit{verbs of judging} \cite{FILLENBAUM197454} or \textit{animal terms} \cite{HENLEY1969176}. The reason for category oriented similarity studies was that ``stimuli can only be compared in so far as they have already been categorised as identical, alike, or equivalent at some higher level of abstraction'' \cite{turner1987rediscovering}.

They observed that if a category contains a superordinate, similarity judgements arrange category members around it. For example, similarity judgements given by humans arrange fruit names around the word \textit{fruit} in such a way that it is their nearest neighbour, making \textit{fruit} the \emph{focal point} of the category of \textit{fruits}.

An important consequence is that high centrality values cannot be achieved in a space with dimensionality of two or three,  because the dimensionality sets the upper bound on the number of points that can share the nearest neighbour.

% \begin{wraptable}[12]{O}{0.5\textwidth}
\begin{table}
  \centering
  \begin{tabular}{lccc}
    \toprule
    Model                                              & WS353 & MEN   & SimLex-999 \\
    \midrule
    \newcite{2002:PSC:503104.503110}                   & 0.55  &       &            \\
    % \addlinespace
    \newcite{Bruni:2012:DST:2390524.2390544}           & 0.75  & 0.76  &            \\
    \newcite{kiela-clark:2014:CVSC}                    & 0.58  & 0.71  &            \\
    \newcite{baroni-dinu-kruszewski:2014:P14-1}                                     \\
    \quad Distributional model                         & 0.62  & 0.72  &            \\
    \quad Neural word embeddings                       & 0.73  & 0.80  &            \\
    \addlinespace
    \newcite{hill2014simlex}                                                        \\
    \quad Distributional model                         & 0.42  & 0.44  & 0.19       \\
    \quad Neural word embeddings                       & 0.44  & 0.43  & 0.28       \\
    \newcite{TACL570}                                                               \\
    \quad Distributional model                   & 0.75, 0.70  & 0.75  & 0.39       \\
    \quad Neural word embeddings                 & 0.79, 0.69  & 0.77  & 0.44       \\
    % \addlinespace
    % \newcite{kiela-hill-clark:2015:EMNLP}              &       & 0.69  & 0.47       \\
    % \newcite{recski-EtAl:2016:RepL4NLP}                &       &       & 0.76       \\
    \addlinespace
    State of the art                                   & 0.81  &       & 0.76       \\
    Upper bound                                        &       & 0.84  & 0.78       \\
    \bottomrule
  \end{tabular}
  \caption[]{Model performance on various datasets. For
    \newcite{Bruni:2012:DST:2390524.2390544} the model with the highest average
    score is shown (TunedFL, Window20). \newcite{TACL570} report two results on
    WS353: on the subset that contains similar items (the first number) and the
    subset that contain related items (the second number).
    The state of the art are \newcite{Halawi:2012:LLW:2339530.2339751} for WS353
    and \newcite{recski-EtAl:2016:RepL4NLP} for
    SimLex-999.
  }
\label{tab:lexical-dataset-comparison}
% \end{wraptable}
\end{table}


\newcite{2002:PSC:503104.503110} proposed a context oriented information retrieval framework. The idea that the meaning of the word \textit{jaguar} is dependent on the context the search is performed. It might be a car, if the query comes from an automotive website, or it might mean an animal if it comes from a website about nature. To evaluate their semantic component they developed a dataset WS353 that consists of 353 diverse noun pairs along with their relatedness scores on a scale from 0 (totally unrelated) to 10 (very much related or identical). The combination of a vector-based method and a WordNet-based method achieved correlation of 0.55. The dataset they proposed is widely used to evaluate algorithms that estimate semantic similarity.

\newcite{Bruni:2012:DST:2390524.2390544} introduced the MEN dataset to test a multimodal semantic space (the model used textual and visual features). The new dataset contains the words that appear in labels of the ESP-Game\footnote{\url{http://www.cs.cmu.edu/~biglou/resources/}} and MIRFLICKR-1M\footnote{\url{http://press.liacs.nl/mirflickr/}} image collections. Compared to WS353 it is sufficiently large to be split to the development part (2000 pairs) and to the test part (1000 pairs) for evaluation. The dataset contain highly similar items (\textit{cathedral}, \textit{church}, 0.94) and also terms that stand in a broader semantic relationship, such as whole-part (\textit{flower}, \textit{petal}, 0.92). The scores are relatedness scores on a scale from 0.0 to 1.0.

The Spearman correlation of the judgements given by two authors of the paper for all 3000 pairs is 0.68. The correlation with the average of their scores with the dataset scores is 0.84, which can be taken as the upper bound. The best presented results are 0.75 for WS353 and 0.78 for MEN, note that two different models produce them.

\newcite{hill2014simlex} presented SimLex-999, a gold standard resource for evaluating distributional semantic models. In contrast to WS353 and MEN it focuses on similarity rather than relatedness. The words \textit{coffee} and \textit{cup} are related but not similar. The dataset consists of 666 noun-pairs, 222 verb-pairs and 111 adjective-pairs. 500 residents of the USA were recruited to collect human judgements. The average pairwise Spearman correlation between two participants is 0.67, however the average correlation of a human rater with the average of all the other raters is 0.78.

They tested several models on WS353, MEN and WordSim-999. They showed that existing models achieved lower correlation on SimLex-999 than on WS353 and MEN suggesting that the new dataset required development of similarity-specific models. The best reported results are 0.44 on WS353, 0.48 on MEN and 0.28 on SimLex-999, these are different models, but all of them are trained on 150 million word RCV1 Corpus \cite{lewis2004rcv1}. Neural models trained on a larger corpus (Wikipedia) yield higher results.

Table~\ref{tab:lexical-dataset-comparison} presents the key models and the results they achieved on WS353, MEN and SimLex-999.

\subsection{Parameter selection}
\label{sec:parameter-selection-intro}

The variance of results on word similarity tasks can be explained by the differences in algorithms or by difference in model parameters, corpus used and other factors. For example, \newcite{baroni-dinu-kruszewski:2014:P14-1} uses a corpus of 2.8 billion tokens and outperforms \newcite{hill2014simlex}, who uses a corpus of 0.15 billion tokens, on MEN by 0.28 points with a distributional model and by 0.37 with neural word embeddings.

\newcite{Bullinaria2007} presented a thorough study of the parameters of distributional models. They tested various vector space parameters and similarity measures.

Instead of raw co-occurrence counts they used the conditional probability of a context word appearing close to a target word $P(c|t)$. In addition to that, they used \emph{point-wise mutual information} $\operatorname{PMI(c,t)} = \log\left(\frac{P(c,t)}{P(c)P(t)}\right)$, positive PMI, which nullifies negative values and probability ratio $\frac{P(c|t)}{P(c)}$. They varied the vector space dimensionality from 1 to 100000 dimensions.  Cosine, Euclidean and City Block geometric measures were used to estimate similarity. They used the 87.9 million word British National Corpus and performed an exhaustive search across the parameter combinations.

A vector space with vector components computed with positive PMI and cosine similarity measure performed the best in their experiments. They also showed that the performance depends on the size of the corpus used, the larger the corpus, the better the results. The context window of size one produced the best results. The model performance peaked when 1000 dimensional vector space was used and dropped afterwards.

\newcite{kiela-clark:2014:CVSC} performed a systematic study of distributional models on various datasets including WS353 and MEN. They confirmed findings of \newcite{Bullinaria2007} that larger corpora lead to better performance. They suggested using ukWaC (2 billion tokens) over the BNC with a small context window of size less than 5 from each side. Positive PMI was the best performed measure in combination with the \emph{correlation similarity} measure, which is the mean-adjusted cosine similarity. The dimensionality of 50000 appeared to be optimal.

\citet{kiela-clark:2014:CVSC} advocated incremental parameter tuning, reasoning that compositional tasks need a vector space model that is good in lexical tasks. Practically, this means that first a good lexical model is identified and only then it is used to find a well performing compositional operator. \citet{BullinariaLevy2012} followed the same reasoning, first good sparse models are identified, and then dimensionality reduction methods are compared to find the best one.

\citet{lapesa-evert:2013:CMCL} argue against incremental tuning because it does not capture parameter interaction. For example, a compositional operator might benefit from a specific weighting scheme, which is not necessarily the best in predicting word similarity. Their goal was to contrast rank-based prediction of semantic priming with distance-based prediction. Because they were introducing rank-based prediction, they could not reuse recommendations of parameters that are derived from distance-based similarity experiments. They covered a broad range of parameters of distributional models aiming at identification of parameter configurations that achieve good performance in predicting semantic priming.

They used linear regression to determine the importance of individual parameters and their combinations. Distributional semantics model parameters, such as the weighting scheme, vector space dimensionality and similarity metric, were considered predictors of model performance: the parameters of distributional models were independent parameters of a linear model and distributional model performance scores was dependant variable. Analysis of variance was used to determine the most important parameters and their interactions. They showed that a statistical association measures---such as t-score or z-score \cite{Evert05}---with combination of ranked-based prediction over cosine similarities yielded the best results. Would they performed only iterative tuning, they would test ranked-based prediction only with PMI weighting, which underperformed in their experiments.

In a successive study that included word similarity, \citet{lapesa2014large} showed that the combination of the similarity score together with the weighting scheme was the most important parameter combination, supporting findings of \newcite{Bullinaria2007} and \newcite{kiela-clark:2014:CVSC}, who suggested to use PMI together with cosine similarity.

%With their analysis they were also to perform model selection that is robust to overfitting. In most cases, the result of their robust parameter optimisation was close to the highest score. They considered the highest score they achieved to be overfitted because 537600 models were tested and there was a high chance that the best performance happened by chance.

\newcite{baroni-dinu-kruszewski:2014:P14-1} and \newcite{TACL570} performed systematic parameter studies on distributional and neural models. Notably, the work of \newcite{TACL570} evaluated models on all three datasets (WS353, MEN and SimLex-999, see Table~\ref{tab:lexical-dataset-comparison}). Their best distributional model outperformed the distributional models presented in \citet{hill2014simlex}, \citet{kiela-clark:2014:CVSC} and \citet{baroni-dinu-kruszewski:2014:P14-1}.

\subsection{Disambiguation of verbs in context}
\label{sec:disamb}

The transitive verb disambiguation dataset
% \footnote{This and the sentence  similarity datasets are available at \url{http://www.cs.ox.ac.uk/activities/compdistmeaning/}}
described in \newcite{Grefenstette:2011:ETV:2140490.2140497} consists of ambiguous transitive verbs together with their arguments, landmark verbs (which identify one of the verb senses) and human judgements (which specify the similarity to the landmarks of the disambiguated sense of the verb in the given context). This is similar to the intransitive dataset described in \newcite{mitchell-lapata:2008:ACLMain}.

Consider the sentence \textit{system meets specification}. \textit{Meets} is an ambiguous transitive verb, and \textit{system}
and \textit{specification} are its arguments. Possible landmarks for \emph{meet} are \textit{satisfy} and \textit{visit}. For this sentence, the human judgements show that the disambiguated verb meaning is similar to the landmark \textit{satisfy}, and less similar to \textit{visit}.

The task is to estimate the similarity of the sense of a verb in a context with a given landmark. To estimate similarity, the verb is composed with its arguments, it is done the same for the landmark and the arguments, and the similarity of the two vectors is computed. To evaluate performance, the human judgements are averaged for the same verb, argument and landmark entries, and these average values are used to calculate the correlation.

\subsection{Sentence similarity}
\label{sec:sentence-similarity}

The transitive sentence similarity dataset described in \newcite{kartsaklis-sadrzadeh-pulman:2013:CoNLL-2013} consists of transitive sentence pairs and human similarity judgements. The task is to estimate similarity between two sentences. The evaluation is the same as in the disambiguation task (Section~\ref{sec:disamb}).

In general, there is no systematic study of parameters of compositional models similar to lexical studies discussed in Section~\ref{sec:parameter-selection-intro}.

\section{Conclusion}
\label{sec:conclusion-comp}

Similarity is an important notion in psychological theories of knowledge and behaviour. It is also useful in explaining language. Many NLP systems benefit from internal similarity components. However the exact definition of similarity is task-dependant: an IR system needs to know whether the words are related (for example, the verb \textit{swim} is related to the noun \textit{lake}), language models benefit from semantic similarity (the noun \textit{lake} is similar to the noun\textit{reservoir}, but not to the verb \textit{swim}), dialog systems need to know the similarity of roles utterances---not individual words---play in a discourse (\textit{Hi!} and \textit{Good morning.} are both greetings).

Representation of words and multi-word units (sentences, utterances, documents) to measure similarity within one framework is theoretically challenging. \newcite{goodman1972problems} argues that the similarity relation does not exist, because everything can be shown to be similar to everything else. \newcite{medin1993respects} and \newcite{Markman1996} response that similarity has to be contextualised to be measured, that is, the features of interest need to be defined before the measurement.

According to \newcite{harris1954distributional}, the word meaning does not need to be obtained to measure similarity, instead, the differences of occurrences of two words quantify the difference in their meaning. In this way the problems with representing meaning in isolation raised by Frege are avoided, because word meaning is not constructed.

The principle of compositionality, that the representations of compounds are built from their parts, is the hallmark of categorical compositional semantics \cite{DBLP:journals/corr/abs-1003-4394}. It extends the composition mechanism from formal semantics by replacing the representation of atoms and relations with tensors of various orders.

In formal semantics, the meaning of a phrase is obtained by applying backward and forward application rules. Consider a phrase \textit{John walks}. In this example, the string \textit{John} is the atom $\mathit{John'}$ of category $\mathtt{np}$ and type $e$, \textit{walks} is the relation $\lambda x.\mathit{walks'}(x)$ of category $\mathtt{np\backslash{}s}$ and type $e \to t$. During the phrase composition, the backward application rule is applied, so the category of the whole string becomes $\mathtt{s}$, its type is $t$ which is either \textit{true} or \textit{false} and the meaning is computed by the evaluation of the formula $\mathit{walks'}(\mathit{John'})$, basically the atom $\mathit{John'}$ is applied to the formula $\lambda x.\mathit{walks'(x)}$.

Categorical compositional semantics replaces atomic symbols in formal semantics with vectors (first-order tensors), and relations with higher-order tensors (so \textit{walks} is represented by a second-order tensor, which is a matrix). To obtain the representation of a compound, tensor contraction is used instead of function application.

Word similarity (in the broad psychological sense) has been studied extensively: many datasets were proposed that focus on relatedness and similarity (in the specific sense that \textit{lake} is similar to \textit{reservoir}, but is not similar to \textit{swim}). Several studies were carried out to identify the best parameter choice for the models of similarity.

Phrase similarity on the other hand, has been studies much less. There are several datasets proposed that focus on different aspects of similarity (for example, its application in word sense disambiguation within a context and similarity measurement of sentences). Several compositional methods are developed. However, the compositional methods were not evaluated in a systematic way to identify the best parameters (for example, the co-occurrence weighting function among the others).

This thesis addresses this gap by carrying out an extensive parameter study of compositional models of similarity.

%%% Local Variables:
%%% mode: latex
%%% TeX-master: "thesis.tex"
%%% TeX-engine: xetex
%%% End:

\chapter{Experimental methodology}
\label{sec:methodology}



%%% Local Variables:
%%% mode: latex
%%% TeX-master: "thesis"
%%% End:

% \chapter{Intrinsic experiments}
% \label{cha:experiments}

% Table~\ref{tab:parameters} lists parameters and their values. As the source corpus we use the concatenation of Wackypedia and ukWaC \cite{ukwac} with a symmetric 5-word window \cite{milajevs-EtAl:2014:EMNLP2014}; the evaluation metric is the correlation with human judgements as is standard with SimLex \cite{hill2014simlex} and other lexical datasets.

\todo[inline]{
  What's the best selection method.
  Does Max overfit?  Statistical significance.
  Title: mention PMI?
  Lexical comparison with other work.
}

% We derive our parameter selection heuristics by greedily selecting parameters (\texttt{cds}, \texttt{neg}) that lead to the highest average performance for each combination of frequency weighting, PMI variant and dimensionality $D$. Figures~\ref{fig:interaction-cds} and \ref{fig:interaction-neg} show the interaction of \texttt{cds} and \texttt{neg} with other parameters. We also vary the similarity measure (cosine and correlation  \cite{kiela-clark:2014:CVSC}), but do not report results here due to space limits.

\chapter{Similarity and relatedness of words}
\label{sec:lexical}

\section{SimLex-999}
\label{sec:simlex-999}

\subsection{Max selection}
\label{sec:max-selection-simlex}

\begin{wrapfigure}[6]{O}{0.5\textwidth}
  \vspace{-1em}
  \includegraphics[width=0.5\textwidth]{supplement/figures/SimLex999-results}
  \caption{SimLex-999 results}
  \label{fig:SimLex999-results}
\end{wrapfigure}

%%% Local Variables:
%%% mode: latex
%%% TeX-master: "../thesis"
%%% End:


Figure~\ref{fig:SimLex999-results} illustrates the results based on the best model selection and Table~\ref{tab:parameters} shows the results together with picked parameters. Note that the maximum selection is identical with cross-validation: they pick the same models.

In general, model performance increases as the dimensionality increases. However, the best result of 0.389 is achieved with a 2000 dimensional space, this could, however, be an example of overfitting. Model performance becomes stable for dimensions greater than 20000.

For spaces with dimensionality less than 5000 \texttt{freq} 1 and inner product yield best results. Otherwise, cosine with \logNSCPMI/, smoothing $\alpha=0.75$ and shifting $k=0.7$ gives the best results.

\begin{table}
  \centering

  \begin{tabular}{rrlllrl}
\toprule
 dimensionality &  SimLex999 &  freq &  discr &     cds &  neg &     similarity \\
\midrule
           1\,000 &      0.369 &     1 &   spmi &       1 &  0.2 &  inner\_product \\
           2\,000 &      0.389 &     1 &  scpmi &  global &  0.7 &  inner\_product \\
           3\,000 &      0.376 &     1 &   spmi &    0.75 &  0.2 &  inner\_product \\
           5\,000 &      0.363 &  logn &  scpmi &  global &  1.0 &            cos \\
          10\,000 &      0.371 &  logn &  scpmi &       1 &  0.7 &            cos \\
          20\,000 &      0.381 &  logn &  scpmi &    0.75 &  0.7 &            cos \\
          30\,000 &      0.383 &  logn &  scpmi &    0.75 &  0.7 &            cos \\
          40\,000 &      0.384 &  logn &  scpmi &    0.75 &  0.7 &            cos \\
          \textbf{50\,000} &      \textbf{0.385} &  \textbf{logn} &  \textbf{scpmi} &    \textbf{0.75} &  \textbf{0.7} &            \textbf{cos} \\
\bottomrule
\end{tabular}


  \caption[SimLex-999 Max selection]{SimLex-999 Max selection. Correlation values that are not statistically
    significant from the highest value of 0.389 are in bold. For SimLex-999,
    $\sigma^{0.9}_{0.05} = 0.023$ making the values greater than 0.366
    indistinguishable from the highest result.}
  \label{tab:Simlex999-max-selection}
\end{table}


\subsection{Heuristics}
\label{sec:heuristics-simlex}

\begin{wraptable}[8]{O}{0.5\textwidth}
  \vspace{-60pt}
  \centering

  \begin{tabular}{lr}
\toprule
      parameter &  partial $R^2$ \\
\midrule
     similarity &      0.379 \\
           freq &      0.268 \\
            neg &      0.241 \\
 dimensionality &      0.084 \\
          discr &      0.077 \\
            cds &      0.064 \\
\bottomrule
\end{tabular}


  \caption{SimLex-999 feature ablation.}
  \label{tab:SimLex999-ablation}
\end{wraptable}


% \begin{figure}

  \centering

  \begin{subfigure}[t]{0.49\textwidth}
    \includegraphics[width=\textwidth]{supplement/figures/SimLex999-interaction-similarity}
    \caption{similarity}
    \label{fig:SimLex999-interaction-similarity}
  \end{subfigure}
  \begin{subfigure}[t]{0.49\textwidth}
    \includegraphics[width=\textwidth]{supplement/figures/SimLex999-interaction-freq}
    \caption{\texttt{freq}}
    \label{fig:SimLex999-interaction-freq}
  \end{subfigure}

  \begin{subfigure}[t]{0.49\textwidth}
    \includegraphics[width=\textwidth]{supplement/figures/SimLex999-interaction-neg}
    \caption{\texttt{neg}}
    \label{fig:SimLex999-interaction-neg}
  \end{subfigure}
  \begin{subfigure}[t]{0.49\textwidth}
    \includegraphics[width=\textwidth]{supplement/figures/SimLex999-interaction-discr}
    \caption{\texttt{discr}}
    \label{fig:SimLex999-interaction-discr}
  \end{subfigure}

  \begin{subfigure}[t]{0.49\textwidth}
    \includegraphics[width=\textwidth]{supplement/figures/SimLex999-interaction-cds}
    \caption{\texttt{cds}}
    \label{fig:SimLex999-interaction-cds}
  \end{subfigure}

  \caption{SimLex-999 parameter interaction. Parameters are shown in the order of their influence.}
  \label{fig:SimLex999-interaction}
\end{figure}

%%% Local Variables:
%%% mode: latex
%%% TeX-master: "../thesis"
%%% End:


The linear model achieves an adjusted $R^2$ of 0.867, indicating that the model is able to predict model performance based on parameter selection quite well. Table~\ref{tab:SimLex999-ablation} shows partial $R^2$ scores for parameters. The most influential parameters in decreasing order are similarity, \texttt{freq} and \texttt{neg}.

% \begin{wrapfigure}{O}{0.5\textwidth}
\begin{figure}[h]
  % \vspace{-30pt}
  \centering

  \includegraphics[width=0.5\textwidth]{supplement/figures/SimLex999-interaction-similarity}

  \caption{SimLex-999 influence of the similarity measure.}
  \label{fig:SimLex999-similarity}
\end{figure}

Figure~\ref{fig:SimLex999-similarity} shows the average performance of similarity measures. Correlation outperforms all other measures for all dimensions and peaks at the dimensionality of 20000 after correlation is chosen as the similarity measure.

% \begin{figure}[h]
% \begin{wrapfigure}{O}{0.5\textwidth}
  % \vspace{-30pt}
  \centering

  \includegraphics[width=0.5\textwidth]{supplement/figures/SimLex999-interaction-freq}

  \caption{SimLex-999 influence of \texttt{freq}.}
  \label{fig:SimLex999-freq}
\end{figure}

The influence of \texttt{freq}, the second parameter, is shown on Figure~\ref{fig:SimLex999-freq}. $\log n$ frequency outperforms other choices for all dimensions. After 20000 dimensions $\log n$'s performance stabilises: variance decreases and the performance stays constant.

\begin{figure}[h]
% \begin{wrapfigure}{O}{0.5\textwidth}
  % \vspace{-30pt}
  \centering

  \includegraphics[width=0.5\textwidth]{supplement/figures/SimLex999-interaction-neg}

  \caption{SimLex-999 influence of \texttt{neg}.}
  \label{fig:SimLex999-neg}
\end{figure}

The third parameter \texttt{neg} of 0.7 shows the best performance (Figure~\ref{fig:SimLex999-neg}). However, there is little difference between models with dimensionality greater than 20000, apart from the models that do not perform shifting, whose performance peaks at 20000 dimensions and decreases afterwards with increasing variance.

% \begin{figure}[h]
% \begin{wrapfigure}{O}{0.5\textwidth}
  % \vspace{-30pt}
  \centering

  \includegraphics[width=0.5\textwidth]{supplement/figures/SimLex999-interaction-discr}

  \caption{SimLex-999 influence of \texttt{discr}. PMI and CPMI are not shown because at the step before models with shifting were chosen.}
  \label{fig:SimLex999-discr}
\end{figure}

There is little difference between SPMI and SCPMI performance with a little advantage to SCPMI (Figure~\ref{fig:SimLex999-discr}).

% \begin{figure}[h]
\begin{wrapfigure}[5]{O}{0.5\textwidth}
  \vspace{-30pt}
  \centering

  \includegraphics[width=0.5\textwidth]{supplement/figures/SimLex999-interaction-cds}
  \caption{SimLex-999 influence of \texttt{cds}.}
  \label{fig:SimLex999-cds}
\end{wrapfigure}

Finally, models benefit from context distribution smoothing, spaces with less than 10000 dimensions produce the best results with $\alpha = 1$, for spaces with higher dimensionality $\alpha = 0.75$ is the most advantageous (Figure~\ref{fig:SimLex999-cds}).

\todo[noline]{Contrast or compare results with \cite{milajevs-sadrzadeh-purver:2016:ACL-SRW}}

\subsection{Difference between Max selection and heuristics on SimLex-999}

\begin{table}
  \centering

  \begin{tabular}{rrlllll}
\toprule
 dimensionality &  SimLex999 &  freq &  discr &   cds &  neg &   similarity \\
\midrule
           1\,000 &       0.33 &  logn &  scpmi &     1 &  0.7 &  correlation \\
           2\,000 &       0.35 &  logn &  scpmi &     1 &  0.7 &  correlation \\
           3\,000 &       0.35 &  logn &  scpmi &     1 &  0.7 &  correlation \\
           5\,000 &       0.35 &  logn &  scpmi &     1 &  0.7 &  correlation \\
          10\,000 &       0.37 &  logn &  scpmi &  0.75 &  0.7 &  correlation \\
          20\,000 &       \textbf{0.38} &  logn &  scpmi &  0.75 &  0.7 &  correlation \\
          30\,000 &       \textbf{0.38} &  logn &  scpmi &  0.75 &  0.7 &  correlation \\
          40\,000 &       \textbf{0.38} &  logn &  scpmi &  0.75 &  0.7 &  correlation \\
          50\,000 &       \textbf{0.38} &  logn &  scpmi &  0.75 &  0.7 &  correlation \\
\bottomrule
\end{tabular}


  \caption[SimLex-999 selection based on heuristics]{SimLex-999 selection based on heuristics. The highest value is 0.384.
  The values that are grater than 0.361 are indistinguishable from the highest score.}
  \label{tab:Simlex999-heuristics-selection}
\end{table}


As expected manual parameter selection is more stable as Table~\ref{tab:Simlex999-heuristics-selection} shows. Both selection models agree on parameters for highly dimensional spaces ($D \geq 2000$), with an exception of similarity: Max selection prefers cosine, while manual prefers correlation based similarity measure. Because of this, manual selection does not pick the best result of the 2000 dimensional model, but at 50000 dimensions  a model selected manually scores 0.001 lower: 0.384 versus 0.385 as also seen on Figure~\ref{fig:SimLex999-results}.

The average relative difference between Max selection and heuristics is 0.039.

\section{MEN}
\label{sec:men}

\subsection{Max selection}
\label{sec:max-selection-men}

% \begin{figure}
\begin{wrapfigure}[7]{O}{0.5\textwidth}
  \vspace{-1em}
  \includegraphics[width=0.5\textwidth]{supplement/figures/men-results}
  \caption{MEN results}
  \label{fig:men-results}
\end{wrapfigure}

%%% Local Variables:
%%% mode: latex
%%% TeX-master: "../thesis"
%%% End:


Figure~\ref{fig:men-results} shows the selection results. Again, cross-validation results are identical with Max selection. Table~\ref{tab:men-max-selection} shows the results together with the selected models.

\begin{table}[b]
  \centering

  \begin{tabular}{rrlllrl}
\toprule
 dimensionality &    men &  freq &  discr &     cds &  neg &   similarity \\
\midrule
           1000 &  0.686 &     1 &  scpmi &  global &  1.4 &  correlation \\
           2000 &  0.728 &  logn &  scpmi &       1 &  0.7 &          cos \\
           3000 &  0.737 &  logn &  scpmi &       1 &  0.7 &          cos \\
           5000 &  0.743 &  logn &  scpmi &    0.75 &  0.7 &          cos \\
          10000 &  0.753 &  logn &  scpmi &    0.75 &  1.0 &  correlation \\
          20000 &  0.763 &  logn &  scpmi &    0.75 &  1.0 &  correlation \\
          30000 &  0.765 &  logn &  scpmi &    0.75 &  1.0 &  correlation \\
          40000 &  0.765 &  logn &  scpmi &    0.75 &  1.0 &  correlation \\
          50000 &  0.765 &  logn &  scpmi &    0.75 &  1.0 &  correlation \\
\bottomrule
\end{tabular}


  \caption{MEN Max selection}
  \label{tab:men-max-selection}
\end{table}


Model performance monotonically increases as the dimensionality increases. The highest score of 0.765 is achieved by 3 spaces with $D \geq 30000$, \logNSCPMI/, smoothed context distribution ($\alpha = 0.75$), shifted PMI values ($k = 1$) and the similarity measure based on correlation.

In comparison with SimLex-999, models with ``more extreme'' parameters give better results. For example, $\alpha = 0.75$ is the best for models tested on SimLex-999 with dimensionality starting with 20000, while for models tested on MEN, this parameter choice is the best starting with 5000. Similar behaviour is observed for \texttt{neg} and similarity. For highly dimensional spaces the switch from SimLex-999 to MEN changes the best \texttt{neg} choice from 0.7 to 1 and similarity from cosine to correlation. Such a switch of parameter choices might suggest the difference between \textit{relatedness} and \textit{similarity}.

\subsection{Heuristics}
\label{sec:heuristics-men}

\begin{wraptable}[18]{O}{0.5\textwidth}
  \vspace{50pt}
  \centering

  \begin{tabular}{lr}
\toprule
      parameter &  partial $R^2$ \\
\midrule
            neg &  0.31 \\
           freq &  0.21 \\
     similarity &  0.18 \\
          discr &  0.12 \\
 dimensionality &  0.11 \\
            cds &  0.09 \\
\bottomrule
\end{tabular}


  \caption{MEN feature ablation.}
  \label{tab:men-ablation}
\end{wraptable}

The linear model gives an adjusted $R^2$ of 0.733, which is lover than on SimLex-999, but is still high. Table~\ref{tab:men-ablation} shows partial $R^2$ scores for the explored parameters. The most influential parameter is \texttt{neg}, followed by \texttt{freq} and similarity. This is different in case of SimLex-999 where these parameters influence ``order'' is reversed.

\begin{figure}[h]
% \begin{wrapfigure}{O}{0.5\textwidth}
  % \vspace{-30pt}
  \centering

  \includegraphics[width=0.5\textwidth]{supplement/figures/men-interaction-neg}

  \caption{MEN influence of \texttt{neg}.}
  \label{fig:men-neg}
\end{figure}

\texttt{neg} with $k = 2$ is preferable for spaces with dimensionality less than 20000, for spaces with more dimensions,
\todo{compare with Levy, they also suggest $k=5$.}
$k = 5$ is more beneficial (Figure~\ref{fig:men-neg}).  We, however, expect that for spaces with more than 500000 dimensions even higher values should be preferred. This contrasts with the heuristics derived from SimLex-999, where single \texttt{neg} value of 0.7 is chosen.

% \begin{figure}[h]
% \begin{wrapfigure}{O}{0.5\textwidth}
  % \vspace{-30pt}
  \centering

  \includegraphics[width=0.5\textwidth]{supplement/figures/men-interaction-freq}

  \caption{MEN influence of \texttt{freq}.}
  \label{fig:men-freq}
\end{figure}

Regarding the frequency component, $\log n$ outperforms all other choices (Figure~\ref{fig:men-freq}). It is exactly same choice as for heuristics based on SimLex-999.

\begin{figure}[b]
% \begin{wrapfigure}{O}{0.5\textwidth}
  % \vspace{-30pt}
  \centering
  \begin{subfigure}[t]{0.49\textwidth}
    \hspace{-20pt}
    \includegraphics[width=1.1\textwidth]{supplement/figures/men-interaction-similarity}

  \caption{similarity}
  \label{fig:men-similarity}
  \end{subfigure}
  \begin{subfigure}[t]{0.49\textwidth}
    \includegraphics[width=\textwidth]{supplement/figures/men-interaction-discr}

  \caption{\texttt{discr}}
  \label{fig:men-discr}
  \end{subfigure}

  \caption{MEN.}
\end{figure}

Correlation is the preferred similarity measure (Figure~\ref{fig:men-similarity}), again this is inline with the choice based on SimLex-999.

% \begin{figure}[h]
% \begin{wrapfigure}{O}{0.5\textwidth}
  % \vspace{-30pt}
  \centering

  \includegraphics[width=0.5\textwidth]{supplement/figures/men-interaction-discr}

  \caption{MEN influence of \texttt{discr}.}
  \label{fig:men-discr}
\end{figure}

SPMI is the preferred discriminativeness (Figure~\ref{fig:men-discr}), however it is closely followed by CPMI and SCPMI. This contrasts with SimLex-999, where SCPMI is preferred, however in both cases the difference between the two discriminativeness choices is minimal.

\begin{figure}[b]
%\begin{wrapfigure}[9]{O}{0.5\textwidth}
  % \vspace{-30pt}
  \centering

  \includegraphics[width=0.5\textwidth]{supplement/figures/men-interaction-cds}

  \caption{MEN influence of \texttt{cds}.}
  \label{fig:men-cds}
\end{figure}

Global context probability gives on average higher results for MEN (Figure~\ref{fig:men-cds}), while SimLex-999 prefers context distribution smoothing (Figure~\ref{fig:SimLex999-cds}).

\subsection{Difference between Max selection and heuristics on MEN}

\begin{table}[b]
  \centering

  \begin{tabular}{rrlllll}
\toprule
 dimensionality &    men &  freq & discr &     cds & neg &   similarity \\
\midrule
           1\,000 &  0.684 &  logn &  spmi &  global &   2 &  correlation \\
           2\,000 &  0.721 &  logn &  spmi &  global &   2 &  correlation \\
           3\,000 &  0.730 &  logn &  spmi &  global &   2 &  correlation \\
           5\,000 &  0.735 &  logn &  spmi &  global &   2 &  correlation \\
          10\,000 &  0.745 &  logn &  spmi &  global &   2 &  correlation \\
          20\,000 &  0.757 &  logn &  spmi &  global &   5 &  correlation \\
          30\,000 &  0.759 &  logn &  spmi &  global &   5 &  correlation \\
          40\,000 &  0.759 &  logn &  spmi &  global &   5 &  correlation \\
          50\,000 &  0.758 &  logn &  spmi &  global &   5 &  correlation \\
\bottomrule
\end{tabular}


  \caption{MEN selection based on heuristics}
  \label{tab:men-heuristics-selection}
\end{table}


The two selection procedures agree on fewer parameters than the ones bases on SimLex-999. Both agree on discrimination ($\log n$) and similarity score for spaces with dimensionality greater than 10000 (correlation). While SCPMI is chosen by Max selection, SPMI is preferred by the selection based on heuristics, however the difference between the two is minimal. In contrast to the Max selection, which chooses the models with context distribution smoothing, heuristics prefer models with global context probabilities. Also, heuristics pick models with higher shifting values $\alpha$ (2 and 5), in contrast to Max selection, where 0.7 and 1 are picked. Table~\ref{tab:men-heuristics-selection} summarises the parameter selection based on heuristics.

The average difference between Max selection and heuristics is 0.008.

\section{Selected model transfer to another dataset}
\label{sec:select-model-transf}

\subsection{Difference between heuristics based on MEN and SimLex-999}

Heuristics based on MEN agree with ones bases on SimLex-999 on two parameters: frequency ($\log n$) and similarity (correlation). The methods disagree on \texttt{discr} (SCPMI versus SPMI, but again the difference is neglectable), context distribution (smoothed versus global) and shifting parameter, for which higher values for MEN are preferred.

\subsection{From SimLex-999 to MEN}

\begin{figure}
  \centering

  \begin{subfigure}[t]{0.49\textwidth}
    \includegraphics[width=\textwidth]{supplement/figures/SimLex999-transfer}
    \caption{Transfer from SimLex-999 to MEN}
    \label{fig:SimLex999-transfer}
  \end{subfigure}
  \begin{subfigure}[t]{0.49\textwidth}
    \includegraphics[width=\textwidth]{supplement/figures/men-transfer}
    \caption{Transfer from MEN to SimLex-999}
    \label{fig:men-transfer}
  \end{subfigure}

  \caption{Model transfer between lexical evaluation datasets}
  \label{fig:lexical-transfer}
\end{figure}

%%% Local Variables:
%%% mode: latex
%%% TeX-master: "../thesis"
%%% End:


The models selected using heuristics based on the SimLex-999 dataset perform well on MEN: for all dimensions the selected models are close to the best possible score (Figure~\ref{fig:SimLex999-transfer}). The average difference with the upper bound is 0.006.

The max based selection comes close to the upper bound for models with dimensionality greater than 5000. The average difference with the upper bound is 0.039.

In this case, heuristic based selection leads to better performance than the Max based selection.

\subsection{From MEN to SimLex-999}

Heuristics transferred from MEN to SimLex-999 behave less efficient, they do not always outperform Max selection, though for highly dimensional spaces the difference decreases (Figure~\ref{fig:men-transfer}). The average difference is 0.062, which is ten times more than the transition from SimLex-999 to MEN.

Max selection neither picks the best possible results when transferred from MEN to SimLex-999, however the average difference is lower: it is 0.042. This is similar to the transition in other direction.

Max based selection leads to better performance than the heuristics for MEN.

\section{Universal parameter selection for lexical datasets}
\label{sec:universal-lexical-param-selection}

\begin{figure}
  \centering

  \begin{subfigure}[t]{0.49\textwidth}
    \includegraphics[width=\textwidth]{supplement/figures/lexical-results-SimLex999}
    \caption{SimLex-999.}
    \label{fig:lexical-results-simlex}
  \end{subfigure}
  \begin{subfigure}[t]{0.49\textwidth}
    \includegraphics[width=\textwidth]{supplement/figures/lexical-results-men}
    \caption{MEN.}
    \label{fig:lexical-results-men}
  \end{subfigure}

  \caption{Performance of models based on the selection over the average lexical performance.}
  \label{fig:lexical-results}
\end{figure}

%%% Local Variables:
%%% mode: latex
%%% TeX-master: "../thesis"
%%% End:


Figure~\ref{fig:lexical-results} shows performance of the models based on the average of the normalised scores over SimLex-999 and MEN. The performance of selected models on both datasets and the normalised average is shown on Table~\ref{tab:lexical-max-selection} (Max selection) and Table~\ref{tab:lexical-heuristics-selection} (selection based on heuristics).
%
\todo[noline]{Compare with Baroni and Levy.}

\subsection{Max selection}
\label{sec:max-selection}

In general, the more dimensions, the better the results are. The selection yields the best results at $D = 50000$ for SimLex-999 and at $D = 3000$ for MEN. While for SimLex-999,  the Max selection approaches the upper limit after 20000 dimensions; for MEN, it peaks and slightly deviates from the upper bound as the dimensionality increases.

The Max parameter selection based on the combination of the two lexical datasets is closer to the Max selection based on SimLex-999 (Table~\ref{tab:Simlex999-max-selection}) than on MEN (Table~\ref{tab:men-max-selection}).

\begin{table}
  \centering

  \begin{tabular}{rrrrlllrl}
\toprule
 dimensionality &  SimLex999 &   men &  lexical &  freq &  discr &     cds &  neg & similarity \\
\midrule
           1\,000 &       0.35 &  0.68 &     0.89 &     1 &   spmi &  global &  1.4 &        cos \\
           2\,000 &       0.36 &  0.72 &     0.94 &  logn &  scpmi &  global &  1.0 &        cos \\
           3\,000 &       0.36 &  0.74 &     0.94 &  logn &  scpmi &       1 &  0.7 &        cos \\
           5\,000 &       0.36 &  0.74 &     0.95 &  logn &  scpmi &       1 &  0.7 &        cos \\
          1\,0000 &       0.37 &  0.75 &     0.97 &  logn &  scpmi &       1 &  0.7 &        cos \\
          2\,0000 &       \textbf{0.38} &  \textbf{0.76} &     \textbf{0.99} &  logn &  scpmi &    0.75 &  0.7 &        cos \\
          3\,0000 &       \textbf{0.38} &  \textbf{0.76} &     \textbf{0.99} &  logn &  scpmi &    0.75 &  0.7 &        cos \\
          4\,0000 &       \textbf{0.38} &  \textbf{0.76} &     \textbf{0.99} &  logn &  scpmi &    0.75 &  0.7 &        cos \\
          5\,0000 &       \textbf{0.38} &  \textbf{0.76} &     \textbf{0.99} &  logn &  scpmi &    0.75 &  0.7 &        cos \\
\bottomrule
\end{tabular}


  \caption[Lexical (combined SimLex-999 and MEN) Max selection]{Lexical (combined SimLex-999 and MEN) Max selection. For the
    individual dataset scores, the scores in bold are indistinguishable from the
  highest score. For SimLex-999, the highest score is 0.385 and the scores above
0.362 are indistinguishable. Form MEN, the highest score is 0.762 and the scores
above 0.749 are indistinguishable. The highest combined score is 0.992 and the scores above 0.954 are indistinguishable.}
  \label{tab:lexical-max-selection}
\end{table}


\subsection{Heuristics}

\begin{wraptable}[11]{O}{0.5\textwidth}
  % \vspace{-1em}
  \centering

  \begin{tabular}{lr}
\toprule
      parameter &  partial $R^2$ \\
\midrule
     similarity &     0.30 \\
            neg &     0.28 \\
           freq &     0.23 \\
 dimensionality &     0.09 \\
          discr &     0.09 \\
            cds &     0.08 \\
\bottomrule
\end{tabular}


  \caption{Lexical feature ablation}
  \label{tab:lexical-ablation}
\end{wraptable}


The linear model achieves an adjusted $R^2$ of 0.817, which is less then $R^2 = 0.867$ of SimLex-999, but is greater than the $R^2 = 0.733$ of MEN. Table~\ref{tab:lexical-ablation} shows partial $R^2$ for each parameter, the most influential are similarity, \texttt{neg} and \texttt{freq}.

\begin{figure}[b]
% \begin{wrapfigure}{O}{0.5\textwidth}
  % \vspace{-30pt}
  \centering

  \begin{subfigure}[t]{0.49\textwidth}
    \hspace{-20pt}
    \includegraphics[width=1.1\textwidth]{supplement/figures/lexical-interaction-similarity}

    \caption{Similarity}
    \label{fig:lexical-similarity}
  \end{subfigure}
  \begin{subfigure}[t]{0.49\textwidth}
    \includegraphics[width=\textwidth]{supplement/figures/lexical-interaction-neg}

    \caption{\texttt{neg}}
    \label{fig:lexical-neg}
  \end{subfigure}
  \caption{Lexical the influence of similarity and \texttt{neg}}
\end{figure}

Correlation is the similarity measure of choice (Figure~\ref{fig:lexical-similarity}).

% \begin{figure}[h]
% \begin{wrapfigure}{O}{0.5\textwidth}
  % \vspace{-30pt}
  \centering

  \includegraphics[width=0.5\textwidth]{supplement/figures/lexical-interaction-neg}

  \caption{Lexical influence of \texttt{neg}.}
  \label{fig:lexical-neg}
\end{figure}

For the models with dimensionality less than 20000 shifting should be used with $k = 1$, otherwise $k = 2$ is preferred (Figure~\ref{fig:lexical-neg}).

\begin{figure}[t]
% \begin{wrapfigure}{O}{0.5\textwidth}
  % \vspace{-30pt}
  \centering

  \begin{subfigure}[t]{0.49\textwidth}
    \includegraphics[width=\textwidth]{supplement/figures/lexical-interaction-freq}

    \caption{Lexical influence of \texttt{freq}.}
    \label{fig:lexical-freq}
  \end{subfigure}
  \begin{subfigure}[t]{0.49\textwidth}
    \includegraphics[width=\textwidth]{supplement/figures/lexical-interaction-discr}

    \caption{Lexical influence of \texttt{discr}.}
    \label{fig:lexical-discr}
  \end{subfigure}

  \caption{Lexical influence of \texttt{freq} and \texttt{discr}}
\end{figure}

$\log n$ on average performs the best as the frequency component (Figure~\ref{fig:lexical-freq}).

% \begin{figure}[h]
% \begin{wrapfigure}{O}{0.5\textwidth}
  % \vspace{-30pt}
  \centering

  \includegraphics[width=0.5\textwidth]{supplement/figures/lexical-interaction-discr}

  \caption{Lexical influence of \texttt{discr}.}
  \label{fig:lexical-discr}
\end{figure}

SCPMI is the preferred discrimination component, but SCPMI is very close to it (Figure~\ref{fig:lexical-discr}).

% \begin{figure}[h]
\begin{wrapfigure}[7]{O}{0.5\textwidth}
  \vspace{-30pt}
  \centering

  \includegraphics[width=0.5\textwidth]{supplement/figures/lexical-interaction-cds}

  \caption{Lexical influence of \texttt{cds}.}
  \label{fig:lexical-cds}
\end{wrapfigure}

Global context probabilities on average behave the best (Figure~\ref{fig:lexical-cds}).

\begin{table}[t]
  \centering

  \begin{tabular}{rrrrlllll}
\toprule
 dimensionality &  SimLex999 &    men &  lexical &  freq &  discr &     cds & neg &   similarity \\
\midrule
           1\,000 &      0.334 &  0.681 &    0.875 &  logn &  scpmi &  global &   1 &  correlation \\
           2\,000 &      0.349 &  0.718 &    0.918 &  logn &  scpmi &  global &   1 &  correlation \\
           3\,000 &      0.350 &  0.726 &    0.924 &  logn &  scpmi &  global &   1 &  correlation \\
           5\,000 &      0.353 &  0.733 &    0.933 &  logn &  scpmi &  global &   1 &  correlation \\
          10\,000 &      \textbf{0.362} &  0.743 &    0.951 &  logn &  scpmi &  global &   1 &  correlation \\
          20\,000 &      \textbf{0.367} &  \textbf{0.759} &    0.968 &  logn &  scpmi &  global &   2 &  correlation \\
          30\,000 &      \textbf{0.372} &  \textbe{0.761} &    0.976 &  logn &  scpmi &  global &   2 &  correlation \\
          40\,000 &      \textbf{0.373} &  \textbf{0.760} &    0.977 &  logn &  scpmi &  global &   2 &  correlation \\
          50\,000 &      \textbe{0.376} &  \textbf{0.759} &    \textbe{0.980} &  logn &  scpmi &  global &   2 &  correlation \\
\bottomrule
\end{tabular}


  \caption{Lexical (combined SimLex-999 and MEN) selection based on heuristics}
  \label{tab:lexical-heuristics-selection}
\end{table}


\subsection{Comparison with single dataset based selections}

Both selection methods mostly agree on frequency ($\log n$) and discriminativeness (SCPMI).

Context probability distribution smoothing varies between the selection methods, but follows the corresponding procedures based on MEN.

The Max based selection for \texttt{neg} follows the Max selection on SimLex-999.

Even though the similarity choice is different between the Max and heuristic selections, it is consistent with SimLex-999 in both cases and with MEN for the heuristic-based selection.

For the Max-based selection, the average difference is 0.020 on SimLex-999 and 0.004 for MEN.

For the heuristics-based selection, the average difference is 0.048 for SimLex-999 and 0.010 for MEN.

Max selection behaves better than the heuristics based on the average difference, but we can not check how well these two selections behave on other lexical datasets.

Based on the experiments, \logNSCPMI/ with shifting close to 1 is the quantification of choice for the lexical tasks, however more work needs to be done to find robust choice for context distribution smoothing and similarity measure choice.

\chapter{Similarity of sentences}
\label{sec:sentential}

\todo[inline]{Selection description: global best; best operator, its general behaviour, general behaviour of others; parameters.}

\section{KS14}
\label{sec:ks14}

\subsection{Max selection}
\label{sec:max-selection-ks14}

\begin{figure}[b]
  \centering

    \includegraphics[width=\textwidth]{supplement/figures/ks14-results}
  \caption{KS14 results.}
  \label{fig:ks14-results}
\end{figure}

%%% Local Variables:
%%% mode: latex
%%% TeX-master: "../thesis"
%%% End:


Figure~\ref{fig:ks14-results} shows the performance of compositional model on the sentence similarity dataset KS14 \cite{kartsadrqpl2014}.
All operators outperform the non-compositional \texttt{head} operator. Table~\ref{tab:ks14-max-selection} shows the performance of models selected by Max selection together with the selected parameters.

\todo[noline]{Significance test}
Kronecker with low dimensions and thus with correlation as the similarity measure gives the highest scores. As the dimensionality increases, Kronecker performance stays constant. Addition is slightly better than multiplication, but the performance of both peaks at 2000 dimensions and decreases as the dimensionality increases.

\texttt{Head} parameters are similar to the lexical Max selection (Table~\ref{tab:lexical-max-selection}), with an exception of \texttt{neg}, where values similar to MEN (Table~\ref{tab:men-max-selection}) are chosen.

All compositional operators agree in the choice of \texttt{freq} ($\log n$), \texttt{discr} (SCPMI) and similarity (correlation, note that Kronecker was tested only with the inner product for $D > 3000$ because of limited computational resources).

Compositional operators perform best with constant \texttt{freq} of 1, in contrast to the lexical setting, where $\log n$ is more beneficial. This might be because during composition the $\log n$ term dominates over the PMI value and minimises its effect.

Local context probabilities perform better in compositional tasks. Multiplication benefits from the unsmoothed distribution probability, while highly dimensional models perform best with smoothing ($\alpha = 0.75$). The only exception are additive models with $D < 5000$, where global probabilities perform best.

For low dimensional spaces, addition performs best with sparse spaces ($k > 1$, $D < 5000$), but for high dimensional spaces, addition performs best  with dense spaces ($k = 0,7$, $D \geq 5000$).

Multiplication, independently of dimensionality, performs best with dense spaces ($k = 0.2$).

Kronecker, in contrast to addition, performs best with dense low dimensional models ($k = 0.2$, $D < 5000$) and sparser high dimensional models ($k = 0.7$, $D \geq 5000$). But this difference might depend on the similarity measure, which is inner product for $D \geq 5000$.

\subsection{Heuristics}
\label{sec:heuristics}

\begin{wraptable}[11]{O}{0.5\textwidth}
  %\vspace{-10pt}
  \centering

  \begin{tabular}{lr}
\toprule
      parameter &  partial $R^2$ \\
\midrule
            neg &  0.33 \\
           freq &  0.31 \\
       operator &  0.30 \\
            cds &  0.14 \\
     similarity &  0.06 \\
          discr &  0.05 \\
 dimensionality &  0.03 \\
\bottomrule
\end{tabular}


  \caption{KS14 feature ablation}
  \label{tab:ks14-ablation}
\end{wraptable}


The linear model achieves an $R^2 = 0.794$. The partial $R^2$ are shown on Table~\ref{tab:ks14-ablation}. The most influential parameters are \texttt{neg}, \texttt{freq}, compositional operator and \texttt{cds}. Interestingly, similarity has much less influence on this compositional dataset than on lexical datasets, where for Sim-Lex-999 (Table~\ref{tab:SimLex999-ablation}) and combined (Table~\ref{tab:lexical-ablation}) it is the most influencing parameter. Also, note that dimensionality has the lowest partial $R^2$.

\subsubsection{Shifting}

\begin{figure}
% \begin{wrapfigure}{O}{0.5\textwidth}
  % \vspace{-30pt}
  \centering

  \begin{subfigure}[t]{\textwidth}
    \includegraphics[width=1.1\textwidth]{supplement/figures/ks14-interaction-neg}

  \caption{\texttt{neg}}
  \label{fig:ks14-neg}
  \end{subfigure}

  \begin{subfigure}[t]{\textwidth}
    \includegraphics[width=1.1\textwidth]{supplement/figures/ks14-interaction-freq}

  \caption{\texttt{freq}}
  \label{fig:ks14-freq}
  \end{subfigure}

  \caption{KS14 influence of \texttt{neg} and \texttt{freq}}
\end{figure}

For \texttt{head}, the best \texttt{neg} choice of $k$ is 1 for spaces with dimensionality less than 5000 (Figure~\ref{fig:ks14-neg}). For $5000 \leq D < 30000$, \texttt{head} behaves best with $k = 1.4$ and for $D \geq 3000$ \texttt{neg} should be set to 2.

For addition, spaces with $D < 20000$ should be used with $k = 1.4$, and with $k = 2$ otherwise.

For multiplication as with addition, there are three most beneficial choices: for $D < 10000$ $k = 0.5$, for $10000 \leq D < 30000$ $k = 0.7$ and, finally, for $D > 30000$ $k = 1$.

\todo{See whether the trend ``if small then dense and if big then sparse'' holds.}
Kronecker shows similar behaviour of $k$ as dimensionality increases as multiplication, but prefers sparser spaces: for $D < 3000$ $k = 0.5$, for $3000 \leq D < 20000$ and $D \geq 20000$ $k = 1$.

\subsubsection{Frequency}
The best option of \texttt{freq} for \texttt{head} is $\log n$ (Figure~\ref{fig:ks14-freq}). The constant frequency 1 is very close $\log n$, but its performance declines for spaces with $D > 20000$.

For addition, frequency should be set to 1 for spaces with $D < 5000$ and to $\log n$ otherwise.

There is one choice of frequency for multiplication: 1.

\todo{Check whether \texttt{freq}:n is bad for operators that do multiplication.}
Kronecker follows addition with regard to \texttt{freq}, but the split point is $D = 10000$: low dimensional spaces should be used with constant frequency 1, and high dimensional spaces with $\log n$.

\subsubsection{Context distribution smoothing}

\begin{figure}
% \begin{wrapfigure}{O}{0.5\textwidth}
  % \vspace{-30pt}
  \centering

  \begin{subfigure}[t]{\textwidth}
    \includegraphics[width=1.1\textwidth]{supplement/figures/ks14-interaction-cds}

  \caption{\texttt{cds}}
  \label{fig:ks14-cds}
  \end{subfigure}

  \begin{subfigure}[t]{\textwidth}
    \includegraphics[width=1.1\textwidth]{supplement/figures/ks14-interaction-similarity}

  \caption{\texttt{similarity}}
  \label{fig:ks14-similarity}
  \end{subfigure}

  \caption{KS14 influence of \texttt{cds} and \texttt{similarity}}
\end{figure}

\texttt{Head} with spaces with dimensionality less than 20000 should be used with global probabilities, and more dimensional models should be used with smoothed local probabilities: $\alpha = 0.75$ (Figure~\ref{fig:ks14-cds}).

All other operators perform best with global context probability.

\subsubsection{Similarity}
\texttt{Head} with spaces with $D < 20000$ performs best with cosine similarity, while more dimensional models prefer correlation as the similarity measure (Figure~\ref{fig:ks14-similarity}).

Other operators prefer correlation, however its not known correlation and cosine performance with Kronecker with spaces with $D > 3000$ as those were not tested, leaving inner product as the only choice.

\subsubsection{Discriminativeness}

\begin{figure}[b]
% \begin{wrapfigure}{O}{0.5\textwidth}
  % \vspace{-30pt}
  \centering

  \includegraphics[width=1.1\textwidth]{supplement/figures/ks14-interaction-discr}

  \caption{KS14 influence of \texttt{discr}}
  \label{fig:ks14-discr}
\end{figure}

\texttt{Head} with $D < 20000$ prefers SCPMI as the discriminativeness weighting. SPMI is preferred otherwise (Figure~\ref{fig:ks14-discr}).

For addition, SPMI is the better choice. For multiplication, SCPMI is more beneficial.

For Kronecker the two choices are very close to each other. However, for spaces with dimensionality less than 20000, SPMI is slightly better; for spaces with greater dimensionality, SCPMI is better.

\subsection{Difference between Max selection and heuristics on KS14}

\todo[noline]{Difference between heuristics based on various datasets.}

Table~\ref{tab:ks14-heuristics-selection}\todo{finish}\ldots

%  
\clearpage
\KOMAoptions{paper=A3,pagesize}
\recalctypearea

% \newgeometry{margin=1.5cm}

\begin{landscape}
\thispagestyle{empty} %% Remove header and footer.

\begin{figure}
  \centering

  \begin{subfigure}[t]{0.6\textwidth}
    \includegraphics[width=\textwidth]{supplement/figures/KS14-max_-selection-freq}
    \caption{Max. Freq.}
    \label{fig:}
  \end{subfigure}
  \begin{subfigure}[t]{0.6\textwidth}
    \includegraphics[width=\textwidth]{supplement/figures/KS14-cross_validation-selection-freq}
    \caption{CV. Freq.}
    \label{fig:}
  \end{subfigure}

  \begin{subfigure}[t]{0.6\textwidth}
    \includegraphics[width=\textwidth]{supplement/figures/KS14-max_-selection-neg}
    \caption{Max. Neg.}
    \label{fig:}
  \end{subfigure}
  \begin{subfigure}[t]{0.6\textwidth}
    \includegraphics[width=\textwidth]{supplement/figures/KS14-cross_validation-selection-neg}
    \caption{CV. Neg.}
    \label{fig:}
  \end{subfigure}

  \begin{subfigure}[t]{0.6\textwidth}
    \includegraphics[width=\textwidth]{supplement/figures/KS14-max_-selection-cds}
    \caption{Max. CDS.}
    \label{fig:}
  \end{subfigure}
  \begin{subfigure}[t]{0.6\textwidth}
    \includegraphics[width=\textwidth]{supplement/figures/KS14-cross_validation-selection-cds}
    \caption{CV. CDS.}
    \label{fig:}
  \end{subfigure}

  \begin{subfigure}[t]{0.6\textwidth}
    \includegraphics[width=\textwidth]{supplement/figures/KS14-max_-selection-similarity}
    \caption{Max. Sim.}
    \label{fig:}
  \end{subfigure}
  \begin{subfigure}[t]{0.6\textwidth}
    \includegraphics[width=\textwidth]{supplement/figures/KS14-cross_validation-selection-similarity}
    \caption{CV. Sim.}
    \label{fig:}
  \end{subfigure}

  \begin{subfigure}[t]{0.6\textwidth}
    \includegraphics[width=\textwidth]{supplement/figures/KS14-max_-selection-discr}
    \caption{Max. Discr.}
    \label{fig:}
  \end{subfigure}
  \begin{subfigure}[t]{0.6\textwidth}
    \includegraphics[width=\textwidth]{supplement/figures/KS14-cross_validation-selection-discr}
    \caption{CV. Discr.}
    \label{fig:}

  \end{subfigure}


  \caption{KS14 selection.}
  \label{fig:selection_ks14}
\end{figure}

\end{landscape}
\restoregeometry

\clearpage
\KOMAoptions{paper=A4,pagesize}
\recalctypearea



\section{GS11}
\label{sec:gs11}

\todo[inline]{Mention \cite{hashimoto-tsuruoka:2016:P16-1}}

\subsection{Max selection}
\label{sec:max-selection-gs11}

\begin{figure}[t]
  \centering

    \includegraphics[width=\textwidth]{supplement/figures/gs11-results}
  \caption{GS11 results.}
  \label{fig:gs14-results}
\end{figure}

%%% Local Variables:
%%% mode: latex
%%% TeX-master: "../thesis"
%%% End:


Figure~\ref{fig:gs14-results} shows performance of the compositional models on the transitive verb disambiguation task \cite{Grefenstette:2011:ESC:2145432.2145580}. Table~\ref{tab:gs11-max-selection} shows the selected model performance together with chosen parameters.

Multiplication with 20000 dimensions gives the highest result of 0.532. Kronecker gets close with the score of 0.516 with $D = 50000$. Addition does not outperform the \texttt{head} operator: addition scores 0.338, while \texttt{head}'s best performance is 0.432.

\texttt{Head}'s behaviour is unstable for dimensions less than 20000, and its best behaviour might be the case of overfitting similarly with SimLex-999. Models with dimensions greater than 20000 behave similarly to each other, even though the parameters are different.

In general, parameter selection is very different than the one based on KS14 (Table~\ref{tab:ks14-max-selection}). Compositional operators behave best with $\log n$ frequency, especially Kronecker. PMI often outperforms other discriminativeness components in case of \texttt{head} and addition. Global context probability estimation behaves better than local. Correlation is not always the best similarity measure.

Addition behaviour degrades as dimensionality increases, multiplication behaviour increases, but becomes unstable for spaces with high number of dimensions. Kronecker depends the least on the dimensionality.

Addition works best with dense models. Multiplication and Kronecker prefer dense low dimensional space and sparse high dimensional spaces.


\subsection{Heuristics}
\label{sec:heuristics-gs11}

The linear model achieves a $R^2 = 0.753$. The partial $R^2$ scores are shown on Table~\ref{tab:gs11-ablation}. The most influential parameters are a compositional operator, \texttt{freq} and \texttt{neg}. This is the same as in case of KS14, but in the reversed order \ref{tab:ks14-ablation}. However, for GS11 the operator choice is the most important, in case of KS14, the partial $R^2$ scores of the top 3 parameters are much closer to each other.

\todo[noline]{wraptable \ref{tab:gs11-ablation}}
% \begin{wraptable}[3]{O}{0.5\textwidth}
\begin{table}[b]
  % \vspace{-30pt}
  \centering

  \begin{tabular}{lr}
\toprule
      parameter &  partial $R^2$ \\
\midrule
       operator &  0.367 \\
           freq &  0.213 \\
            neg &  0.179 \\
     similarity &  0.088 \\
            cds &  0.054 \\
          discr &  0.039 \\
 dimensionality &  0.039 \\
\bottomrule
\end{tabular}


  \caption{GS11 feature ablation.}
  \label{tab:gs11-ablation}
\end{table}


\subsubsection{Frequency}

\begin{figure}
% \begin{wrapfigure}{O}{0.5\textwidth}
  % \vspace{-30pt}
  \centering

  \begin{subfigure}[t]{\textwidth}
    \includegraphics[width=1.1\textwidth]{supplement/figures/gs11-interaction-freq}

  \caption{\texttt{freq}}
  \label{fig:gs11-freq}
  \end{subfigure}

  \begin{subfigure}[t]{\textwidth}
    \includegraphics[width=1.1\textwidth]{supplement/figures/gs11-interaction-neg}

  \caption{\texttt{neg}}
  \label{fig:gs11-neg}
  \end{subfigure}

  \caption{GS11.}
\end{figure}


$\log n$ on average behaves best for all operators (Figure~\ref{fig:gs11-freq}).

\subsubsection{Shifting}

\texttt{Head} on average works best with shifted models. For models with dimensionality less than 3000, $k = 0.5$, otherwise $k = 0.7$ is more beneficial (Figure~\ref{fig:gs11-neg}).

For addition, models without shifting behaves best for $D < 20000$, however for more dimensional spaces, $k = 0.2$ should be preferred.

Multiplication also works best with unshifted low dimensional spaces ($D < 5000$) and with $k = 0.7$ for high dimensional spaces.

Kronecker prefers shifting. For spaces with dimensionality less then 20000 $k = 0.7$ and $k = 1$ otherwise.

\subsubsection{Similarity}

\begin{figure}
% \begin{wrapfigure}{O}{0.5\textwidth}
  % \vspace{-30pt}
  \centering

  \begin{subfigure}[t]{\textwidth}
    \includegraphics[width=1.1\textwidth]{supplement/figures/gs11-interaction-similarity}

  \caption{Similarity}
  \label{fig:gs11-similarity}
  \end{subfigure}

  \begin{subfigure}[t]{\textwidth}
    \includegraphics[width=1.1\textwidth]{supplement/figures/gs11-interaction-cds}

  \caption{\texttt{cds}}
  \label{fig:gs11-cds}
  \end{subfigure}

  \caption{GS11 influence of similarity and \texttt{cds}}
\end{figure}


\texttt{Head} and multiplication work best with cosine similarity. Addition with correlation and Kronecker with inner product (Figure~\ref{fig:gs11-similarity}).

\subsubsection{Context distribution smoothing}

\texttt{Head} with $D < 10000$ works best with global context probabilities. For more dimensional spaces, local context probabilities $\alpha = 1$ should be preferred (Figure~\ref{fig:gs11-cds}).

Addition works best with local probabilities. In the low dimensional case, when $D < 20000$, unsmoothed estimation ($\alpha = 1$) is preferred and $\alpha = 0.75$ should be chosen otherwise.

Multiplication works best with global context probabilities. Kronecker with smoothed local ($\alpha = 0.75$).

\subsubsection{Discriminativeness}

\begin{figure}
% \begin{wrapfigure}{O}{0.5\textwidth}
  % \vspace{-30pt}
  \centering

  \includegraphics[width=1.1\textwidth]{supplement/figures/gs11-interaction-discr}

  \caption{GS11 \texttt{discr}.}
  \label{fig:gs11-discr}
\end{figure}


\texttt{Head} works best with SPMI, but SCPMI is very close (Figure~\ref{fig:gs11-discr}).

Addition works best with PMI for $D < 20000$ and SCPMI otherwise.

Multiplication is similar to addition that it prefers PMI in the low dimensional case and SCPMI in the high dimensional case, but the change happens at 5000 dimensions.

Kronecker with less than 5000 dimensions prefers SCPMI and SPMI otherwise.

\todo[noline]{Heuristics selection table.}

\subsection{Difference between Max selection and heuristics on GS11}

Only logarithmic frequency component ($\log n$) was chosen by heuristics (Table~\ref{tab:gs11-heuristics-selection}), while there is a mix of 1 and $\log n$ in the Max selection (Table~\ref{tab:gs11-max-selection}).

Kronecker and most of multiplication discriminativeness choices agree, while for \texttt{head} and addition there is little agreement of parameter selection. Same goes for context distribution smoothing and shifting.

Similarity choice is same for Kronecker and addition, but \texttt{head} and multiplication---according to heuristics---should be used with cosine similarity, while there is no single metric that leads to maximum performance.

\section{PhraseRel}
\label{sec:phraserel-experiment}

\todo[inline]{Outdated: heuristics!!!}

\subsection{Max selection}
\label{sec:max-selection-phraserel}

\begin{figure}
  \centering

    \includegraphics[width=\textwidth]{supplement/figures/phraserel-results}
  \caption{PhraseRel results.}
  \label{fig:phraserel-results}
\end{figure}

%%% Local Variables:
%%% mode: latex
%%% TeX-master: "../thesis"
%%% End:


Figure~\ref{fig:phraserel-results} shows the performance of the models on the PhraseRel dataset. All operators outperform non-compositional \texttt{head} baseline. Table~\ref{tab:phraserel-max-selection} shows the models that yield the best result together with model parameters.

Multiplication, in general, outperforms all other operators, and with dimensions of 10000 and 20000 gets the perfect score. Model performance weakly depends on the dimensionality for all operators.

Addition and Kronecker achieve the best score with constant frequency, \texttt{head} works best with linear and multiplication with sublinear ($\log n$).

SPMI is the preferred discriminativeness component for the low-dimensional spaces ($D < 10000$) for \texttt{head}, otherwise, SCPMI is the best behaving \texttt{discr}. For addition and the spaces with $D > 1000$, SPMI is the best, while for the spaces with the same dimensionality multiplication prefers CPMI. Kronecker, most of the times, prefers SCPMI.

\texttt{Head} with dimensions less than 20000 works best with local smoothed context probabilities, however for more dimensional spaces global context probabilities are more competitive. Addition, contrary, prefers smoothed local context probabilities for spaces with dimensions more than 5000. Multiplication exhibits different pattern: when a model contains few dimensions, it prefers local smoothed context probabilities, and for highly-dimensional spaces it prefers local, but unsmoothed context probabilities. Kronecker is inconsistent with regards to the choice of \texttt{cds}, but models with $D \geq 30000$ global context probabilities perform the best.

Regarding shifting, \texttt{head} prefers sparse spaces $k > 1$, but as dimensionality increases the optimal $k$ values decreases. Addition does not show a consistent behaviour with regard to this parameter. Multiplication, in general, benefits from dense unshifted spaces. Kronecker works best with sparse spaces with increasing sparsity as the dimensionality increases.

\texttt{Head} benefits from the correlation as the similarity measure, as does multiplication. Addition works best with correlation with spaces $D < 10000$, and with the inner product for more dimensional spaces. Multiplication works best with correlation. Kronecker, for spaces with less than 5000 dimensions, works best with correlation and with the inner product otherwise.

\subsection{Heuristics}
\label{sec:heuristics-phraserel}

\begin{wraptable}[5]{O}{0.5\textwidth}
% \begin{table}[b]
  \vspace{-50pt}
  \centering

  \begin{tabular}{lr}
\toprule
      parameter &  partial $R^2$ \\
\midrule
            neg &      0.579 \\
       operator &      0.355 \\
            cds &      0.075 \\
           freq &      0.043 \\
     similarity &      0.033 \\
 dimensionality &      0.032 \\
          discr &      0.021 \\
\bottomrule
\end{tabular}


  \caption{PhraseRel feature ablation.}
  \label{tab:phraserel-ablation}
\end{wraptable}


\begin{figure}
% \begin{wrapfigure}{O}{0.5\textwidth}
  % \vspace{-30pt}
  \centering

  \begin{subfigure}[t]{\textwidth}
    \includegraphics[width=1.1\textwidth]{supplement/figures/phraserel-interaction-neg}

  \caption{\texttt{neg}}
  \label{fig:phraserel-neg}
  \end{subfigure}

  \begin{subfigure}[t]{\textwidth}
    \includegraphics[width=1.1\textwidth]{supplement/figures/phraserel-interaction-cds}

  \caption{\texttt{cds}}
  \label{fig:phraserel-cds}
  \end{subfigure}

  \caption{PhraseRel.}
\end{figure}

The linear model achieves an $R^2 = 0.856$. The partial $R^2$ scores are shown on Table~\ref{tab:phraserel-ablation}. The most influential parameters are \texttt{neg}, operator and \texttt{cds}, but the first two have the partial $R^2$ scores much higher then the other parameters. Table~\ref{tab:phraserel-heuristics-selection} shows the performance of the picked models.

\subsubsection{Shifting}
\label{sec:shifting-phraserel}

\texttt{Head} should be used with $k = 1.4$, addition should be used with $k = 2$ and multiplication should be used with $k = 0.5$ (Figure~\ref{fig:phraserel-neg}).

Kronecker has three optimal values of $k$ that is proportional to dimensionality. For models with dimensionality less than 5000, $k = 0.5$ is preferred; for $5000 \geq D < 20000$, the most beneficial choice of \texttt{neg} is $k = 1$; finally, for spaces with more than 20000 dimensions, $k$ should be set to 1.4.

\subsubsection{Context distribution smoothing}
\label{sec:cont-distr-smooth-phraserel}

The best choice of \texttt{head} is dependant on dimensionality: spaces with less than 10000 dimensions benefit from smoothed local context probabilities ($\alpha = 0.75$) (Figure~\ref{fig:phraserel-cds}). Addition and multiplication work best with global context probabilities, while Kronecker prefers unsmoothed local probabilities ($\alpha = 1$).

\subsubsection{Frequency}
\label{sec:frequency-phraserel}

\begin{figure}[b]
% \begin{wrapfigure}{O}{0.5\textwidth}
  % \vspace{-30pt}
  \centering

  \begin{subfigure}[t]{\textwidth}
    \includegraphics[width=1.1\textwidth]{supplement/figures/phraserel-interaction-freq}

  \caption{\texttt{freq}}
  \label{fig:phraserel-freq}
  \end{subfigure}

  \begin{subfigure}[t]{\textwidth}
    \includegraphics[width=1.1\textwidth]{supplement/figures/phraserel-interaction-similarity}

  \caption{similarity}
  \label{fig:phraserel-similarity}
  \end{subfigure}

  \caption{PhraseRel.}
\end{figure}


\texttt{Head} works best with linear frequency, but the difference with other options is small (Figure~\ref{fig:phraserel-freq}).

Addition benefits from linear frequency, sublinear frequency is very close.

Multiplication works best with sublinear frequency, but linear is very close to it.

Finally, Kronecker works best with $\log n$ with spaces with dimensionality less than 5000, and with linear frequency with more dimensional spaces.

\subsubsection{Similarity}
\label{sec:similarity-phraserel}

\texttt{Head} works best with correlation as the similarity measure with models with $D < 5000$, and with cosine for more dimensional ones (Figure~\ref{fig:phraserel-similarity}). Note, however, that the difference between the two is very small.

Addition benefits from cosine when $D < 20000$ and from inner product otherwise. But in case of addition, all three similarity measures are close to each others.

Multiplication works best with correlation.

Where tested, correlation behaves best with Kronecker.

\subsubsection{Discriminativeness}
\label{sec:discriminativeness-phraserel}

\begin{figure}[b]
% \begin{wrapfigure}{O}{0.5\textwidth}
  % \vspace{-30pt}
  \centering

  \includegraphics[width=1.1\textwidth]{supplement/figures/PhraseRel-interaction-discr}

  \caption{PhraseRel \texttt{discr}}
  \label{fig:phraserel-discr}
\end{figure}


\texttt{Head} is the only ``operator'' that prefers from different discriminativeness components depending on dimensionality. For models with $D < 5000$, SPMI is the best, while for other dimensions SCPMI is more competitive.

Addition and Kronecker benefit from SPMI, while multiplication from SCPMI.

\subsection{Difference between Max selection and heuristics on PhraseRel}
\label{sec:diff-phraserel}

Manual parameter selection is more stable than the one based on the maximum values. However in cases where different parameters are picked, there is little or no difference between these parameter choices. For example studied similarity measures yield similar average performance for addition, see Figure~\ref{fig:phraserel-freq}.

Manual heuristics do not pick the best result of 1 (Table~\ref{tab:phraserel-max-selection}), but are close with a multiplicative model with 20000 and 30000 dimensions yielding the score of 0.964 (Table~\ref{tab:phraserel-heuristics-selection}).

The average relative difference between the max selection and the selection based on heuristics is 0.022 for \texttt{head}, 0.072 for addition, 0.041 for multiplication and 0.061 for Kronecker.

\section{Selected model transfer across the datasets}
\label{sec:select-model-transf-comp}

\subsection{Difference between heuristics}
\label{sec:diff-betw-heur-comp}

There is little agreement on parameter selection based on heuristics among the 3 compositional datasets. The only consistent choice is global context probability (\texttt{cds}) and SCPMI discriminativeness for multiplicative models.

There is more pairwise agreement, for example similarity based on correlation for additive models on KS14 and GS11 and $\log n$ frequency for multiplicative models between GS11 and PhraseRel. The pairwise agreement might be a sign of overfitting, because there is no clear pattern. On the other side, the difference in performance between parameter choices might be neglectable as some parameters consistently show low $R^2$ scores, for example \texttt{discr}.

\subsection{From KS14}
\label{sec:from-ks14}

\begin{figure}[t]
  \centering
    \includegraphics[width=\textwidth]{supplement/figures/KS14-transfer}
    \caption{Transfer from KS14.}
    \label{fig:ks14-transfer}
\end{figure}

%%% Local Variables:
%%% mode: latex
%%% TeX-master: "../thesis"
%%% End:


Figure~\ref{fig:ks14-transfer} show the behaviour of models selected on the KS14 when they are transferred to GS11 and PhraseRel. During the transfer there is little difference in performance between the selection methods, except of multiplicative models where heuristics show better performance and 5000 dimensional Kronecker where heuristics give lower results than the max-based selection.

\todo[noline]{Significance tests}
Heuristic-based selecting on average is closer to the upper bound. When transferred to GS11 the average difference with the upper bound is 0.335 for max and 0.238 for heuristics. When transferred to PhraseRel the average difference is 0.093 for max and 0.091 for heuristics.

\subsection{From GS11}
\label{sec:from-gs11}

\begin{figure}[b]
  \centering
    \includegraphics[width=\textwidth]{supplement/figures/GS11-transfer}
    \caption{Transfer from GS11.}
    \label{fig:gs11-transfer}
\end{figure}

%%% Local Variables:
%%% mode: latex
%%% TeX-master: "../thesis"
%%% End:


Figure~\ref{fig:gs11-transfer} shows that there is little difference between max and heuristic selections. In case of \texttt{head} composition, heuristics lead to higher performance, while for low-dimensional multiplicative models heuristics fall behind the max selection on the KS14 dataset.

\todo[noline]{Significance tests}
When GS11 models are transferred to KS14, the average difference with the upper bound is 0.119 and 0.106 for max and heuristics respectively. For the transfer to PhraseRel, the differences are 0.133 for max and 0.188 for heuristics. Again, the heuristic-based selection outperforms the max based.

\subsection{From PhraseRel}
\label{sec:from-phraserel}

\begin{figure}[b]
  \centering
    \includegraphics[width=\textwidth]{supplement/figures/PhraseRel-transfer}
    \caption{Transfer from Phraserel}
    \label{fig:phraserel-transfer}
\end{figure}

%%% Local Variables:
%%% mode: latex
%%% TeX-master: "../thesis"
%%% End:


Figure~\ref{fig:phraserel-transfer} shows that the performance of models based on PhraseRel is less stable, especially for selection by maximum performance.

\todo[noline]{Significance tests}
Transfer to KS14 yields the average differences of 0.152 for max and 0.136 for heuristics. Transfer to GS11 yields the average differences of 0.454 for max and 0.509 for heuristics. Note that the PhraseRel to GS11 transfer is the only case where max selection on average is better than heuristics.

\todo[inline]{Here we showed, in contrast to the lexical evaluation, that the heuristics might be indeed be more beneficial than the max-based selection.}

\section{Universal parameter selection for compositional datasets}
\label{sec:robust-param-comp-selecion}

\begin{figure}
  \centering

  \begin{subfigure}[t]{\textwidth}
    \includegraphics[width=\textwidth]{supplement/figures/compositional-results-ks14}
    \caption{KS14.}
    \label{fig:compositional-results-ks14}
  \end{subfigure}

  \begin{subfigure}[t]{\textwidth}
    \includegraphics[width=\textwidth]{supplement/figures/compositional-results-gs11}
    \caption{GS11.}
    \label{fig:compositional-results-gs11}
  \end{subfigure}

  \begin{subfigure}[t]{\textwidth}
    \includegraphics[width=\textwidth]{supplement/figures/compositional-results-phraserel}
    \caption{PhraseRel.}
    \label{fig:compositional-results-phraserel}
  \end{subfigure}


  \caption{Performance of models based on the selection over the average compositional performance}
  \label{fig:compositional-results}
\end{figure}

%%% Local Variables:
%%% mode: latex
%%% TeX-master: "../thesis"
%%% End:


Figure~\ref{fig:compositional-results} show the performance of the models based on the combined selection over the KS14, GS11 and PhraseRel datasets. The performance of selected models together with the selected parameters is shown on the Table~\ref{tab:compositional-max-selection} (Max selection) and Table~\ref{tab:compositional-heuristics-selection} (selection based on heuristics).

\subsection{Max selection}
\label{sec:max-selection-compositional}

Models with many dimensions not always perform better than their low-dimensional counterparts. Particularly, only \texttt{head} and multiplication benefit from the high number of dimensions. Addition and Kronecker are closer to the upper bound with the dimensionality of few thousands.

\subsection{Heuristics}
\label{sec:heuristics-compositional}

\begin{wraptable}[13]{O}{0.5\textwidth}
% \begin{table}
  \vspace{1em}
  \centering

  \begin{tabular}{lr}
\toprule
      parameter &  partial $R^2$ \\
\midrule
            neg &           0.40 \\
           freq &           0.29 \\
       operator &           0.21 \\
            cds &           0.15 \\
     similarity &           0.08 \\
          discr &           0.06 \\
 dimensionality &           0.05 \\
\bottomrule
\end{tabular}


  \caption{Compositional feature ablation}
  \label{tab:compositional-ablation}
\end{wraptable}

The linear model achieves the $R^2 = 0.769$. Table~\ref{tab:compositional-ablation} shows the partial $R^2$ values for the parameters. The most influential parameters are \texttt{neg}, \texttt{freq} and compositional operator.

\subsubsection{Neg}
\label{sec:neg-compositional}

\begin{figure}
% \begin{wrapfigure}{O}{0.5\textwidth}
  % \vspace{-30pt}
  \centering

  \begin{subfigure}[t]{\textwidth}
    \includegraphics[width=1.1\textwidth]{supplement/figures/compositional-interaction-neg}

  \caption{\texttt{neg}}
  \label{fig:compositional-neg}
  \end{subfigure}

  \begin{subfigure}[t]{\textwidth}
    \includegraphics[width=1.1\textwidth]{supplement/figures/compositional-interaction-freq}

  \caption{\texttt{freq}}
  \label{fig:compositional-freq}
  \end{subfigure}

  \caption{Compositional influence of \texttt{neg} and \texttt{freq}}
\end{figure}


For \texttt{head} and models with $D < 10000$ the \texttt{neg} should be set to 1, otherwise it should be 1.4 (Figure~\ref{fig:compositional-neg}).

For addition, 1 is the best choice of \texttt{neg}.

Multiplication benefits from denser spaces. If the dimensionality is less than 10000, then \texttt{neg} should be set to 0.5, otherwise 0.7 is a good choice.

Kronecker benefits from the \texttt{neg} of 0.7 if $D < 10000$ and from 1 for the more dimensional cases. This is similar to multiplication, but Kronecker prefers less dense vectors.

\subsubsection{Freq}
\label{sec:freq-compositional}

$\log n$ is the frequency value of choice of all operators with an exception of multiplication, where the constant frequency is preferred (Figure~\ref{fig:compositional-freq}).

\subsubsection{Context distribution smoothing}
\label{sec:cont-distr-smooth-compositional}

\begin{figure}[b]
% \begin{wrapfigure}{O}{0.5\textwidth}
  % \vspace{-30pt}
  \centering

  \begin{subfigure}[t]{\textwidth}
    \includegraphics[width=1.1\textwidth]{supplement/figures/compositional-interaction-cds}

  \caption{\texttt{cds}}
  \label{fig:compositional-cds}
  \end{subfigure}

  \begin{subfigure}[t]{\textwidth}
    \includegraphics[width=1.1\textwidth]{supplement/figures/compositional-interaction-similarity}

  \caption{Similarity}
  \label{fig:compositional-similarity}
  \end{subfigure}

  \caption{Compositional influence of \texttt{cds} and similarity}
\end{figure}


As Figure~\ref{fig:compositional-cds} shows, global context probability is the preferred choice of context probability in all cases, but in cases of Kronecker composition with $D > 3000$, where smoothed local probabilities are better ($\alpha = 0.75$).

\subsubsection{Similarity}
\label{sec:similarity-compositional}

Correlation is the dominant choice of the similarity measure (Figure~\ref{fig:compositional-similarity}). However, cosine is preferred in the case of \texttt{head}, with $D > 5000$, and inner product is the only choice for the composition with Kronecker with $D > 3000$.

\subsubsection{Discr}
\label{sec:discr-compositional}

\begin{figure}[b]
% \begin{wrapfigure}{O}{0.5\textwidth}
  % \vspace{-30pt}
  \centering

  \includegraphics[width=1.1\textwidth]{supplement/figures/compositional-interaction-discr}

  \caption{Compositional \texttt{discr}.}
  \label{fig:compositional-discr}
\end{figure}


SPMI is the choice of \texttt{discr} that leads to the best average performance in most cases (Figure~\ref{fig:compositional-discr}). However the difference between SPMI and SCPMI is very small.

The exceptions are Multiplicative composition with $D \geq 10000$ and Kronecker with $D \leq 5000$ where SCPMI outperforms SPMI.

\subsection{Comparison with the single dataset beased selections}
\label{sec:comp-with-single-comp}

Manual selection based on a combination of the compositional datasets is more stable with regards to the chosen parameter values than the selection based on the highest values, even though manual selection does not always achieve the performance of Max selection, see Figure~\ref{fig:compositional-results}.

The average difference with the upper bound is 0.040 and 0.041 for Max and heuristics, respectively, when applied to KS14. For GS11, the difference is 0.045 (Max) and 0.127 (Heuristics). For PhraseRel, the difference is 0.055 (Max) and 0.084 (Heuristics).

\chapter[Universal model]{Universal model for both lexical and compositional tasks}
\label{sec:universal-param-selection}

\todo[inline]{From lexical to compositional.}

\section{Operator dependant (universal) models}
\label{sec:model-selection}

\subsection{Max selection}
\label{sec:max-selection-universal}

Table~\ref{tab:universal-max-selection} shows the performance of the models selected by a combined score of the lexical and compositional datasets. Parameter selection is much more stable than on all previous max-based selections. $\log n$ is a dominant \texttt{freq} choice, cosine is the measure of choice for multiplication and Kronecker, if available. Correlation is the similarity measure for additive composition. Interestingly, if shifting applied, then 1 or 0.7 are chosen as \texttt{neg} values.

\todo[noline]{This is one of the major findings.}
Compositional operator preference depends on the focus of a model.
Addition with many dimensions gives the best results on lexical tasks: 0.384 on SimLex-999 and 0.761 on MEN. Kronecker, on the other side, gives the highest values for compositional datasets: 0.798 on KS14, 0.514 on GS11 and 0.964 on PhraseRel. Multiplication, however, gives the highest ``combined'' score of 0.945, showing that the selection is the closest to the upper bound.

\subsection{Heuristics}
\label{sec:heuristics-universal}

\begin{wraptable}[14]{O}{0.5\textwidth}
  % \vspace{-4ex}
  \centering

  \begin{tabular}{lr}
\toprule
      parameter &  partial $R^2$ \\
\midrule
           freq &      0.323 \\
            neg &      0.289 \\
     similarity &      0.218 \\
            cds &      0.102 \\
          discr &      0.091 \\
 dimensionality &      0.066 \\
       operator &      0.050 \\
\bottomrule
\end{tabular}


  \caption{Universal (operator dependant) feature ablation}
  \label{tab:universal-ablation}
\end{wraptable}


Performance of the models selected manually shown on Table~\ref{tab:universal-heuristics-selection}. Again, there is a lot consistency between parameters. The linear model achieves $R^2 = 0.828$. The most influencing parameters are \texttt{freq}, \texttt{neg} and a similarity measure, refer to Table~\ref{tab:universal-ablation}.

Heuristics for addition choose models that score the highest on lexical tasks: 0.384 on SimLex-999 and 0.764 on MEN (Table~\ref{tab:universal-heuristics-selection}). Moreover, with more than 20000 dimensions there is no difference between the selection procedures (Max or heuristics) of the additive and Kronecker-based models.

Kronecker is strong in compositional tasks scoring 0.795 on KS14, 0.516 on GS11 and 0.929 on PhraseRel (Table~\ref{tab:universal-heuristics-selection}).

Multiplication, again, is a compromise between the two: it gives the highest combined score of 0.941. The highest Kronecker's combined is 0.913, while addition's highest score is only 0.843.

\subsection{Comparison}
\label{sec:comparison-universal}

\begin{figure}
  \centering

  \begin{subfigure}[t]{\textwidth}
    \includegraphics[width=\textwidth]{supplement/figures/universal-results-simlex999}
    \caption{SimLex999.}
    \label{fig:universal-results-simlex}
  \end{subfigure}

  \begin{subfigure}[t]{\textwidth}
    \includegraphics[width=\textwidth]{supplement/figures/universal-results-men}
    \caption{MEN.}
    \label{fig:universal-results-men}
  \end{subfigure}


  \begin{subfigure}[t]{\textwidth}
    \includegraphics[width=\textwidth]{supplement/figures/universal-results-ks14}
    \caption{KS14.}
    \label{fig:universal-results-ks14}
  \end{subfigure}

  \begin{subfigure}[t]{\textwidth}
    \includegraphics[width=\textwidth]{supplement/figures/universal-results-gs11}
    \caption{GS11.}
    \label{fig:universal-results-gs11}
  \end{subfigure}

  \begin{subfigure}[t]{\textwidth}
    \includegraphics[width=\textwidth]{supplement/figures/universal-results-PhraseRel}
    \caption{PhraseRel.}
    \label{fig:universal-results-phraserel}
  \end{subfigure}


  \caption{Performance of models based on the selection over the average universal performance}
  \label{fig:universal-results}
\end{figure}

%%% Local Variables:
%%% mode: latex
%%% TeX-master: "../thesis"
%%% End:


On lexical tasks, there is little difference between the model selection methods especially for spaces with more than 30000 dimensions, as Figures~\ref{fig:universal-results-simlex} and \ref{fig:universal-results-men} show.

On compositional tasks, dimensionality does not contribute as much as in case of lexical tasks, with an exception of addition on the GS11 dataset, where performance decreases as dimensionality increases.

\todo[noline]{One of the major findings.}
When the models that are selected on lexical tasks are applied in a compositional setting, they perform worse than the models selected based on the universal score. This suggests that a model that is good on lexical tasks will not necessarily perform well on a compositional task. In addition, sometimes this difference increases as dimensionality increases, for example this is the case for multiplication, the most notable difference is observed on KS14 (Figure~\ref{fig:universal-results-ks14}).

\todo[inline]{Compare the average differences of every dataset.}

\section{Operator independant universal models}

\todo[inline]{We have seen in the previous section that even though parameter selection is different between operators, there are choices that are shared. Given this and the fact that the difference between some of the choices is marginal, we try to look for truly universal parameters.}

\subsection{Max selection}
\label{sec:max-selection-single}

Table~\ref{tab:single-max-selection} shows \todo{describe how it was combined} the combined score for all datasets abstracting over a compositional operator.

The parameter selection shows a clear pattern. \texttt{freq} shows a clear pattern: low-dimensional spaces perform best with 1 as the frequency choice, while high dimensional models perform better with $\log n$. Cosine is a better suited similarity measure for models with few dimensions, correlation with many. Finally, global context probabilities are the best in a low-dimensional case, while local context probabilities perform best with many dimensions.

\todo[inline]{Average difference with the upper bound.}

\subsection{Heuristics}
\label{sec:heuristics-single}

Heuristics in general repeat the parameter selection choice, but the switch happens with more dimensions (at 20000, not at 5000) refer to Table~\ref{tab:single-heuristics-selection} for the results and Figure~\ref{fig:single-params} for the parameter behavior.

\begin{wraptable}[12]{O}{0.5\textwidth}
%\begin{table}[b]
  %\vspace{-2em}
  \centering

  \begin{tabular}{lr}
\toprule
      parameter &  partial $R^2$ \\
\midrule
           freq &   0.434 \\
     similarity &   0.269 \\
            neg &   0.196 \\
          discr &   0.092 \\
            cds &   0.090 \\
 dimensionality &   0.034 \\
\bottomrule
\end{tabular}


  \caption{Universal (operator independent) feature ablation}
  \label{tab:single-ablation}
  % \end{wraptable}
\end{wraptable}

The linear model gives $R^2 = 898$. The most influential parameters are \texttt{freq}, similarity measure and \texttt{neg}, refer to Table~\ref{tab:single-ablation}.

% \begin{wrapfigure}{O}{0.5\textwidth}
\begin{figure}[b]
  % \vspace{-30pt}
  \centering

  \begin{subfigure}[t]{0.49\textwidth}
  \includegraphics[width=\textwidth]{supplement/figures/single-interaction-freq}

  \caption{\texttt{freq}}
  \label{fig:single-similarity}
  \end{subfigure}
  \begin{subfigure}[t]{0.49\textwidth}

  \includegraphics[width=1.1\textwidth]{supplement/figures/single-interaction-similarity}

  \caption{Similarity measure}
  \label{fig:single-similarity}
  \end{subfigure}

  \begin{subfigure}[t]{0.49\textwidth}

  \includegraphics[width=\textwidth]{supplement/figures/single-interaction-neg}

  \caption{\texttt{neg}}
  \label{fig:single-neg}
  \end{subfigure}
  \begin{subfigure}[t]{0.49\textwidth}

  \includegraphics[width=\textwidth]{supplement/figures/single-interaction-discr}

  \caption{\texttt{discr}}
  \label{fig:single-discr}
  \end{subfigure}

  \begin{subfigure}[t]{0.49\textwidth}

  \includegraphics[width=\textwidth]{supplement/figures/single-interaction-cds}

  \caption{\texttt{cds}}
  \label{fig:single-cds}
  \end{subfigure}

  \caption{Single.}
  \label{fig:single-params}
\end{figure}


% \subsubsection{Tweaked IR evaluation}
% \label{sec:tweak-ir-eval}

% \todo[inline]{Discuss \cite{Milajevs:2015:IMN:2808194.2809448} and link to \ref{sec:phraserel}}

% \section{PhraseRel: relevance of sentences}
% \label{sec:sentential-relevance}

%%% Local Variables:
%%% mode: latex
%%% TeX-master: "thesis"
%%% End:

\chapter{Reflection}
\label{cha:reflection}

\section{The notion of similarity}
\label{sec:notion-similarity}

In contrast to linguistic datasets which contain randomly paired words from a broad selection, datasets that come from psychology contain entries that belong to a single category such as \textit{verbs of judging} \cite{FILLENBAUM197454} or \textit{animal terms} \cite{HENLEY1969176}. The reason for category oriented similarity studies is that ``stimuli can only be compared in so far as they have already been categorised as identical, alike, or equivalent at some higher level of abstraction'' \cite{turner1987rediscovering}. Moreover, because of the \emph{extension effect} \cite{medin1993respects}, the similarity of two entries in a context is less than the similarity between the same entries when the context is extended. ``For example, \textit{black} and \textit{white} received a similarity rating of 2.2 when presented by themselves; this rating increased to 4.0 when \textit{black} was simultaneously compared with \textit{white} and \textit{red} (\textit{red} only increased 4.2 to 4.9)'' \cite{medin1993respects}. In the first case \textit{black} and \textit{white} are more dissimilar because they are located on the extremes of the greyscale, but in the presence of \textit{red} they become more similar because they are both monochromes.

Both MEN and SimLex-999 provide pairs that do not share any similarity to control for false positives, and they do not control for the comparison scale. This makes similarity judgements ambiguous as it is not clear what low similarity values mean: incompatible notions or contrast in meaning. SimLex-999 assigns low similarity scores to the incompatible pairs (0.48, \textit{trick} and \textit{size}) and to antonymy (0.55, \textit{smart} and \textit{dumb}), but \textit{smart} and \textit{dumb} have relatively much more in common than \textit{trick} and \textit{size}!

\section{3 ways of measuring similarity}
\label{sec:3-ways-measuring}

\subsection{Similarity in context}
\label{sec:similarity-context}

A noun phrase can be similar to a noun as in \textit{female lion} and \textit{lioness}, and to another noun phrases as in \textit{yellow car} and \textit{cheap taxi}. The same similarity principle can be applied to phrases as to words. In this case, similarity is measured in \emph{context}, but is still comparison of the phrases' head words which meaning is modified by arguments they appear with \cite{Kintsch2001173,mitchell-lapata:2008:ACLMain,mitchell2010composition,Dinu:2010:MDS:1870658.1870771,Baroni2010nouns,thater-furstenau-pinkal:2011:IJCNLP-2011,Seaghdha:2011:PMS:2145432.2145545}. With verbs this idea can be applied to compare transitive verbs with intransitive. For example, \textit{to cycle} is similar to \textit{ride a bicycle}.

% TODO: mention \cite{wieting2015paraphrase}

Sentential similarity might be treated as the similarity of the heads in the contexts. That is, the similarity between \textit{sees} and \textit{notices} in \textit{John \textbf{sees} Mary} and \textit{John \textbf{notices} a woman}. This approach abstracts away grammatical difference between the sentences and concentrates on semantics and fits the proposed model as the respect for the head, which is a lexical entity, has to be found.

\subsection{Composition of similarity scores}
\label{sec:comp-simil-scor}

\cite{turney2012domain}

\subsection{Similarity of compounds}
\label{sec:similarity-compounds}

\section{Relevance: another semantic measure}
\label{sec:meas-relev-vect}


\section{Problems with model's hyperparameter selection during evaluation}
\label{sec:param-comb-model}


%%% Local Variables:
%%% mode: latex
%%% TeX-master: "thesis"
%%% End:

\chapter{Relationships between sentences}
\label{sec:sentential}

\lettrine[lines=5,loversize=0.25]{I}{n} this chapter, we study the relationship between lexical representations and various methods of composition and look for the optimal lexical representations for additive, multiplicative and Kronecker-based compositional methods.\footnotemark{} We do not make any assumption regarding lexical representations and test all of them to see their behavioural patterns in compositional setting.

\footnotetext{This chapter includes the results presented in \newcite{milajevs-EtAl:2014:EMNLP2014} at EMNLP 2014.}

Here we report the results on the compositional datasets KS14 \cite{kartsadrqpl2014}, GS11 \cite{Grefenstette:2011:ETV:2140490.2140497} and PhraseRel (Chapter~\ref{sec:phraserel}). For the first two datasets, the evaluation is done similarly to the lexical evaluation (Chapter~\ref{sec:lexical}) by computing Sperman-$\rho$ rank correlation between human judgements and model estimation. If a dataset provides several human judgements for a sentence pair, the judgements are averaged before computing the correlation. For PhraseRel, we report relevant@3, the measure that is the proportion of sentences for which the top-3 most similar neighbours contain at least one sentence that was judged relevant with respect to the source sentence.

We show that the optimal choice of lexical parameters depends on the method of composition, so, for example, with addition the vectors should be sparser than the vectors that are used with multiplication and Kronecker. We also show that the parameter choice that is optimal for lexical tasks is sub-optimal for compositional tasks especially for multiplication and Kronecker. 

\section{Experiments on KS14 dataset}
\label{sec:ks14}

KS14 is a sentence similarity dataset prepared by \citet{kartsadrqpl2014}. It consists of pairs of transitive sentences that are judged by similarity. The goal is to achieve high correlation with human judgements in predicting sentential similarity. We also report behaviour of a baseline operator \texttt{head}, which ignores the subject and the object of a sentence and makes the vector of a sentence equal to the vector of the head word, in our case the verb. The minimum significant difference for KS14 is $\sigma^{0.9}_{0.05} = 0.07$.

\subsection{Max selection}
\label{sec:max-selection-ks14}

Figure~\ref{fig:ks14-results} shows the performance of compositional models on the sentence similarity dataset KS14. All operators outperform the non-compositional \texttt{head} operator, addition and Kronecker statistically significantly outperform \texttt{head} operator. Table~\ref{tab:ks14-max-selection} shows the performance of models selected by Max selection together with the selected parameters.

% \todo[noline]{Significance test}
Kronecker with a few thousand dimensions and correlation as the similarity measure gives the highest scores, supporting H\ref{hyp:order} that word order is important in predicting similarity. As the dimensionality increases, Kronecker performance stays constant. Addition is slightly better than multiplication, but the performance of both peaks at 2\,000 dimensions and decreases as dimensionality increases. There is no statistically significant difference between the highest score of Kronecker and the highest scores of addition and multiplication.

\begin{figure}[b]
  \centering

    \includegraphics[width=\textwidth]{supplement/figures/ks14-results}
  \caption{KS14 results.}
  \label{fig:ks14-results}
\end{figure}

%%% Local Variables:
%%% mode: latex
%%% TeX-master: "../thesis"
%%% End:


Parameters of the baseline compositional method \texttt{head} are similar to the lexical Max selection (Table~\ref{tab:lexical-max-selection}), with an exception of \texttt{neg}, which controls vector sparsity, where higher values that are similar to MEN (Table~\ref{tab:men-max-selection}) are chosen.

All compositional operators agree in the choice of \texttt{freq} ($\log n$), \texttt{discr} (SCPMI) and similarity (correlation---note that Kronecker was tested only with the inner product for $D > 3\,000$ because of limited computational resources).

Compositional operators perform best with constant \texttt{freq} of 1, in contrast to the lexical setting, where $\log n$ is more beneficial. This might be because during composition the $\log n$ term dominates over the PMI value and minimises its effect.

Local context probabilities perform better in compositional tasks than in lexical tasks. Multiplication benefits from the unsmoothed distribution probability, while high-dimensional models perform best with smoothing ($\alpha = 0.75$), supporting H\ref{hyp:cds} that context smoothing is needed only for high-dimensional models. The only exceptions are additive models with $D < 5\,000$, where global probabilities perform best.

For low-dimensional spaces, addition performs best with sparse spaces ($k > 1$, $D < 5\,000$), but for high-dimensional spaces, addition performs best  with dense spaces ($k = 0,7$, $D \geq 5\,000$). This is against H\ref{hyp:neg} that states the opposite.

Multiplication, independently of dimensionality, performs best with dense spaces ($k = 0.2$).

Kronecker---in contrast to addition---performs best with dense low-dimensional models ($k = 0.2$, $D < 5\,000$) and sparser high-dimensional models ($k = 0.7$, $D \geq 5\,000$), which complies with H\ref{hyp:neg}. However, this difference might be explained by the change of the similarity measure, which is the inner product for $D \geq 5\,000$.

\subsection{Heuristics}
\label{sec:heuristics}

\begin{wraptable}[11]{O}{0.5\textwidth}
  %\vspace{-10pt}
  \centering

  \begin{tabular}{lr}
\toprule
      parameter &  partial $R^2$ \\
\midrule
            neg &  0.33 \\
           freq &  0.31 \\
       operator &  0.30 \\
            cds &  0.14 \\
     similarity &  0.06 \\
          discr &  0.05 \\
 dimensionality &  0.03 \\
\bottomrule
\end{tabular}


  \caption{KS14 feature ablation}
  \label{tab:ks14-ablation}
\end{wraptable}


The linear model achieves an $R^2$ of 0.794. The partial $R^2$s are shown in Table~\ref{tab:ks14-ablation}. The most influential parameters are \texttt{neg}, \texttt{freq}, compositional operator and \texttt{cds}. Interestingly, similarity has much less influence on this compositional dataset than on lexical datasets, where for Sim-Lex-999 (Table~\ref{tab:SimLex999-ablation}) and combined (Table~\ref{tab:lexical-ablation}) it is the most influential parameter. Also, note that dimensionality has the lowest partial $R^2$.

\subsubsection{Shifting}

For the baseline operator \texttt{head}, the best shifting choice of $k$ is 1 for spaces with dimensionality less than 5\,000 (Figure~\ref{fig:ks14-neg}). For $5\,000 \leq D < 30\,000$, \texttt{head} behaves best with $k = 1.4$. For $D \geq 3\,000$, $k$ should be set to 2.

For addition, spaces with $D < 20\,000$ should be used with $k = 1.4$, and with $k = 2$ otherwise.

For multiplication, there are three most beneficial choices: for $D < 10\,000$ $k = 0.5$, for $10\,000 \leq D < 30\,000$ $k = 0.7$ and, finally, for $D > 30\,000$ $k = 1$.

Kronecker shows a behaviour  similar to multiplication for $k$ as dimensionality increases, but prefers sparser spaces. For $D < 3\,000$: $k = 0.5$, for $3\,000 \leq D < 20\,000$: $k = 0.7$ and for $20\,000 \leq D$: $k = 1$.

All operators behave in accordance to H\ref{hyp:neg} that low-dimensional spaces benefit from being dense, while high-dimensional spaces benefit from being sparse.

\begin{figure}
% \begin{wrapfigure}{O}{0.5\textwidth}
  % \vspace{-30pt}
  \centering

  \begin{subfigure}[t]{\textwidth}
    \includegraphics[width=1.1\textwidth]{supplement/figures/ks14-interaction-neg}

  \caption{\texttt{neg}}
  \label{fig:ks14-neg}
  \end{subfigure}

  \begin{subfigure}[t]{\textwidth}
    \includegraphics[width=1.1\textwidth]{supplement/figures/ks14-interaction-freq}

  \caption{\texttt{freq}}
  \label{fig:ks14-freq}
  \end{subfigure}

  \caption{KS14 influence of \texttt{neg} and \texttt{freq}}
\end{figure}


\subsubsection{Frequency}
The best option of frequency for the baseline operator \texttt{head} is $\log n$ (Figure~\ref{fig:ks14-freq}). The constant frequency 1 is very close to $\log n$, but its performance declines for spaces with $D > 20\,000$.

For addition, frequency should be set to 1 for spaces with $D < 5\,000$ and to $\log n$ otherwise.

There is one choice of frequency for multiplication: 1. However, $\log n$ behaves similarly to it when $D \geq 3\,000$.

Kronecker follows addition with regard to frequency, but the split point is $D = 10\,000$: low-dimensional spaces should be used with constant frequency 1, and high-dimensional spaces with $\log n$.

Generally, H\ref{hyp:freq}---that non-constant frequency is beneficial for high-dimensional spaces---is supported by all operators in this dataset.

\subsubsection{Context distribution smoothing}

\begin{figure}
% \begin{wrapfigure}{O}{0.5\textwidth}
  % \vspace{-30pt}
  \centering

  \begin{subfigure}[t]{\textwidth}
    \includegraphics[width=1.1\textwidth]{supplement/figures/ks14-interaction-cds}

  \caption{\texttt{cds}}
  \label{fig:ks14-cds}
  \end{subfigure}

  \begin{subfigure}[t]{\textwidth}
    \includegraphics[width=1.1\textwidth]{supplement/figures/ks14-interaction-similarity}

  \caption{\texttt{similarity}}
  \label{fig:ks14-similarity}
  \end{subfigure}

  \caption{KS14 influence of \texttt{cds} and \texttt{similarity}}
\end{figure}

The baseline operator \texttt{head} with spaces with dimensionality less than 20\,000 should be used with global probabilities, and more dimensional models should be used with smoothed, local probabilities: $\alpha = 0.75$ (Figure~\ref{fig:ks14-cds}).

All other operators perform best with global context probability. Even though local context probability with $\alpha = 1$ is close to global, H\ref{hyp:cds} is not supported by this dataset: context distribution smoothing does not affect high-dimensional spaces.

\subsubsection{Similarity}
The baseline operator \texttt{head} on spaces with $D < 20\,000$ performs best with cosine similarity, while more dimensional models prefer correlation as the similarity measure (Figure~\ref{fig:ks14-similarity}), strictly according to H\ref{hyp:similarity}: the adjustment by the mean value that is performed by correlation is need by high-dimensional models.

Other operators work best with correlation. Addition and multiplication also support H\ref{hyp:similarity} that correlation is beneficial in high-dimensional spaces. In the case of multiplication, correlation dominates over cosine, even for small values of $D$. There is little to say about Kronecker, as it is tested only with the inner product for spaces with $D > 3\,000$, due to its computational complexity ($\mathcal{O}(n^2)$ with respect to the number of vector components).

\subsubsection{Discriminativeness}

\begin{figure}[b]
% \begin{wrapfigure}{O}{0.5\textwidth}
  % \vspace{-30pt}
  \centering

  \includegraphics[width=1.1\textwidth]{supplement/figures/ks14-interaction-discr}

  \caption{KS14 influence of \texttt{discr}}
  \label{fig:ks14-discr}
\end{figure}

The baseline operator \texttt{head} with $D < 20\,000$ prefers SCPMI as the discriminativeness weighting. SPMI is preferred otherwise (Figure~\ref{fig:ks14-discr}).

For addition, SPMI is the better choice. For multiplication, SCPMI is more beneficial, as expected by H\ref{hyp:comp-pmi-cpmi}: the compression of PMI values improves the performance of compositional models.

For Kronecker, the two choices are very close to each other. For spaces with dimensionality less than 20\,000, SPMI is slightly better; for spaces with greater dimensionality---SCPMI.

\subsection{Difference between Max selection and heuristics on KS14}

Table~\ref{tab:ks14-heuristics-selection} shows the selection based on heuristics, which is more homogeneous than the selection based on the highest score (Table~\ref{tab:ks14-max-selection}). Both methods agree on the similarity choice (with the exception of \texttt{head}).

Multiplication agrees on the majority of parameters, except \texttt{cds} and \texttt{neg}. Local probabilities ($\alpha = 1$) and $k = 0.2$ give the highest score, while manual selection picked global context probabilities with \texttt{neg} in the range of 0.5 to 1, in accordance with H\ref{hyp:neg} that high-dimensional spaces should be sparser.

The average difference between Max and heuristic-based selections is 0.022. Per operator, the differences are: 0.028 (\texttt{head}), 0.012 (addition), 0.018 (multiplication) and 0.028 (Kronecker). All values are within the 10\% set by H\ref{hyp:10percent}.

\section{Experiments on GS11 dataset}
\label{sec:gs11}

GS11 is a dataset of transitive sentences \cite{Grefenstette:2011:ESC:2145432.2145580,Grefenstette:2011:ETV:2140490.2140497}. It consists of ambiguous transitive verbs together with their arguments and two landmark verbs that each disambiguate a particulate sense of the ambiguous verb. Human judgements provide pairwise similarity scores between the sentence with the ambiguous verb and the two sentences with the landmark verbs.

\subsection{Max selection}
\label{sec:max-selection-gs11}

\begin{figure}[t]
  \centering

    \includegraphics[width=\textwidth]{supplement/figures/gs11-results}
  \caption{GS11 results.}
  \label{fig:gs14-results}
\end{figure}

%%% Local Variables:
%%% mode: latex
%%% TeX-master: "../thesis"
%%% End:


Figure~\ref{fig:gs14-results} shows performance of the compositional models on the verb disambiguation task. Table~\ref{tab:gs11-max-selection} shows the selected model performance together with chosen parameters.

Multiplication with 20\,000 dimensions gives the highest result of 0.532. Kronecker gets close with a score of 0.516 with $D = 50\,000$, giving no support to H\ref{hyp:order} that word order is important for similarity measurement. Addition does not outperform the baseline \texttt{head} operator: addition scores 0.338, while \texttt{head}'s best performance is 0.432.

The behaviour of the baseline operator \texttt{head} is unstable for dimensions less than 20\,000, and its best behaviour might be the case of overfitting similarly with SimLex-999. However, models with dimensions greater than 20\,000 yield similar scores, even though the parameters are different.

In general, Max parameter selection is very different than the one based on KS14 (Table~\ref{tab:ks14-max-selection}). Compositional operators behave best with $\log n$ frequency, especially Kronecker. PMI often outperforms other discriminativeness components in the case of \texttt{head} and addition. Global context probability estimation behaves better than local and correlation is not always the best similarity measure.

Addition's behaviour degrades as dimensionality increases, whereas multiplication's behaviour increases, but becomes unstable for spaces with more than 20\,000 dimensions. Kronecker depends on the dimensionality the least.

Addition works best with dense models. Multiplication and Kronecker prefer dense, low-dimensional spaces and sparse, high-dimensional spaces, supporting H\ref{hyp:neg} that a frequency component is required for high-dimensional spaces.

\subsection{Heuristics}
\label{sec:heuristics-gs11}

The linear model achieves an $R^2$ of 0.753. The partial $R^2$ scores are shown in Table~\ref{tab:gs11-ablation}. The most influential parameters are a compositional operator, \texttt{freq} and \texttt{neg}. This is the same as in the case of KS14, but in reverse order (Table~\ref{tab:ks14-ablation}).

\subsubsection{Frequency}

% \begin{wraptable}[3]{O}{0.5\textwidth}
\begin{table}[b]
  % \vspace{-30pt}
  \centering

  \begin{tabular}{lr}
\toprule
      parameter &  partial $R^2$ \\
\midrule
       operator &  0.367 \\
           freq &  0.213 \\
            neg &  0.179 \\
     similarity &  0.088 \\
            cds &  0.054 \\
          discr &  0.039 \\
 dimensionality &  0.039 \\
\bottomrule
\end{tabular}


  \caption{GS11 feature ablation.}
  \label{tab:gs11-ablation}
\end{table}

On average, $\log n$ behaves best for all operators (Figure~\ref{fig:gs11-freq}). For $D \leq 5\,000$, 1 is also a good frequency choice, supporting H\ref{hyp:freq} that non-constant frequency component is required for high-dimensional spaces.

\subsubsection{Shifting}

The baseline operator \texttt{head} on average works best with shifted models. For models with dimensionality less than 3\,000, $k = 0.5$ is best, otherwise $k = 0.7$ is more beneficial (Figure~\ref{fig:gs11-neg}).

For addition, models without shifting behave best for $D < 20\,000$, however, for more dimensional spaces, $k = 0.2$ should be preferred. This is a weak support of H\ref{hyp:neg} (high-dimensional vectors should be sparse) because unshifted spaces can be seen as shifted with a very small $\alpha$ value.

Multiplication also works best with unshifted low-dimensional spaces ($D < 5\,000$) and with $k = 0.7$ for high-dimensional spaces, supporting H\ref{hyp:neg}.

Kronecker prefers shifting. For spaces with dimensionality less than 20\,000 $k = 0.7$ and $k = 1$ otherwise. This is inline with H\ref{hyp:neg}.

\begin{figure}
% \begin{wrapfigure}{O}{0.5\textwidth}
  % \vspace{-30pt}
  \centering

  \begin{subfigure}[t]{\textwidth}
    \includegraphics[width=1.1\textwidth]{supplement/figures/gs11-interaction-freq}

  \caption{\texttt{freq}}
  \label{fig:gs11-freq}
  \end{subfigure}

  \begin{subfigure}[t]{\textwidth}
    \includegraphics[width=1.1\textwidth]{supplement/figures/gs11-interaction-neg}

  \caption{\texttt{neg}}
  \label{fig:gs11-neg}
  \end{subfigure}

  \caption{GS11.}
\end{figure}


\subsubsection{Similarity}

The baseline operator \texttt{head} and multiplication work best with cosine similarity, addition with correlation and Kronecker with inner product (Figure~\ref{fig:gs11-similarity}).

Addition strictly supports H\ref{hyp:similarity} that cosine is optimal for high-dimensional spaces, while multiplication supports it by behaving similarly to cosine and correlation.

\subsubsection{Context distribution smoothing}

The baseline operator \texttt{head} with $D < 10\,000$ works best with global context probabilities. For more dimensional spaces, local context probabilities $\alpha = 1$ should be preferred (Figure~\ref{fig:gs11-cds}).

Addition works best with local probabilities. In the low-dimensional case, when $D < 20\,000$, unsmoothed estimation ($\alpha = 1$) is preferred and $\alpha = 0.75$ should be chosen otherwise.

Multiplication works best with global context probabilities and Kronecker with smoothed local ($\alpha = 0.75$).

There is no support of H\ref{hyp:cds} (context distribution smoothing leads to optimal results with high-dimensional models) because global context probabilities outperform other choices for multiplication, addition behaves the same with all options and Kronecker works best with $\alpha = 0.75$.

\begin{figure}
% \begin{wrapfigure}{O}{0.5\textwidth}
  % \vspace{-30pt}
  \centering

  \begin{subfigure}[t]{\textwidth}
    \includegraphics[width=1.1\textwidth]{supplement/figures/gs11-interaction-similarity}

  \caption{Similarity}
  \label{fig:gs11-similarity}
  \end{subfigure}

  \begin{subfigure}[t]{\textwidth}
    \includegraphics[width=1.1\textwidth]{supplement/figures/gs11-interaction-cds}

  \caption{\texttt{cds}}
  \label{fig:gs11-cds}
  \end{subfigure}

  \caption{GS11 influence of similarity and \texttt{cds}}
\end{figure}


\subsubsection{Discriminativeness}

\begin{figure}
% \begin{wrapfigure}{O}{0.5\textwidth}
  % \vspace{-30pt}
  \centering

  \includegraphics[width=1.1\textwidth]{supplement/figures/gs11-interaction-discr}

  \caption{GS11 \texttt{discr}.}
  \label{fig:gs11-discr}
\end{figure}


The baseline operator \texttt{head} works best with SPMI, but SCPMI is very close (Figure~\ref{fig:gs11-discr}). Addition works best with PMI for $D < 20\,000$ and SCPMI otherwise.

Multiplication is similar to addition in that it prefers PMI in the low-dimensional case and SCPMI in the high-dimensional case, but the change happens at 5\,000 dimensions rather than 20\,000.

Kronecker with less than 5\,000 dimensions prefers SCPMI and SPMI, which is opposite to addition and multiplication.

This dataset does not give evidence to support H\ref{hyp:comp-pmi-cpmi}---PMI values should be compressed when used in compositional setting---because there is almost no difference between *PMI and *CPMI.

\subsection{Difference between Max selection and heuristics on GS11}

Only logarithmic frequency component ($\log n$) was chosen by heuristics (Table~\ref{tab:gs11-heuristics-selection}), while there is a mix of 1 and $\log n$ in the Max selection (Table~\ref{tab:gs11-max-selection}).

Kronecker and most of multiplication's discriminativeness choices agree, while for \texttt{head} and addition there is little agreement between parameter selection. The same goes for context distribution smoothing and shifting.

Similarity choice is the same for Kronecker and addition, but \texttt{head} and multiplication---according to heuristics---should be used with cosine similarity, while there is no single metric that leads to maximum performance.

The overall average normalised difference in results between Max and heuristic-based selections is 0.054. Per operator, the differences are: 0.090 \texttt{head}, 0.077 addition, 0.043 multiplication and 0.006 Kronecker. All the values are within the 10\% boundary set by H\ref{hyp:10percent}.

\section{Experiments on PhraseRel dataset}
\label{sec:phraserel-experiment}

PhraseRel is a dataset that is built for evaluation of distributional models, but instead of similarity judgements, relevance judgements are provided. The evaluation measure is relevance@3. This is the proportion of the retrieval results for which there is a relevant document among the top three ranked documents.

\subsection{Max selection}
\label{sec:max-selection-phraserel}

\begin{figure}
  \centering

    \includegraphics[width=\textwidth]{supplement/figures/phraserel-results}
  \caption{PhraseRel results.}
  \label{fig:phraserel-results}
\end{figure}

%%% Local Variables:
%%% mode: latex
%%% TeX-master: "../thesis"
%%% End:


Figure~\ref{fig:phraserel-results} shows the performance of the models on the PhraseRel dataset. All operators outperform the non-compositional \texttt{head} baseline. Table~\ref{tab:phraserel-max-selection} shows the models that yield the best result together with model parameters.

Multiplication, in general, outperforms all other operators, and with the dimensions of 10\,000 and 20\,000, gets the perfect score of 1. The fact that Kronecker (a word order sensitive operator) is outperformed by multiplication (a word order insensitive operator) gives no support of H\ref{hyp:order} that word order matters. Model performance weakly depends on the dimensionality for all operators. Multiplication and Kronecker show strong performance similarly to the preliminary study, where similarity was assumed to correspond to relevance \cite{Milajevs:2015:IMN:2808194.2809448}.

Addition and Kronecker achieve the best score with constant frequency, \texttt{head} works best with linear frequency and multiplication with sublinear ($\log n$) frequency.

SPMI is the preferred discriminativeness component for the low-dimensional spaces ($D < 10\,000$) for the baseline \texttt{head}, otherwise, SCPMI is the best behaving \texttt{discr}. For addition and the spaces with $D > 1\,000$, SPMI is the best, while for the spaces with the same dimensionality, multiplication prefers CPMI, which is in line with H\ref{hyp:comp-pmi-cpmi}: PMI values should be compressed when used in compositional tasks. Kronecker, most of the time, prefers SCPMI.

The baseline operator \texttt{head} with dimensions less than 20\,000 works best with local smoothed context probabilities, however, for more dimensional spaces, global context probabilities are more competitive. On the contrary, addition prefers smoothed local context probabilities for spaces with dimensions more than 5\,000. Multiplication exhibits different pattern: when a model contains few dimensions, it prefers local, smoothed context probabilities, and for high-dimensional spaces it prefers local, but unsmoothed context probabilities, which goes against H\ref{hyp:cds} (context distribution smoothing should be optimal with high-dimensional spaces). Kronecker is inconsistent with regards to the choice of \texttt{cds}, but for models with $D \geq 30\,000$, global context probabilities perform the best.

Regarding shifting, the baseline operator \texttt{head} prefers sparse spaces $k > 1$, but as dimensionality increases, the optimal $k$ values decrease. Addition does not show a consistent behaviour with regard to this parameter. Multiplication, in general, benefits from dense, unshifted spaces. Kronecker works best with sparse spaces with increasing sparsity as the dimensionality increases, supporting H\ref{hyp:neg}: low-dimensional models should be dense, but high-dimensional models should be sparse.

The baseline operator \texttt{head} benefits from the correlation as the similarity measure, as does multiplication. Addition works best with correlation with spaces $D < 10\,000$, and with the inner product for more dimensional spaces. Multiplication works best with correlation. Kronecker, for spaces with less than 5\,000 dimensions, works best with correlation and with the inner product, otherwise.

\subsection{Heuristics}
\label{sec:heuristics-phraserel}

\begin{wraptable}[5]{O}{0.5\textwidth}
% \begin{table}[b]
  \vspace{-50pt}
  \centering

  \begin{tabular}{lr}
\toprule
      parameter &  partial $R^2$ \\
\midrule
            neg &      0.579 \\
       operator &      0.355 \\
            cds &      0.075 \\
           freq &      0.043 \\
     similarity &      0.033 \\
 dimensionality &      0.032 \\
          discr &      0.021 \\
\bottomrule
\end{tabular}


  \caption{PhraseRel feature ablation.}
  \label{tab:phraserel-ablation}
\end{wraptable}


The linear model achieves an $R^2$ of 0.822. The partial $R^2$ scores are shown in Table~\ref{tab:phraserel-ablation}. The most influential parameters are \texttt{neg}, operator and \texttt{cds}, but the first two have partial $R^2$ scores much higher than the other parameters. Table~\ref{tab:phraserel-heuristics-selection} shows the performance of the chosen models.

\subsubsection{Shifting}
\label{sec:shifting-phraserel}

The baseline operator \texttt{head} should be used with $k = 1.4$, addition should be used with $k = 2$ and multiplication should be used with $k = 0.5$ (Figure~\ref{fig:phraserel-neg}).

Kronecker has three optimal values of $k$ that are proportional to dimensionality. For models with dimensionality less than 5\,000, $k = 0.5$ is preferred; for $5\,000 \leq D < 20\,000$, the most beneficial choice of \texttt{neg} is $k = 1$; finally, for spaces with more than 20\,000 dimensions, $k$ should be set to 1.4. This supports H\ref{hyp:neg} that model sparsity should increase as dimensionality increases.

\begin{figure}
% \begin{wrapfigure}{O}{0.5\textwidth}
  % \vspace{-30pt}
  \centering

  \begin{subfigure}[t]{\textwidth}
    \includegraphics[width=1.1\textwidth]{supplement/figures/phraserel-interaction-neg}

  \caption{\texttt{neg}}
  \label{fig:phraserel-neg}
  \end{subfigure}

  \begin{subfigure}[t]{\textwidth}
    \includegraphics[width=1.1\textwidth]{supplement/figures/phraserel-interaction-cds}

  \caption{\texttt{cds}}
  \label{fig:phraserel-cds}
  \end{subfigure}

  \caption{PhraseRel.}
\end{figure}


\subsubsection{Context distribution smoothing}
\label{sec:cont-distr-smooth-phraserel}

The best choice for the baseline operator \texttt{head} deponds on dimensionality: spaces with less than 10\,000 dimensions benefit from smoothed local context probabilities ($\alpha = 0.75$, Figure~\ref{fig:phraserel-cds}). Addition and multiplication work best with global context probabilities, while Kronecker prefers unsmoothed local probabilities ($\alpha = 1$).

\subsubsection{Frequency}
\label{sec:frequency-phraserel}

The baseline operator\texttt{head} works best with linear frequency, but the difference between other options is small (Figure~\ref{fig:phraserel-freq}). Addition benefits from linear frequency, but sublinear frequency is very close. Multiplication works best with sublinear frequency, but linear is very close to it. Finally, Kronecker works best with $\log n$ with spaces with dimensionality less than 5\,000, and with linear frequency with more dimensional spaces.

In general, H\ref{hyp:freq} (non-linear frequency should be used with high-dimensional spaces) holds, because there is no difference between 1 and $\log n$ choices in model performance.

\begin{figure}[b]
% \begin{wrapfigure}{O}{0.5\textwidth}
  % \vspace{-30pt}
  \centering

  \begin{subfigure}[t]{\textwidth}
    \includegraphics[width=1.1\textwidth]{supplement/figures/phraserel-interaction-freq}

  \caption{\texttt{freq}}
  \label{fig:phraserel-freq}
  \end{subfigure}

  \begin{subfigure}[t]{\textwidth}
    \includegraphics[width=1.1\textwidth]{supplement/figures/phraserel-interaction-similarity}

  \caption{similarity}
  \label{fig:phraserel-similarity}
  \end{subfigure}

  \caption{PhraseRel.}
\end{figure}


\subsubsection{Similarity}
\label{sec:similarity-phraserel}

For all operators, there is little difference between cosine and correlation, weakly supporting H\ref{hyp:similarity} that correlation should be used with high-dimensional spaces and cosine with low-dimensional spaces.

The baseline operator \texttt{head} works best with correlation as the similarity measure with models with $D < 5\,000$, and with cosine for more dimensional ones (Figure~\ref{fig:phraserel-similarity}). Note, however, that the difference between the two is very small.

Addition benefits from cosine when $D < 20\,000$ and from inner product otherwise. But, in the case of addition, all three similarity measures are close to each other.

Multiplication works best with correlation. Where tested, correlation behaves best with Kronecker.

\subsubsection{Discriminativeness}
\label{sec:discriminativeness-phraserel}

\begin{figure}[b]
% \begin{wrapfigure}{O}{0.5\textwidth}
  % \vspace{-30pt}
  \centering

  \includegraphics[width=1.1\textwidth]{supplement/figures/PhraseRel-interaction-discr}

  \caption{PhraseRel \texttt{discr}}
  \label{fig:phraserel-discr}
\end{figure}


The baseline \texttt{head} prefers different discriminativeness components depending on dimensionality. For models with $D < 5\,000$, SPMI is the best, while for other dimensions SCPMI is more competitive.

Addition and Kronecker benefit from SPMI, and multiplication from SCPMI, apart from the dimensionality of 10\,000. H\ref{hyp:comp-pmi-cpmi}, that PMI compression improves results in compositional tasks, is only supported by multiplication.

\subsection{Difference between Max selection and heuristics on PhraseRel}
\label{sec:diff-phraserel}

Manual parameter selection is more stable than the one based on maximum values. However, in cases where different parameters are picked, there is little or no difference between these parameter choices. For example, studied similarity measures yield similar average performance for addition, see Figure~\ref{fig:phraserel-freq}.

Manual heuristics do not pick the best result of 1 (Table~\ref{tab:phraserel-max-selection}), but are close with a multiplicative model with 20\,000 and 30\,000 dimensions, yielding a score of 0.964 (Table~\ref{tab:phraserel-heuristics-selection}).

The average relative difference between the Max selection and the selection based on heuristics is 0.022 for \texttt{head}, 0.072 for addition, 0.041 for multiplication and 0.061 for Kronecker.

Over all compositional methods, the difference is 0.049, which is within the 10\% limit set by H\ref{hyp:10percent}.

\section{Selected model transfer across the datasets}
\label{sec:select-model-transf-comp}

\begin{figure}[t]
  \centering
    \includegraphics[width=\textwidth]{supplement/figures/KS14-transfer}
    \caption{Transfer from KS14.}
    \label{fig:ks14-transfer}
\end{figure}

%%% Local Variables:
%%% mode: latex
%%% TeX-master: "../thesis"
%%% End:


\subsection{Difference between heuristics}
\label{sec:diff-betw-heur-comp}

There is little agreement on parameter selection based on heuristics among the three compositional datasets. The only consistent choice is global context probability (\texttt{cds}) and SCPMI discriminativeness for multiplicative models.

There is more pairwise agreement, for example, similarity based on correlation for additive models on KS14 and GS11 and $\log n$ frequency for multiplicative models between GS11 and PhraseRel. The pairwise agreement might be a sign of overfitting because there is no clear pattern. On the other side, the difference in performance between parameter choices might be negligible, as some parameters consistently show low $R^2$ scores, for example \texttt{discr}. Consequently, there is inconsistency in the supported hypotheses.

\subsection{Model transfer from KS14}
\label{sec:from-ks14}

Figure~\ref{fig:ks14-transfer} shows the behaviour of models selected on the KS14 when they are transferred to GS11 and PhraseRel. During the transfer, there is little difference in performance between the selection methods, except in multiplicative models where heuristics show better performance and 5\,000-dimensional Kronecker where heuristics give lower results than the Max-based selection.

Heuristic-based selection, on average, is closer to the upper bound than Max-based selection, supporting H\ref{hyp:overfitting} that Max selection overfits. However, both are beyond the 10\% boundary set by H\ref{hyp:10percent}. When transferred to GS11, the average difference with the upper bound is 0.335 for Max and 0.238 for heuristics. When transferred to PhraseRel the average difference is 0.093 for Max and 0.091 for heuristics.

\subsection{Model transfer from GS11}
\label{sec:from-gs11}

\begin{figure}[b]
  \centering
    \includegraphics[width=\textwidth]{supplement/figures/GS11-transfer}
    \caption{Transfer from GS11.}
    \label{fig:gs11-transfer}
\end{figure}

%%% Local Variables:
%%% mode: latex
%%% TeX-master: "../thesis"
%%% End:


Figure~\ref{fig:gs11-transfer} shows that there is little difference between Max and heuristic-based selections. In the case of \texttt{head} composition, heuristics lead to higher performance, while for low-dimensional multiplicative models heuristics fall behind the Max selection on the KS14 dataset.

% \todo[noline]{Significance tests}
When GS11 models are transferred to KS14, the average difference with the upper bound is 0.119 and 0.106 for Max and heuristics respectively. For the transfer to PhraseRel, the differences are 0.133 for Max and 0.188 for heuristics. Again, the heuristic-based selection outperforms the Max based. This supports H\ref{hyp:overfitting} that Max selection overfits, but the results are beyond the 10\% limit of H\ref{hyp:10percent}.

\subsection{Model transfer from PhraseRel}
\label{sec:from-phraserel}

\begin{figure}[b]
  \centering
    \includegraphics[width=\textwidth]{supplement/figures/PhraseRel-transfer}
    \caption{Transfer from Phraserel}
    \label{fig:phraserel-transfer}
\end{figure}

%%% Local Variables:
%%% mode: latex
%%% TeX-master: "../thesis"
%%% End:


Figure~\ref{fig:phraserel-transfer} shows that the performance of models based on PhraseRel is less stable, especially for selection by maximum performance.

% \todo[noline]{Significance tests}
Transfer to KS14 yields the average differences of 0.152 for Max and 0.136 for heuristics. Transfer to GS11 yields the average differences of 0.454 for Max and 0.509 for heuristics. Note that the transfer from PhraseRel to GS11 is the only case where Max selection, on average, is better than heuristics.

In general, over all compositional datasets, we see---in contrast to the lexical evaluation---that the Max-based selection might be prone to overfitting (H\ref{hyp:overfitting}). However, the result difference is far beyond the 10\% limit set by H\ref{hyp:10percent}, which might be due to the different nature of the tasks: similarity, disambiguation and relevance (compositional) versus similarity and relatedness (lexical).

\section{Universal parameter selection for compositional datasets}
\label{sec:robust-param-comp-selecion}

To find parameters that are good for all three tasks, we combine their scores. As with lexical tasks, we normalise the scores on each dataset and report the average performance. The combined score is calculated the following way:

{
\scriptsize
\begin{align}
\operatorname{score}_\mathit{compositional}&(\mathit{m}, \mathit{o}) =
&\frac{1}{3}\times%
\frac{\operatorname{score}_\mathit{KS14}(\mathit{m}, \mathit{o})}%
{\max_m\operatorname{score}_\mathit{KS14}(m, \mathit{o})}%
+%
\frac{1}{3}\times%
\frac{\operatorname{score}_\mathit{GS11}(\mathit{m}, \mathit{o})}%
{\max_m\operatorname{score}_\mathit{GS11}(m, \mathit{o})}%
+%
\frac{1}{3}\times%
\frac{\operatorname{score}_\mathit{PhraseRel}(\mathit{m, \mathit{o}})}%
{\max_m\operatorname{score}_\mathit{PhraseRel}(m, \mathit{o})}%
\end{align}
%\normalsize
}
where $m$ is a model and $o$ is an operator.

Figure~\ref{fig:compositional-results} shows the performance of the models based on the combined selection over the KS14, GS11 and PhraseRel datasets. 

The performance of selected models together with the selected parameters is shown in Table~\ref{tab:compositional-max-selection} (Max selection) and Table~\ref{tab:compositional-heuristics-selection} (selection based on heuristics).

\subsection{Max selection}
\label{sec:max-selection-compositional}

Models with many dimensions do not always perform better than their low-dimensional counterparts. Particularly, only \texttt{head} and multiplication benefit from the high number of dimensions. Addition and Kronecker are closer to the upper bound with dimensionality of a few thousand.

Regarding the hypotheses, there is support of H\ref{hyp:freq} (non-constant frequency should be used with high-dimensional spaces) for addition, multiplication and Kronecker, H\ref{hyp:cds} (context distribution smoothing is optimal for high-dimensional spaces) for multiplication and H\ref{hyp:neg} (low-dimensional spaces should be dense, high-dimensional---sparse) for Kronecker.

\subsection{Heuristics}
\label{sec:heuristics-compositional}

\begin{wraptable}[13]{O}{0.5\textwidth}
% \begin{table}
  \vspace{1em}
  \centering

  \begin{tabular}{lr}
\toprule
      parameter &  partial $R^2$ \\
\midrule
            neg &           0.40 \\
           freq &           0.29 \\
       operator &           0.21 \\
            cds &           0.15 \\
     similarity &           0.08 \\
          discr &           0.06 \\
 dimensionality &           0.05 \\
\bottomrule
\end{tabular}


  \caption{Compositional feature ablation}
  \label{tab:compositional-ablation}
\end{wraptable}


The linear model achieves  $R^2 = 0.769$. Table~\ref{tab:compositional-ablation} shows the partial $R^2$ values for the parameters. The most influential parameters are \texttt{neg}, \texttt{freq} and compositional operator.

\subsubsection{Neg}
\label{sec:neg-compositional}

For the baseline operator \texttt{head} and models with $D < 10\,000$, the \texttt{neg} should be set to 1, otherwise, it should be 1.4 (Figure~\ref{fig:compositional-neg}).

For addition, 1 is the best choice of \texttt{neg}, but the performance of $k$ values follows H\ref{hyp:neg}: the more dimensions a model has, the sparser it should be.

Multiplication benefits from denser spaces. If the dimensionality is less than 10\,000, then \texttt{neg} should be set to 0.5, otherwise 0.7 is a good choice, confirming H\ref{hyp:neg}.

Kronecker benefits from the \texttt{neg} of 0.7 if $D < 10\,000$ and from 1 for the more dimensional cases, also supporting H\ref{hyp:neg}. This is similar to multiplication, but Kronecker prefers less dense vectors.

\begin{figure}
% \begin{wrapfigure}{O}{0.5\textwidth}
  % \vspace{-30pt}
  \centering

  \begin{subfigure}[t]{\textwidth}
    \includegraphics[width=1.1\textwidth]{supplement/figures/compositional-interaction-neg}

  \caption{\texttt{neg}}
  \label{fig:compositional-neg}
  \end{subfigure}

  \begin{subfigure}[t]{\textwidth}
    \includegraphics[width=1.1\textwidth]{supplement/figures/compositional-interaction-freq}

  \caption{\texttt{freq}}
  \label{fig:compositional-freq}
  \end{subfigure}

  \caption{Compositional influence of \texttt{neg} and \texttt{freq}}
\end{figure}


\subsubsection{Freq}
\label{sec:freq-compositional}

The frequency value of choice of all operators is $\log n$ with an exception of multiplication, where the constant frequency is preferred (Figure~\ref{fig:compositional-freq}).

For low-dimensional vector spaces ($D \leq 5\,000$), 1 behaves well, giving support of H\ref{hyp:freq} that non-constant frequency is needed only with high-dimensional spaces.

\subsubsection{Context distribution smoothing}
\label{sec:cont-distr-smooth-compositional}

As Figure~\ref{fig:compositional-cds} shows, global context probability is the preferred choice of context probability in all cases, with an exception of Kronecker with $D > 3\,000$, where smoothed, local probabilities are better ($\alpha = 0.75$), supporting H\ref{hyp:cds} that context distribution smoothing should be used with high-dimensional spaces.

\subsubsection{Similarity}
\label{sec:similarity-compositional}

Correlation is the dominant choice of the similarity measure (Figure~\ref{fig:compositional-similarity}). However, cosine is preferred in the case of \texttt{head}, with $D > 5\,000$, and inner product is the only choice for composition with Kronecker with $D > 3\,000$.

There is no distinction between cosine and correlation for all compositional operators, which neither supports nor disputes H\ref{hyp:similarity} that expects correlation to outperform cosine with high-dimensional spaces.

\subsubsection{Discr}
\label{sec:discr-compositional}

\begin{figure}[b]
% \begin{wrapfigure}{O}{0.5\textwidth}
  % \vspace{-30pt}
  \centering

  \begin{subfigure}[t]{\textwidth}
    \includegraphics[width=1.1\textwidth]{supplement/figures/compositional-interaction-cds}

  \caption{\texttt{cds}}
  \label{fig:compositional-cds}
  \end{subfigure}

  \begin{subfigure}[t]{\textwidth}
    \includegraphics[width=1.1\textwidth]{supplement/figures/compositional-interaction-similarity}

  \caption{Similarity}
  \label{fig:compositional-similarity}
  \end{subfigure}

  \caption{Compositional influence of \texttt{cds} and similarity}
\end{figure}


\begin{figure}[b]
% \begin{wrapfigure}{O}{0.5\textwidth}
  % \vspace{-30pt}
  \centering

  \includegraphics[width=1.1\textwidth]{supplement/figures/compositional-interaction-discr}

  \caption{Compositional \texttt{discr}.}
  \label{fig:compositional-discr}
\end{figure}


SPMI is the choice of \texttt{discr} that leads to the best average performance in most cases (Figure~\ref{fig:compositional-discr}). However, the difference between SPMI and SCPMI is very small.

The exceptions are Multiplicative composition with $D \geq 10\,000$ and Kronecker with $D \leq 5\,000$ where SCPMI outperforms SPMI, as expected by H\ref{hyp:comp-pmi-cpmi} that PMI values compression is needed for composition.

\subsection{Comparison with the selection based on one dataset}
\label{sec:comp-with-single-comp}

Manual selection based on a combination of the compositional datasets is more stable with regards to the chosen parameter values than the selection based on the highest values, even though manual selection does not always achieve the performance of Max selection, see Figure~\ref{fig:compositional-results}.

The average difference with the upper bound is 0.040 and 0.041 for Max and heuristics, respectively, when applied to KS14. For GS11, the difference is 0.045 (Max) and 0.127 (Heuristics). For PhraseRel, the difference is 0.055 (Max) and 0.084 (Heuristics).

The numbers are much lower than the transfer of spaces selected on the basis of one dataset (Section~\ref{sec:select-model-transf-comp}). The average normalised difference is within the 10\% limit (H\ref{hyp:10percent}), with an exception of heuristics on GS11. This is evidence that there might be one universal model that fits various tasks (H\ref{hyp:universal}). The fact that the average normalised difference is smaller for Max-based selection is against H\ref{hyp:overfitting} that expects Max selection to overfit. A combination of datasets that covers several phenomena (in our case, similarity, disambiguation and relatedness) might be more effective than manual heuristic-based selection.

The model selection procedures improve from the combination of datasets. One needs to keep in mind that in this case, we test model performance on the same dataset as we do parameter selection.

\section{Conclusion}
\label{sec:conclusion-comp}

Phrasal experiments support most of the hypotheses stated in
Section~\ref{sec:hypotheses}.

We see the confirmation of hypotheses on optimal parameter dependence on dimensionality (H\ref{hyp:dimen}). In particular: H\ref{hyp:freq} (non-constant frequency is beneficial with high-dimensional spaces) and H\ref{hyp:neg} (high-dimensional spaces should be sparser). They are supported by all datasets and confirm H\ref{hyp:dimen}. It is worth noting that even though an optimal choice of context distribution smoothing does not depend on dimensionality on KS14 and GS11, in the combined case the dependence holds.

Models that are selected on the experiments on a single dataset are prone to overfitting. Neither manual selection of parameters prevents it and the average normalised difference is above 10\% on model transfer. This confirms H\ref{hyp:overfitting} that Max selection overfits and rejects H\ref{hyp:10percent} that the relative gap in performance of the best model and a manually selected model is within 10\%.

Model selection based on the combination of datasets performs much better on each dataset (contrary to a single-dataset selected models). Both selection methods are within 10\%, supporting H\ref{hyp:10percent}.

Max selection models outperform heuristic selection models, suggesting that there is no overfitting in this case and H\ref{hyp:overfitting} is not valid. In this case, the dataset combination covered three phenomena (similarity, disambiguation and relevance) and the precaution of overfitting by heuristics might be redundant.

This also suggests that there is a unique parameter choice that is universally applicable to compositional tasks (H\ref{hyp:universal}). The universal spaces are studied in Chapter~\ref{sec:universal-param-selection}.

\chapter[Universal models]{Universal models for both lexical and compositional tasks}
\label{sec:universal-param-selection}

\lettrine[lines=5,loversize=0.25]{P}{reviously}, we identified good models for specific datasets or task types: lexical and compositional. We managed to identify models that are good for either lexical tasks or compositional. This chapter investigates how well one model can perform on all tasks in contrast to the task-tailored models of previous chapters.

This is achieved by performing the evaluation on a combined score over the previous two lexical and three phrasal datasets. We not only combine the datasets together but also look for parameters that are good across all datasets with all compositional operators. Once optimal parameters are identified, they are tested on categorical compositional methods.

Even though the identified parameters have to compromise over different tasks (lexical and compositional) and compositional methods we achieve the new state-of-the-art results with Kronecker on KS14 and GS11. Moreover, the selected lexical representations improve over the results of the categorical compositional methods reported in the literature.

\section{Operator-dependent universal models}
\label{sec:model-selection}

First, we compute combined performance scores for the following operators: head, addition, multiplication and Kronecker. We normalise the scores of every dataset and weight lexical and compositional datasets equally. Within a dataset category, the datasets are weighted equally. Such a weighting scheme is still simple, it treats the task types equally and does not focus on a particular dataset.

The combined score for a model and an operator is computed as:

{\scriptsize
  \begin{align}
    \begin{split}
\operatorname{score}_\mathit{universal}(\mathit{m}, \mathit{o}) = & %
\frac{1}{2}\times\left(
\frac{1}{2}\times%
\frac{\operatorname{score}_\mathit{SimLex-999}(\mathit{m})}%
{\max_m\operatorname{score}_\mathit{SimLex-999}(m)}%
+%
\frac{1}{2}\times%
\frac{\operatorname{score}_\mathit{MEN}(\mathit{m})}%
{\max_m\operatorname{score}_\mathit{MEN}(m)}%
\right) +
\\
&\frac{1}{2}\times\left(
\frac{1}{3}\times%
\frac{\operatorname{score}_\mathit{KS14}(\mathit{m}, \mathit{o})}%
{\max_m\operatorname{score}_\mathit{KS14}(m, \mathit{o})}%
+%
\frac{1}{3}\times%
\frac{\operatorname{score}_\mathit{GS11}(\mathit{m}, \mathit{o})}%
{\max_m\operatorname{score}_\mathit{GS11}(m, \mathit{o})}%
+%
\frac{1}{3}\times%
\frac{\operatorname{score}_\mathit{PhraseRel}(\mathit{m, \mathit{o}})}%
{\max_m\operatorname{score}_\mathit{PhraseRel}(m, \mathit{o})}%
\right)
\end{split}
\end{align}
% \normalsize
}
where $m$ stands for a model and $o$ is a compositional operator.

\subsection{Max selection}
\label{sec:max-selection-universal}

Table~\ref{tab:universal-max-selection} shows the performance of the models evaluated with a combined score of the lexical and compositional datasets.
Parameter selection is much more stable than on all previous Max-based selections. The dominant \texttt{freq} choice is $\log n$, cosine is the measure of choice for multiplication and Kronecker (if available). Correlation is the best similarity measure for the additive composition. The optimal choice of a similarity measure does not depend on dimensionality---this observation does not support H\ref{hyp:similarity} that expects such dependence.

Interestingly, when shifting is applied, dense spaces perform better: 1 and 0.7 are the optimal \texttt{neg} values. Multiplication supports H\ref{hyp:neg} that high-dimensional spaces optimally perform with sparse models.

\begin{wraptable}[14]{O}{0.5\textwidth}
  % \vspace{-4ex}
  \centering

  \begin{tabular}{lr}
\toprule
      parameter &  partial $R^2$ \\
\midrule
           freq &      0.323 \\
            neg &      0.289 \\
     similarity &      0.218 \\
            cds &      0.102 \\
          discr &      0.091 \\
 dimensionality &      0.066 \\
       operator &      0.050 \\
\bottomrule
\end{tabular}


  \caption{Universal (operator dependant) feature ablation}
  \label{tab:universal-ablation}
\end{wraptable}


% \todo[noline]{This is one of the major findings.}
The preference of a compositional operator depends on a model. Addition with many dimensions gives the best results on lexical tasks: 0.384 on SimLex-999 and 0.761 on MEN. Kronecker, on the other side, gives the highest values on compositional datasets supporting H\ref{hyp:order} (the word order is important for compositional operators): 0.798 on KS14, 0.514 on GS11 and 0.964 on PhraseRel. Multiplication, however, is a good compromise between the two. It gives the highest ``combined'' score of 0.945.

\subsection{Heuristics}
\label{sec:heuristics-universal}

Performance of the models selected manually is shown in Table~\ref{tab:universal-heuristics-selection}. Again, there is a lot of consistency between parameters. The linear model achieves $R^2 = 0.828$. The most influencing parameters are \texttt{freq}, \texttt{neg} and a similarity measure. See Table~\ref{tab:universal-ablation} for more details.

Heuristics for addition choose models that score the highest on lexical tasks: 0.384 on SimLex-999 and 0.764 on MEN (Table~\ref{tab:universal-heuristics-selection}). Moreover, with more than 20\,000 dimensions, there is no difference between the selection procedures (Max or heuristics) of the additive and Kronecker-based models.

Kronecker is strong in compositional tasks scoring 0.795 on KS14, 0.516 on GS11 and 0.929 on PhraseRel, which is in line with H\ref{hyp:order} that word order is important (Table~\ref{tab:universal-heuristics-selection}).

Multiplication and Kronecker support H\ref{hyp:neg}: the optimal shifting value $k$ depends on the dimensionality. Addition and Kronecker are consistent with H\ref{hyp:cds}: low-dimensional spaces benefit from global context probabilities, while high-dimensional spaces benefit from smoothed context probabilities with $\alpha=0.75$.

Multiplication, again, is a compromise between the two: it gives the highest combined score of 0.941. The highest Kronecker's combined score is 0.913, while addition's highest score is only 0.843.

Regarding the hypotheses, we clearly see on Figure~\ref{fig:universal-freq} that there is little difference between 1 and $\log n$ frequencies for the low-dimensional spaces, but $\log n$ is the best choice for the high-dimensional spaces, which is consistent with H\ref{hyp:freq}.

Shifting performs also in accordance with H\ref{hyp:neg}: for low-dimensional spaces $k=0.7$ or even $k=0.5$ leads to the highest result, while for high-dimensional spaces $k=1$ or $k=1.4$ are optimal.

We also see that there is no difference between the cosine and correlation similarity measures giving a weak support of H\ref{hyp:similarity} that correlation is optimal for high-dimensional models.

Addition and Kronecker work best with global context probabilities on low-dimensional spaces, but benefit from local probabilities ($\alpha=0.75$) for high-dimensional spaces, supporting H\ref{hyp:cds}: context distribution smoothing is beneficial with high-dimensional spaces. Multiplication, however, does not follow H\ref{hyp:cds}, as $\alpha=0.75$ leads to weak performance for all dimensions.

There is no support of H\ref{hyp:comp-pmi-cpmi}: for all operators, there is little difference between SPMI and SCPMI, so compression of PMI values might not, in general, boost the performance of compositional models.

\subsection{Comparison of the selection methods}
\label{sec:comparison-universal}

On lexical tasks, there is little difference between the model selection methods, especially for spaces with more than 30\,000 dimensions, as Figures~\ref{fig:universal-results-simlex} and \ref{fig:universal-results-men} show.

The average relative differences for Max selection are 0.034 (SimLex-999), 0.010 (MEN), 0.033 (KS14), 0.109 (GS11) and 0.061 (PhraseRel). For manual heuristics, the differences are 0.047 (SimLex-999), 0.009 (MEN), 0.034 (KS14), 0.114 (GS11) and 0.065 (PhraseRel). The numbers between different selection methods are close, with the exceptions of SimLex-999 (where Max selection is 0.013 points lower), and GS11 (where Max is lower by 0.05 points).

The high relative difference on GS11 is due to a poor performance of addition and \texttt{head}. The average normalised difference for addition is 0.219 for Max selection and 0.224 for heuristics. For the baseline compositional operator \texttt{head}, the differences are 0.105 and 0.141 respectively. The differences for multiplication and Kronecker are less than 0.07, which is in accordance with the 10\% margin of H\ref{hyp:10percent}.

Contrary to H\ref{hyp:overfitting}, Max selection does not overfit, probably due to a broad selection of evaluation datasets.

% \todo[noline]{One of the major findings.}
When the models that are selected on lexical tasks are applied in a compositional setting, they perform worse than the models selected based on the universal score. This suggests that a model that is good for lexical tasks will not necessarily perform well on a compositional task, rejecting H\ref{hyp:not-lextocomp}.

In addition, the difference between the good lexical models and the upper bound increases as dimensionality increases. This is the case for multiplication, the most notable difference is observed on KS14 (Figure~\ref{fig:universal-results-ks14}).

It worth noting that on compositional tasks dimensionality does not contribute as much as on lexical tasks, with an exception of addition on the GS11 dataset, where performance decreases as dimensionality increases.

\section{An operator-independent universal model}
\label{sec:single}

In the previous section, we have seen that even though parameter selection varies between operators, there are parameter choices that are shared, for example, correlation is the best similarity measure for addition, multiplication and Kronecker (if $D \le 3\,000$). Given this and the fact that the difference between some of the choices is marginal, we try to look for a truly universal parameter combination. The aggregated score of a model is computed as:

{\scriptsize
\begin{align}
  \begin{split}
\operatorname{score}_\mathit{universal}(\mathit{m}) = &%
\frac{1}{2}\left(
\frac{1}{2}%
\frac{\operatorname{score}_\mathit{SimLex-999}(\mathit{m})}%
{\max_m\operatorname{score}_\mathit{SimLex-999}(m)}%
+%
\frac{1}{2}%
\frac{\operatorname{score}_\mathit{MEN}(\mathit{m})}%
{\max_m\operatorname{score}_\mathit{MEN}(m)}%
\right)+
\\
&\frac{1}{2}\Bigg(
\frac{1}{6}%
\frac{\operatorname{score}_\mathit{KS14}(\mathit{m}, \mathit{add})}%
{\max_m\operatorname{score}_\mathit{KS14}(m, \mathit{add})}%
+%
\frac{1}{6}%
\frac{\operatorname{score}_\mathit{GS11}(\mathit{m}, \mathit{add})}%
{\max_m\operatorname{score}_\mathit{GS11}(m, \mathit{add})}%
+%
\frac{1}{6}%
\frac{\operatorname{score}_\mathit{PhraseRel}(\mathit{m, \mathit{add}})}%
{\max_m\operatorname{score}_\mathit{PhraseRel}(m, \mathit{add})}
+
\\
&\phantom{\frac{1}{2}\Bigg(}
\frac{1}{6}%
\frac{\operatorname{score}_\mathit{KS14}(\mathit{m}, \mathit{mult})}%
{\max_m\operatorname{score}_\mathit{KS14}(m, \mathit{mult})}%
+%
\frac{1}{6}%
\frac{\operatorname{score}_\mathit{GS11}(\mathit{m}, \mathit{mult})}%
{\max_m\operatorname{score}_\mathit{GS11}(m, \mathit{mult})}%
+%
\frac{1}{6}%
\frac{\operatorname{score}_\mathit{PhraseRel}(\mathit{m, \mathit{mult}})}%
{\max_m\operatorname{score}_\mathit{PhraseRel}(m, \mathit{mult})}
\Bigg)
\end{split}
\end{align}
\normalsize
}
where $\mathit{add}$ stands for addition and $\mathit{mult}$ stands for multiplication. We do not test Kronecker, because it is not tested on all dimensions with the cosine and correlation similarity measures. We also exclude the baseline operator \texttt{head}.

\subsection{Max selection}
\label{sec:max-selection-single}

Table~\ref{tab:single-max-selection} shows the combined scores for all datasets abstracting over a compositional operator.

The parameter selection shows a clear pattern. Low-dimensional spaces perform best with 1 as the frequency choice, while high-dimensional models perform better with $\log n$, confirming to H\ref{hyp:freq} that non-linear frequency should be used with high-dimensional models. Cosine is a better-suited similarity measure for models with few dimensions and correlation is suited to those with many, which is in line with H\ref{hyp:similarity}. Finally, global context probabilities are the best in a low-dimensional case, while local context probabilities perform best with many dimensions, supporting H\ref{hyp:cds}.

\subsection{Heuristics}
\label{sec:heuristics-single}

\begin{wraptable}[12]{O}{0.5\textwidth}
%\begin{table}[b]
  %\vspace{-2em}
  \centering

  \begin{tabular}{lr}
\toprule
      parameter &  partial $R^2$ \\
\midrule
           freq &   0.434 \\
     similarity &   0.269 \\
            neg &   0.196 \\
          discr &   0.092 \\
            cds &   0.090 \\
 dimensionality &   0.034 \\
\bottomrule
\end{tabular}


  \caption{Universal (operator independent) feature ablation}
  \label{tab:single-ablation}
  % \end{wraptable}
\end{wraptable}


The linear model gives $R^2 = 0.898$. The most influential parameters are \texttt{freq}, similarity measure and \texttt{neg}, refer to Table~\ref{tab:single-ablation}.

Heuristics, in general, repeat the parameter choices of Max selection, but the switch between parameter values happens at a higher number of dimensions (at 20\,000, not at 5\,000). Refer to Table~\ref{tab:single-heuristics-selection} for the results and Figure~\ref{fig:single-params} for the parameter behaviour.

The average normalised differences with the upper bound for Max selection are 0.049 (SimLex-999), 0.032 (MEN), 0.063 (KS14), 0.106 (GS11) and 0.116 (PhraseRel). The differences for heuristics are in general higher: 0.062 (SimLex-999), 0.045 (MEN), 0.076 (KS14), 0.090 (GS11) and 0.139 (PhraseRel).

Max selection is above the 10\% margin of H\ref{hyp:10percent} on GS11 and PraseRel, while heuristics are above the margin only on PhraseRel.

\section{Experiments with categorical compositional operators}
\label{sec:frob-comp-oper}

Sections~\ref{sec:model-selection} and \ref{sec:single} identified models that perform well on a range of tasks. The majority of them are within the 10\% margin set by H\ref{hyp:10percent}. We apply selected models with tensor-based operators on phrasal datasets (KS14, GS11 and PhraseRel).

\begin{table}
  \centering
  \begin{tabular}{lrrr}
\toprule
Operator &   KS14 &   GS11 &  PhraseRel \\
\midrule
Add              &  0.785 &  0.338 &      0.893 \\
Mult             &  0.771 &  0.507 &      \textbf{1.000} \\
Kron             &  \textbf{0.798} &  \textbf{0.516} &      0.964 \\
Relational       &  0.768 &  0.393 &      0.893 \\
Copy-object      &  0.628 &  0.278 &      0.821 \\
Copy-subject     &  0.738 &  0.402 &      0.821 \\
Frobenius-add   &  0.761 &  0.374 &      0.821 \\
Frobenius-mult  &  0.747 &  0.299 &      0.857 \\
Frobenius-outer &  0.765 &  0.385 &      0.893 \\
\bottomrule
\end{tabular}


  \caption{The best scores on compositional tasks based on the universal selections}
  \label{tab:frobenius-results}
\end{table}


Table~\ref{tab:frobenius-results} shows the best results we obtained for each operator, Tables~\ref{tab:frobenius-ks14-results}, \ref{tab:frobenius-gs11-results} and \ref{tab:frobenius-phraserel-results} show all the results together with the model parameters and Figure~\ref{fig:frobenius-results} depicts the data.

In general, the best results are achieved with 3\,000-dimensional models (with an exception on GS11 where 1\,000-dimensional models perform better in 4 out of 6 cases, and copy-subject on KS14). Also, performance increases as dimensionality increases.

Max selection based on Kronecker leads to the highest results. The exceptions are Frobenius multiplication and copy-subject on KS14, where the model that is best with addition also leads to the highest results among tensor-based composition. On PhraseRel, copy-subject performs best with operator-independently selected space.

Relational is the fourth best compositional operator after addition, multiplication or Kronecker on KS14 (0.768) and PhraseRel (0.893). Copy-subject is the best on GS11 (0.402). Frobenius-outer gives the highest result on PhraseRel together with relational.

On GS11 and PhraseRel, newly tested operators outperform addition, whose scores are 0.338 and 0.893, respectively.

While there is a difference between selection methods, there are no clear outliers and the models show similar behaviour.

\section{Putting results into perspective}
\label{sec:comp-with-other}

This section discusses the results of the experiments in the context of our preliminary studies and the work of others.

In an earlier study \cite{milajevs-EtAl:2014:EMNLP2014}, we compared a PPMI-weighted space, an LMI-weighted and SVD-reduced space and a space based on the original word2vec vectors obtained from the Google News corpus \cite{mikolov2013distributed}. The same compositional operators were evaluated as in this thesis. The count-based models selected in the earlier study were the ones that were considered to be efficient in the compositional tasks and were used in the studies that introduced the evaluation datasets: KS14 \cite{kartsadrqpl2014} and GS11 \cite{Grefenstette:2011:ESC:2145432.2145580}. Thus, \newcite{milajevs-EtAl:2014:EMNLP2014} can be seen as a replication of the experiments in previous papers. The experiments showed that on small-scale tasks (KS14 and GS11) count-based models are competitive with neural word vectors, however, word2vec vectors are superior in dialog act tagging and paraphrase detection.

In that study, additive composition with an SVD-reduced space gave the best result of 0.732 on KS14. The best tensor-based result was 0.655, achieved with word2vec and copy-object. All our models (Table~\ref{tab:frobenius-results}), with the exception of copy-object, improve over our previous best scores. Despite being lower, our copy-object (0.628) is close to the word2vec score reported in \newcite{milajevs-EtAl:2014:EMNLP2014}.

On the GS11 dataset, systematic parameter selection leads to spaces that improve over the corresponding operators in the earlier study on all but two of the compositional methods (the exceptions being copy-object and Frobenius mult). In addition to that, multiplication and Kronecker improve over the overall best-reported score of 0.456 in \newcite{milajevs-EtAl:2014:EMNLP2014}. Kronecker yields the highest score of 0.516.

\newcite{kim2015neural} adopt the evaluation procedure of \newcite{milajevs-EtAl:2014:EMNLP2014} to test an extended word2vec model that is tuned for multiplicative interaction of the vectors, not additive, as the original word2vec. They improve on most of the composition operators on the KS14 and GS11 datasets.

They achieve the best result of 0.770 with addition on KS14. Three of our models (addition, multiplication and Kronecker) outperform that score. Also, our results are better than the results reported  by comparing results by operator (our result are shown in brackets for comparison): multiplication 0.440 (0.771), Kronecker 0.623 (0.798), relational 0.665 (0.768), copy-subject 0.454 (0.738), copy-object 0.607 (0.628), Frobenius addition 0.610 (0.761), Frobenius multiplication 0.608 (0.747), Frobenius outer 0.664 (0.765).

On G11, their best score is 0.387, which is lower than the results that we get with multiplication, Kronecker, relational and copy-subject. However, we get lower results with copy-object and Frobenius outer.

\newcite{hashimoto-tsuruoka:2015:CVSC} learn the matrices of transitive verbs using implicit tensor factorisation. The verb matrices are learned in two ways: one only takes into account the verb arguments (the subject and the object, referred as SVO in the paper), another, in addition to that, employs the adjuncts that complement the meaning of the verb phrases (SVOPN). They use copy-subject as the compositional operator. The main baseline is their previous method described in \newcite{hashimoto-EtAl:2014:EMNLP2014}.

In comparison to their SVO results, our are higher: they get 0.480 on GS11 and 0.481 on KS14. These results are lower than our Multiplication and Kronecker on GS11 and all operators on KS14.

The SVOPN model with the score of 0.614 outperforms our best (0.516) on GS11. While they get a higher score on KS14 of 0.744 (though it is obtained with the \newcite{hashimoto-EtAl:2014:EMNLP2014} baseline) it is still lower than all of our results, except copy-subject and copy-object. Interestingly, our result of 0.738 with copy-subject is close to their score but is still lower.

\newcite{hashimoto-tsuruoka:2016:P16-1} jointly learn compositional and non-compositional phrase embeddings by using a compositionality scoring function. They improve on their previous work and get the score of 0.680 on GS11 versus our 0.516.

\newcite{fried-polajnar-clark:2015:ACL-IJCNLP} use low-rank tensors to approximate third-order tensors of verbs. They achieve the scores of 0.47 on GS11 and 0.68 on KS14 with categorical composition and 0.71 on KS14 with addition. While our best result is higher on GS11, categorical operators score lower, the best is copy-subject (0.402). On KS14, our experiments produce higher results (with an exception of copy-object) than their best (additive) model. It is worth noting the study of \newcite{polajnar-rimell-clark:2015:LSDSem} uses discourse features to build vectors, but the experiment results reported there are not compatible with ours because we averaged the human-provided scores before computing correlation, while they treat each human score individually without averaging.

Overall, we improved over the scores by identifying a better set of model parameters rather than developing a more sophisticated model.

\section{Conclusion}
\label{sec:conclusion-universal}

This chapter identified a few spaces that work well on a broad range of lexical and compositional tasks.

Despite the expectation in H\ref{hyp:overfitting}, we see that Max selection does not overfit if the models are evaluated on diverse tasks. In fact, heuristics become too conservative in this case, however, we still suggest manual analysis when a small number of datasets is used.

Our universal models perform within the 10\% margin of H\ref{hyp:10percent} in the majority of the experiments. Moreover, the operator-independent universal space is competitive with spaces that were selected with an operator in mind, supporting the idea that there is a universal vector space for all kind of tasks and H\ref{hyp:universal}.

The selections show that an optimal parameter choice depends on dimensionality (H\ref{hyp:dimen}). As we have seen in Section~\ref{sec:comparison-universal}, a good lexical model might fail in a compositional setting (H\ref{hyp:not-lextocomp}). We have also seen in Section~\ref{sec:max-selection-universal} that good lexical models favour the additive composition, while Kronecker is more optimal for composition and multiplication is a compromise between the two. This, and the fact that a similarity has been found between lexical and compositional tasks \cite[\textcolor{citecolor}{Section~4}]{kiela-clark:2014:CVSC}, might be an explanation of why addition is considered to be the best compositional operator. In our experiments, multiplication and Kronecker consistently outperform addition.

While parameter selection depends on dimensionality, the performance on compositional task depends on it to a much lower extent.

Our selection methodology has produced significantly better results in the KS14 dataset over widely used count-based vectors \cite{milajevs-EtAl:2014:EMNLP2014}, neural vectors in a compositional setting \cite{milajevs-EtAl:2014:EMNLP2014,kim2015neural} and learned verb tensors \cite{fried-polajnar-clark:2015:ACL-IJCNLP,hashimoto-tsuruoka:2016:P16-1,hashimoto-tsuruoka:2015:CVSC}.

While our results on GS11 are close to the current state-of-the-art results \cite{hashimoto-tsuruoka:2016:P16-1}, there is room for improvement, especially for tensor-based compositional operators. The difference in the performance might be explained as the limit of the count-based methods or the unexplored, and therefore untuned, parameters of the verb matrix. For example, we consider all different subject-object occurrences despite their frequency in the corpus. Using only the subject-object pairs that appeared at least 100 times might improve the results.

The gap between the multiplicative and Kronecker composition in our work indicates that the categorical methods can be improved. We see that the word order is important in the task, otherwise, Kronecker would not outperform addition and multiplication. Because categorical methods take word order into account, there is a potential for them to improve. However, it is not clear whether the verb matrices are of good quality. The verb matrices obtained by different ways need to be tested on a lexical similarity task, for example \newcite{2016arXiv160800869G}. Also, the similarity judgments of verbs in SimLex-999 and MEN can be used.



%%% Local Variables:
%%% mode: latex
%%% TeX-master: "thesis"
%%% TeX-engine: xetex
%%% End:


\chapter{Conclusion}

This work is a systematic study of vector space models of meaning, based on the distributional hypothesis \cite{harris1954distributional} and Frege's principle of compositionality \cite{Janssen2001,DBLP:journals/corr/abs-1003-4394}. The goal of this work is to provide the performance numbers of count-based distributional models that are robust to overfitting and are representative of this kind of method in comparison to other meaning models, for example, predictive models \cite{mikolov2013efficient}. The other goal of the current study is to compare the parameters within the count-based models and identify their combinations that lead to high performance of the corresponding models.

The experiments in the study are performed on two lexical tasks---SimLex-999 \cite{hill2014simlex} and MEN \cite{Bruni:2014:MDS:2655713.2655714}---and three phrasal tasks---KS14 \cite{kartsadrqpl2014}, GS11 \cite{Grefenstette:2011:ETV:2140490.2140497} and PhraseRel (Section~\ref{sec:phraserel}). In addition to individual dataset evaluation, the models' performance scores are combined to identify models that perform well on the collections of tasks, namely lexical (Section~\ref{sec:universal-lexical-param-selection}), compositional (Section~\ref{sec:robust-param-comp-selecion}) and universal (Chapter~\ref{sec:universal-param-selection}).

The vector component values are based on the PMI quantification of the co-occurrence counts. The PMI score itself is modified by weighting, shifting, compressing and others---see Section~\ref{sec:quantification} for more details. We identify an optimal parameter choice based on dimensionality of the underlying vector space. In compositional tasks, we experiment with point-wise operators (addition and multiplication) and with categorical operators (Section~\ref{sec:frob-comp-oper}, \newcite{DBLP:journals/corr/abs-1003-4394}).

The experiments (Chapters~\ref{sec:lexical}, \ref{sec:sentential} and \ref{sec:universal-param-selection}) show that the optimal parameter choice depends on dimensionality. While there are more optimal choices for particular datasets, we suggest using SCPMI with $\log n$ frequency, global context probabilities, shifted $k=0.7$ values and correlation as the similarity measure with at least 20000 dimensional space. For categorical tasks, we suggest 3000 dimensions and cosine as the similarity measure, keeping other parameters the same. This paper shows that indeed there might be a single model that is good in a variety of tasks \cite{doi:10.1080/02643294.2016.1176907}.
% * <sdyck@ualberta.ca> 2016-11-05T01:26:50.867Z:
%
% > niversal-param-selection}) show that the optimal parameter choice depends on dimensionality. While there are more optimal choices for particular datasets, we suggest using SCPMI with $\log n$ frequency, global context probabilities, shifted $k=0.7$ values and correlation as the similarity measure with at least 20000 dimensional space. For categorical tasks, we suggest 3000 dimensions and cosine as the similarity measure, keeping other parameters the same. This paper shows that indeed there might be a single model that is good in a variety of tasks \cite{doi:10.1080/02643294.2016.1176907}.
% > Two model selection procedures were tested. One selects the parameters that lead to the highest performance, while the other performs selection based on the average performance of the parameter values. We see that if a single dataset is used for model selection, that the best model is overfitted and suggest using a more elaborated method. However, when the selection is based on a combination of the datasets, then the max-based selection picks models that do not overfit.
%
% Watch present versus past tense.. Are you wanting to talk about your experiments as they already happened or happening? (past tense is probably better). 
%
% ^.

Two model selection procedures were tested. One selects the parameters that lead to the highest performance, while the other performs selection based on the average performance of the parameter values. We see that if a single dataset is used for model selection, that the best model is overfit and suggest using a more elaborated method. However, when the selection is based on a combination of the datasets, then the Max-based selection picks models that do not overfit.

There are several directions for future work. First of all, we explored unreduced spaces (for example, we did not apply SVD \newcite{BullinariaLevy2012}). Apart from a particular dimensionality reduction method being superior, it might interact with other parameters \cite{lapesa2014large} and change the optimal values, in contrast to the reasoning of \newcite{kiela-clark:2014:CVSC} that ``\ldots dimensionality reduction relies on some original non-reduced model, and directly depends on its quality.''

Another direction is the experimentation on a a larger number of datasets. While more datasets are being proposed, for example \newcite{2016arXiv160800869G}, and the current datasets are being criticised \cite{RepEval:2016}, it is important to have datasets that share the same goal (for example, provide similarity judgements), but are constructed by different groups and employ different methods during the dataset construction.

While categorical compositional methods are built on solid theoretical grounds \cite{DBLP:journals/corr/abs-1003-4394} and have been shown previously \cite{Grefenstette:2011:ETV:2140490.2140497,kartsadrqpl2014,fried-polajnar-clark:2015:ACL-IJCNLP,kim2015neural,hashimoto-tsuruoka:2016:P16-1} and in this work to be competitive with other methods, more parameter exploration work has to be done, especially for the way how the verb (or other relational) tensors are built. On the theoretical side, the categorical methods need to be of lower computation complexity, as the current way of building verb matrices is not feasible for vectors over 3000 components.

% Come back to the research questions.

\cleardoublepage
\addcontentsline{toc}{chapter}{Bibliography}
\phantomsection
\bibliographystyle{plainnat}
\bibliography{references,dmilajevs_publications}

\cleardoublepage
\appendix

\chapter{Experimental data}

\begin{table}
  \centering

  \begin{tabular}{lrrlllll}
\toprule
operator &  dimensionality &   KS14 &  freq &  discr &     cds &  neg &     similarity \\
\midrule
    head &            1\,000 &  0.700 &  logn &  scpmi &       1 &  0.7 &    correlation \\
    head &            2\,000 &  0.722 &  logn &  scpmi &       1 &  0.7 &            cos \\
    head &            3\,000 &  0.712 &  logn &  scpmi &  global &  1.4 &            cos \\
    head &            5\,000 &  0.694 &  logn &  scpmi &    0.75 &  0.7 &  inner\_product \\
    head &           10\,000 &  0.713 &     1 &  scpmi &    0.75 &    1 &            cos \\
    head &           20\,000 &  0.719 &  logn &  scpmi &    0.75 &    1 &            cos \\
    head &           30\,000 &  0.720 &  logn &  scpmi &    0.75 &    1 &            cos \\
    head &           40\,000 &  0.724 &  logn &   spmi &    0.75 &  1.4 &            cos \\
    head &           50\,000 &  0.721 &  logn &   spmi &    0.75 &  1.4 &    correlation \\
     add &            1\,000 &  0.781 &     1 &   spmi &  global &  1.4 &    correlation \\
     add &            2\,000 &  0.798 &     1 &   spmi &  global &    1 &    correlation \\
     add &            3\,000 &  0.793 &     1 &   spmi &  global &    1 &    correlation \\
     add &            5\,000 &  0.791 &     1 &  scpmi &    0.75 &  0.7 &    correlation \\
     add &           10\,000 &  0.795 &     1 &   spmi &    0.75 &  0.5 &    correlation \\
     add &           20\,000 &  0.798 &     1 &  scpmi &    0.75 &  0.7 &    correlation \\
     add &           30\,000 &  0.795 &     1 &  scpmi &    0.75 &  0.7 &    correlation \\
     add &           40\,000 &  0.792 &     1 &  scpmi &    0.75 &  0.7 &    correlation \\
     add &           50\,000 &  0.790 &     1 &  scpmi &    0.75 &  0.7 &    correlation \\
    mult &            1\,000 &  0.771 &     1 &  scpmi &       1 &  0.2 &    correlation \\
    mult &            2\,000 &  0.784 &     1 &  scpmi &       1 &  0.2 &    correlation \\
    mult &            3\,000 &  0.782 &     1 &  scpmi &       1 &  0.2 &    correlation \\
    mult &            5\,000 &  0.781 &     1 &  scpmi &       1 &  0.2 &    correlation \\
    mult &           10\,000 &  0.779 &     1 &  scpmi &       1 &  0.2 &    correlation \\
    mult &           20\,000 &  0.775 &     1 &  scpmi &       1 &  0.2 &    correlation \\
    mult &           30\,000 &  0.772 &     1 &  scpmi &       1 &  0.2 &    correlation \\
    mult &           40\,000 &  0.770 &     1 &  scpmi &       1 &  0.2 &    correlation \\
    mult &           50\,000 &  0.773 &     1 &   cpmi &       1 &  N/A &    correlation \\
    kron &            1\,000 &  0.799 &     1 &  scpmi &       1 &  0.2 &    correlation \\
    kron &            2\,000 &  0.810 &     1 &  scpmi &       1 &  0.2 &    correlation \\
    kron &            3\,000 &  0.808 &     1 &  scpmi &       1 &  0.2 &    correlation \\
    kron &            5\,000 &  0.793 &     1 &  scpmi &    0.75 &  0.7 &  inner\_product \\
    kron &           10\,000 &  0.795 &     1 &  scpmi &    0.75 &  0.7 &  inner\_product \\
    kron &           20\,000 &  0.792 &     1 &  scpmi &    0.75 &  0.7 &  inner\_product \\
    kron &           30\,000 &  0.793 &     1 &  scpmi &    0.75 &  0.7 &  inner\_product \\
    kron &           40\,000 &  0.791 &     1 &  scpmi &    0.75 &  0.7 &  inner\_product \\
    kron &           50\,000 &  0.793 &     1 &   spmi &    0.75 &  0.7 &  inner\_product \\
\bottomrule
\end{tabular}


  \caption{KS14 Max selection.}
  \label{tab:ks14-max-selection}
\end{table}

\begin{table}
  \centering

  \begin{tabular}{lrrlllll}
\toprule
operator &  dimensionality &  KS14 &  freq &  discr &     cds &  neg &     similarity \\
\midrule
    head &            1\,000 &  0.69 &  logn &  scpmi &  global &    1 &            cos \\
    head &            2\,000 &  \textbf{0.70} &  logn &  scpmi &  global &    1 &            cos \\
    head &            3\,000 &  0.69 &  logn &  scpmi &  global &    1 &            cos \\
    head &            5\,000 &  0.69 &  logn &  scpmi &  global &  1.4 &            cos \\
    head &           1\,0000 &  0.69 &  logn &  scpmi &  global &  1.4 &            cos \\
    head &           2\,0000 &  \textbf{0.70} &  logn &   spmi &    0.75 &  1.4 &    correlation \\
    head &           3\,0000 &  0.69 &  logn &   spmi &    0.75 &    2 &    correlation \\
    head &           4\,0000 &  0.69 &  logn &   spmi &    0.75 &    2 &    correlation \\
    head &           5\,0000 &  0.69 &  logn &   spmi &    0.75 &    2 &    correlation \\ \addlinespace
     add &            1\,000 &  0.78 &     1 &   spmi &  global &  1.4 &    correlation \\
     add &            2\,000 &  \textbf{0.79} &     1 &   spmi &  global &  1.4 &    correlation \\
     add &            3\,000 &  \textbf{0.79} &     1 &   spmi &  global &  1.4 &    correlation \\
     add &            5\,000 &  \textbf{0.79} &  logn &   spmi &  global &  1.4 &    correlation \\
     add &           1\,0000 &  \textbf{0.79} &  logn &   spmi &  global &  1.4 &    correlation \\
     add &           2\,0000 &  0.78 &  logn &   spmi &  global &    2 &    correlation \\
     add &           3\,0000 &  0.78 &  logn &   spmi &  global &    2 &    correlation \\
     add &           4\,0000 &  0.78 &  logn &   spmi &  global &    2 &    correlation \\
     add &           5\,0000 &  0.77 &  logn &   spmi &  global &    2 &    correlation \\ \addlinespace
    mult &            1\,000 &  0.77 &     1 &  scpmi &  global &  0.5 &    correlation \\
    mult &            2\,000 &  \textbf{0.78} &     1 &  scpmi &  global &  0.5 &    correlation \\
    mult &            3\,000 &  0.77 &     1 &  scpmi &  global &  0.5 &    correlation \\
    mult &            5\,000 &  0.77 &     1 &  scpmi &  global &  0.5 &    correlation \\
    mult &           1\,0000 &  0.77 &     1 &  scpmi &  global &  0.7 &    correlation \\
    mult &           2\,0000 &  0.77 &     1 &  scpmi &  global &  0.7 &    correlation \\
    mult &           3\,0000 &  0.75 &     1 &  scpmi &  global &    1 &    correlation \\
    mult &           4\,0000 &  0.74 &     1 &  scpmi &  global &    1 &    correlation \\
    mult &           5\,0000 &  0.74 &     1 &  scpmi &  global &    1 &    correlation \\ \addlinespace
    kron &            1\,000 &  0.78 &     1 &   spmi &  global &  0.5 &    correlation \\
    kron &            2\,000 &  \textbf{0.80} &     1 &   spmi &  global &  0.5 &    correlation \\
    kron &            3\,000 &  \textbf{0.80} &     1 &   spmi &  global &  0.7 &    correlation \\
    kron &            5\,000 &  0.77 &     1 &   spmi &  global &  0.7 &  inner\_product \\
    kron &           1\,0000 &  0.75 &  logn &   spmi &  global &  0.7 &  inner\_product \\
    kron &           2\,0000 &  0.77 &  logn &  scpmi &  global &    1 &  inner\_product \\
    kron &           3\,0000 &  0.77 &  logn &  scpmi &  global &    1 &  inner\_product \\
    kron &           4\,0000 &  0.77 &  logn &  scpmi &  global &    1 &  inner\_product \\
    kron &           5\,0000 &  0.77 &  logn &  scpmi &  global &    1 &  inner\_product \\
\bottomrule
\end{tabular}


  \caption{KS14 selection based on heuristics.}
  \label{tab:ks14-heuristics-selection}
\end{table}


\begin{table}
  \centering

  \begin{tabular}{lrrlllll}
\toprule
operator &  dimensionality &   GS11 &  freq &  discr &     cds &  neg &     similarity \\
\midrule
    head &            1\,000 &  0.379 &  logn &  scpmi &    0.75 &  0.2 &    correlation \\
    head &            2\,000 &  0.365 &     1 &    pmi &  global &  N/A &  inner\_product \\
    head &            3\,000 &  0.406 &     1 &    pmi &  global &  N/A &  inner\_product \\
    head &            5\,000 &  0.401 &     1 &    pmi &  global &  N/A &  inner\_product \\
    head &           10\,000 &  \textbf{0.432} &     1 &    pmi &  global &  N/A &  inner\_product \\
    head &           20\,000 &  0.369 &     1 &  scpmi &  global &    1 &    correlation \\
    head &           30\,000 &  0.379 &  logn &   spmi &  global &  0.7 &    correlation \\
    head &           40\,000 &  0.375 &  logn &  scpmi &       1 &  0.7 &    correlation \\
    head &           50\,000 &  0.377 &  logn &   spmi &  global &  0.7 &    correlation \\ \addlinespace
     add &            1\,000 &  \textbf{0.338} &     1 &  scpmi &  global &  0.7 &    correlation \\
     add &            2\,000 &  0.313 &     1 &   spmi &  global &  0.2 &    correlation \\
     add &            3\,000 &  0.312 &     1 &    pmi &    0.75 &  N/A &    correlation \\
     add &            5\,000 &  0.302 &     1 &    pmi &       1 &  N/A &            cos \\
     add &           10\,000 &  0.318 &     1 &    pmi &  global &  N/A &            cos \\
     add &           20\,000 &  0.284 &  logn &  scpmi &    0.75 &  0.2 &    correlation \\
     add &           30\,000 &  0.266 &  logn &  scpmi &    0.75 &  0.2 &    correlation \\
     add &           40\,000 &  0.256 &     1 &    pmi &  global &  N/A &    correlation \\
     add &           50\,000 &  0.253 &     1 &    pmi &  global &  N/A &    correlation \\ \addlinespace
    mult &            1\,000 &  0.457 &  logn &    pmi &  global &  N/A &  inner\_product \\
    mult &            2\,000 &  0.470 &  logn &   spmi &  global &  0.5 &            cos \\
    mult &            3\,000 &  0.485 &     1 &  scpmi &  global &  0.7 &            cos \\
    mult &            5\,000 &  0.493 &     1 &  scpmi &  global &  0.7 &            cos \\
    mult &           10\,000 &  0.498 &  logn &   spmi &  global &  0.5 &            cos \\
    mult &           20\,000 &  \textbf{0.532} &     1 &  scpmi &  global &  0.7 &    correlation \\
    mult &           30\,000 &  0.511 &     1 &  scpmi &  global &    1 &    correlation \\
    mult &           40\,000 &  0.521 &     1 &  scpmi &  global &    1 &    correlation \\
    mult &           50\,000 &  0.502 &  logn &   spmi &  global &  0.5 &            cos \\ \addlinespace
    kron &            1\,000 &  0.433 &     1 &  scpmi &  global &  0.2 &    correlation \\
    kron &            2\,000 &  0.492 &  logn &  scpmi &    0.75 &  0.7 &  inner\_product \\
    kron &            3\,000 &  0.501 &  logn &   spmi &    0.75 &  0.5 &  inner\_product \\
    kron &            5\,000 &  0.509 &  logn &   spmi &    0.75 &  0.7 &  inner\_product \\
    kron &           10\,000 &  0.514 &  logn &   spmi &    0.75 &  0.7 &  inner\_product \\
    kron &           20\,000 &  0.509 &  logn &   spmi &    0.75 &  0.7 &  inner\_product \\
    kron &           30\,000 &  0.512 &  logn &   spmi &    0.75 &    1 &  inner\_product \\
    kron &           40\,000 &  0.513 &  logn &   spmi &    0.75 &    1 &  inner\_product \\
    kron &           50\,000 &  \textbf{0.516} &  logn &   spmi &    0.75 &    1 &  inner\_product \\
\bottomrule
\end{tabular}


  \caption{GS14 Max selection.}
  \label{tab:gs11-max-selection}
\end{table}

\begin{table}
  \centering

  \begin{tabular}{lrrlllll}
\toprule
operator &  dimensionality &   GS11 &  freq &  discr &     cds &  neg &     similarity \\
\midrule
    head &            1000 &  0.328 &  logn &   spmi &  global &  0.5 &            cos \\
    head &            2000 &  0.326 &  logn &   spmi &  global &  0.5 &            cos \\
    head &            3000 &  0.354 &  logn &   spmi &  global &  0.7 &            cos \\
    head &            5000 &  0.359 &  logn &   spmi &  global &  0.7 &            cos \\
    head &           10000 &  0.344 &  logn &   spmi &       1 &  0.7 &            cos \\
    head &           20000 &  0.358 &  logn &   spmi &       1 &  0.7 &            cos \\
    head &           30000 &  0.363 &  logn &   spmi &       1 &  0.7 &            cos \\
    head &           40000 &  0.366 &  logn &   spmi &       1 &  0.7 &            cos \\
    head &           50000 &  0.365 &  logn &   spmi &       1 &  0.7 &            cos \\
     add &            1000 &  0.288 &  logn &    pmi &       1 &  N/A &    correlation \\
     add &            2000 &  0.282 &  logn &    pmi &       1 &  N/A &    correlation \\
     add &            3000 &  0.279 &  logn &    pmi &       1 &  N/A &    correlation \\
     add &            5000 &  0.277 &  logn &    pmi &       1 &  N/A &    correlation \\
     add &           10000 &  0.265 &  logn &    pmi &       1 &  N/A &    correlation \\
     add &           20000 &  0.284 &  logn &  scpmi &    0.75 &  0.2 &    correlation \\
     add &           30000 &  0.266 &  logn &  scpmi &    0.75 &  0.2 &    correlation \\
     add &           40000 &  0.250 &  logn &  scpmi &    0.75 &  0.2 &    correlation \\
     add &           50000 &  0.237 &  logn &  scpmi &    0.75 &  0.2 &    correlation \\
    mult &            1000 &  0.438 &  logn &    pmi &  global &  N/A &            cos \\
    mult &            2000 &  0.413 &  logn &    pmi &  global &  N/A &            cos \\
    mult &            3000 &  0.437 &  logn &    pmi &  global &  N/A &            cos \\
    mult &            5000 &  0.488 &  logn &  scpmi &  global &  0.7 &            cos \\
    mult &           10000 &  0.496 &  logn &  scpmi &  global &  0.7 &            cos \\
    mult &           20000 &  0.505 &  logn &  scpmi &  global &  0.7 &            cos \\
    mult &           30000 &  0.500 &  logn &  scpmi &  global &  0.7 &            cos \\
    mult &           40000 &  0.506 &  logn &  scpmi &  global &  0.7 &            cos \\
    mult &           50000 &  0.497 &  logn &  scpmi &  global &  0.7 &            cos \\
    kron &            1000 &  0.413 &  logn &  scpmi &    0.75 &  0.7 &  inner\_product \\
    kron &            2000 &  0.492 &  logn &  scpmi &    0.75 &  0.7 &  inner\_product \\
    kron &            3000 &  0.499 &  logn &  scpmi &    0.75 &  0.7 &  inner\_product \\
    kron &            5000 &  0.509 &  logn &   spmi &    0.75 &  0.7 &  inner\_product \\
    kron &           10000 &  0.514 &  logn &   spmi &    0.75 &  0.7 &  inner\_product \\
    kron &           20000 &  0.508 &  logn &   spmi &    0.75 &    1 &  inner\_product \\
    kron &           30000 &  0.512 &  logn &   spmi &    0.75 &    1 &  inner\_product \\
    kron &           40000 &  0.513 &  logn &   spmi &    0.75 &    1 &  inner\_product \\
    kron &           50000 &  0.516 &  logn &   spmi &    0.75 &    1 &  inner\_product \\
\bottomrule
\end{tabular}


  \caption{GS11 selection based on heuristics.}
  \label{tab:gs11-heuristics-selection}
\end{table}


\begin{table}
  \centering

  \begin{tabular}{lrrlllll}
\toprule
operator &  dimensionality &  PhraseRel &  freq &  discr &     cds &  neg &     similarity \\
\midrule
    head &            1000 &      0.714 &     1 &   spmi &    0.75 &  0.7 &    correlation \\
    head &            2000 &      0.750 &     n &   spmi &    0.75 &  1.4 &    correlation \\
    head &            3000 &      0.750 &  logn &   spmi &    0.75 &    2 &  inner\_product \\
    head &            5000 &      0.750 &     n &   spmi &    0.75 &    2 &    correlation \\
    head &           10000 &      0.750 &     1 &  scpmi &    0.75 &    2 &    correlation \\
    head &           20000 &      0.750 &     1 &   spmi &  global &  0.2 &  inner\_product \\
    head &           30000 &      0.750 &     n &  scpmi &  global &  1.4 &            cos \\
    head &           40000 &      0.750 &     n &  scpmi &  global &  1.4 &    correlation \\
    head &           50000 &      0.750 &     1 &   spmi &       1 &    2 &            cos \\
     add &            1000 &      0.893 &     1 &    pmi &  global &  N/A &            cos \\
     add &            2000 &      0.893 &     1 &   cpmi &       1 &  N/A &    correlation \\
     add &            3000 &      0.857 &     1 &   spmi &    0.75 &  0.7 &    correlation \\
     add &            5000 &      0.893 &     n &   spmi &    0.75 &    5 &    correlation \\
     add &           10000 &      0.857 &     1 &   spmi &       1 &  0.2 &  inner\_product \\
     add &           20000 &      0.857 &     1 &   spmi &    0.75 &  0.2 &  inner\_product \\
     add &           30000 &      0.893 &     n &   spmi &    0.75 &    7 &    correlation \\
     add &           40000 &      0.857 &     1 &   spmi &    0.75 &  0.5 &  inner\_product \\
     add &           50000 &      0.857 &     1 &   spmi &    0.75 &  0.5 &  inner\_product \\
    mult &            1000 &      0.929 &     1 &   cpmi &    0.75 &  N/A &    correlation \\
    mult &            2000 &      0.964 &  logn &   spmi &    0.75 &  0.2 &    correlation \\
    mult &            3000 &      0.929 &     1 &   cpmi &    0.75 &  N/A &            cos \\
    mult &            5000 &      0.964 &  logn &   spmi &    0.75 &  0.2 &    correlation \\
    mult &           10000 &      1.000 &     1 &  scpmi &       1 &  0.7 &    correlation \\
    mult &           20000 &      1.000 &  logn &   cpmi &       1 &  N/A &    correlation \\
    mult &           30000 &      0.964 &  logn &   cpmi &       1 &  N/A &    correlation \\
    mult &           40000 &      0.964 &  logn &   cpmi &       1 &  N/A &    correlation \\
    mult &           50000 &      0.964 &  logn &   cpmi &       1 &  N/A &    correlation \\
    kron &            1000 &      0.929 &     1 &  scpmi &  global &  0.2 &    correlation \\
    kron &            2000 &      0.929 &     1 &   cpmi &    0.75 &  N/A &    correlation \\
    kron &            3000 &      0.964 &     1 &  scpmi &  global &    1 &    correlation \\
    kron &            5000 &      0.893 &     1 &  scpmi &    0.75 &    1 &  inner\_product \\
    kron &           10000 &      0.893 &     1 &  scpmi &    0.75 &    1 &  inner\_product \\
    kron &           20000 &      0.893 &     1 &  scpmi &    0.75 &    1 &  inner\_product \\
    kron &           30000 &      0.929 &     1 &   spmi &       1 &  1.4 &  inner\_product \\
    kron &           40000 &      0.929 &     1 &  scpmi &  global &    2 &  inner\_product \\
    kron &           50000 &      0.929 &     1 &  scpmi &  global &    2 &  inner\_product \\
\bottomrule
\end{tabular}


  \caption{PhraseRel Max selection}
  \label{tab:phraserel-max-selection}
\end{table}

\begin{table}
  \centering

  \begin{tabular}{lrrlllll}
\toprule
operator &  dimensionality &  PhraseRel &  freq &  discr &     cds &  neg &     similarity \\
\midrule
    head &            1000 &      0.643 &     n &   spmi &    0.75 &  1.4 &    correlation \\
    head &            2000 &      0.750 &     n &   spmi &    0.75 &  1.4 &    correlation \\
    head &            3000 &      0.714 &     n &   spmi &    0.75 &  1.4 &    correlation \\
    head &            5000 &      0.714 &     n &  scpmi &    0.75 &  1.4 &    correlation \\
    head &           10000 &      0.750 &     n &  scpmi &  global &  1.4 &            cos \\
    head &           20000 &      0.750 &     n &  scpmi &  global &  1.4 &            cos \\
    head &           30000 &      0.750 &     n &  scpmi &  global &  1.4 &            cos \\
    head &           40000 &      0.750 &     n &  scpmi &  global &  1.4 &            cos \\
    head &           50000 &      0.750 &     n &  scpmi &  global &  1.4 &            cos \\
     add &            1000 &      0.786 &     1 &   spmi &  global &    2 &            cos \\
     add &            2000 &      0.821 &     1 &   spmi &  global &    2 &            cos \\
     add &            3000 &      0.857 &     1 &   spmi &  global &    2 &            cos \\
     add &            5000 &      0.821 &     1 &   spmi &  global &    2 &            cos \\
     add &           10000 &      0.821 &     1 &   spmi &  global &    2 &            cos \\
     add &           20000 &      0.821 &     1 &   spmi &  global &    2 &  inner\_product \\
     add &           30000 &      0.821 &     1 &   spmi &  global &    2 &  inner\_product \\
     add &           40000 &      0.786 &     1 &   spmi &  global &    2 &  inner\_product \\
     add &           50000 &      0.750 &     1 &   spmi &  global &    2 &  inner\_product \\
    mult &            1000 &      0.893 &  logn &  scpmi &  global &  0.5 &    correlation \\
    mult &            2000 &      0.893 &  logn &  scpmi &  global &  0.5 &    correlation \\
    mult &            3000 &      0.893 &  logn &  scpmi &  global &  0.5 &    correlation \\
    mult &            5000 &      0.929 &  logn &  scpmi &  global &  0.5 &    correlation \\
    mult &           10000 &      0.929 &  logn &  scpmi &  global &  0.5 &    correlation \\
    mult &           20000 &      0.964 &  logn &  scpmi &  global &  0.5 &    correlation \\
    mult &           30000 &      0.964 &  logn &  scpmi &  global &  0.5 &    correlation \\
    mult &           40000 &      0.929 &  logn &  scpmi &  global &  0.5 &    correlation \\
    mult &           50000 &      0.929 &  logn &  scpmi &  global &  0.5 &    correlation \\
    kron &            1000 &      0.857 &  logn &   spmi &       1 &  0.5 &    correlation \\
    kron &            2000 &      0.929 &  logn &   spmi &       1 &  0.5 &    correlation \\
    kron &            3000 &      0.929 &  logn &   spmi &       1 &  0.5 &    correlation \\
    kron &            5000 &      0.786 &     1 &   spmi &       1 &    1 &  inner\_product \\
    kron &           10000 &      0.857 &     1 &   spmi &       1 &    1 &  inner\_product \\
    kron &           20000 &      0.821 &     1 &   spmi &       1 &  1.4 &  inner\_product \\
    kron &           30000 &      0.929 &     1 &   spmi &       1 &  1.4 &  inner\_product \\
    kron &           40000 &      0.857 &     1 &   spmi &       1 &  1.4 &  inner\_product \\
    kron &           50000 &      0.821 &     1 &   spmi &       1 &  1.4 &  inner\_product \\
\bottomrule
\end{tabular}


  \caption{PhraseRel selection based on heuristics.}
  \label{tab:phraserel-heuristics-selection}
\end{table}


\begin{table}
  \centering
  \scriptsize
  \begin{tabular}{lrrrrrlllll}
\toprule
operator &  dimensionality &  KS14 &  GS11 &  PhraseRel &  compositional &  freq &  discr &     cds &  neg &     similarity \\
\midrule
    head &            1\,000 &  0.68 &  0.35 &       0.64 &           0.71 &     1 &   spmi &  global &    1 &    correlation \\
    head &            2\,000 &  0.71 &  0.36 &       0.64 &           0.73 &  logn &   spmi &  global &  1.4 &  inner\_product \\
    head &            3\,000 &  0.69 &  0.34 &       0.64 &           0.71 &  logn &   spmi &       1 &  0.5 &  inner\_product \\
    head &            5\,000 &  0.67 &  0.35 &       0.68 &           0.72 &  logn &   spmi &       1 &  0.5 &  inner\_product \\
    head &           10\,000 &  0.71 &  0.34 &       0.68 &           0.73 &  logn &   spmi &    0.75 &    1 &            cos \\
    head &           20\,000 &  0.71 &  0.35 &       0.71 &           \textbf{0.75} &  logn &   spmi &    0.75 &    1 &            cos \\
    head &           30\,000 &  0.71 &  0.35 &       0.71 &           \textbf{0.75} &  logn &   spmi &    0.75 &    1 &            cos \\
    head &           40\,000 &  0.72 &  0.35 &       0.71 &           \textbf{0.75} &  logn &  scpmi &    0.75 &    1 &            cos \\
    head &           50\,000 &  0.71 &  0.36 &       0.71 &           \textbf{0.75} &  logn &   spmi &    0.75 &    1 &            cos \\ \addlinespace
     add &            1\,000 &  0.76 &  0.33 &       0.89 &           \textbf{0.82} &     1 &   spmi &  global &  0.5 &    correlation \\
     add &            2\,000 &  0.78 &  0.30 &       0.89 &           0.81 &     1 &  scpmi &  global &  0.7 &    correlation \\
     add &            3\,000 &  0.77 &  0.30 &       0.86 &           0.79 &     1 &   spmi &  global &  0.5 &    correlation \\
     add &            5\,000 &  0.75 &  0.29 &       0.82 &           0.77 &  logn &  scpmi &    0.75 &  0.2 &    correlation \\
     add &           10\,000 &  0.77 &  0.28 &       0.82 &           0.77 &     1 &   spmi &    0.75 &  0.2 &    correlation \\
     add &           20\,000 &  0.76 &  0.25 &       0.79 &           0.73 &  logn &   cpmi &    0.75 &  N/A &    correlation \\
     add &           30\,000 &  0.75 &  0.24 &       0.79 &           0.72 &  logn &   cpmi &    0.75 &  N/A &    correlation \\
     add &           40\,000 &  0.73 &  0.26 &       0.71 &           0.70 &     1 &    pmi &  global &  N/A &    correlation \\
     add &           50\,000 &  0.72 &  0.25 &       0.71 &           0.69 &     1 &    pmi &  global &  N/A &    correlation \\ \addlinespace
    mult &            1\,000 &  0.75 &  0.45 &       0.89 &           0.89 &     1 &   spmi &  global &  0.5 &    correlation \\
    mult &            2\,000 &  0.74 &  0.47 &       0.89 &           0.90 &  logn &   spmi &  global &  0.5 &    correlation \\
    mult &            3\,000 &  0.75 &  0.47 &       0.89 &           0.90 &     1 &   spmi &  global &  0.5 &    correlation \\
    mult &            5\,000 &  0.74 &  0.49 &       0.89 &           0.91 &     1 &  scpmi &  global &  0.7 &            cos \\
    mult &           10\,000 &  0.75 &  0.50 &       1.00 &           0.95 &  logn &   spmi &  global &  0.5 &    correlation \\
    mult &           20\,000 &  0.77 &  0.53 &       0.96 &           \textbf{0.97} &     1 &  scpmi &  global &  0.7 &    correlation \\
    mult &           30\,000 &  0.74 &  0.50 &       0.96 &           0.94 &  logn &   spmi &       1 &  0.2 &    correlation \\
    mult &           40\,000 &  0.74 &  0.50 &       0.96 &           0.94 &  logn &   spmi &       1 &  0.2 &            cos \\
    mult &           50\,000 &  0.73 &  0.50 &       0.96 &           0.94 &  logn &   spmi &       1 &  0.2 &    correlation \\ \addlinespace
    kron &            1\,000 &  0.79 &  0.43 &       0.93 &           0.91 &     1 &  scpmi &  global &  0.2 &    correlation \\
    kron &            2\,000 &  0.77 &  0.46 &       0.93 &           0.92 &     1 &   spmi &    0.75 &  0.2 &    correlation \\
    kron &            3\,000 &  0.78 &  0.47 &       0.93 &           \textbf{0.93} &     1 &  scpmi &  global &  0.7 &            cos \\
    kron &            5\,000 &  0.77 &  0.50 &       0.86 &           0.92 &  logn &  scpmi &    0.75 &  0.7 &  inner\_product \\
    kron &           10\,000 &  0.73 &  0.51 &       0.86 &           0.91 &  logn &   spmi &    0.75 &  0.7 &  inner\_product \\
    kron &           20\,000 &  0.75 &  0.51 &       0.86 &           0.91 &  logn &   spmi &    0.75 &  0.7 &  inner\_product \\
    kron &           30\,000 &  0.76 &  0.50 &       0.86 &           0.91 &  logn &   spmi &    0.75 &  0.7 &  inner\_product \\
    kron &           40\,000 &  0.72 &  0.51 &       0.89 &           0.91 &  logn &   spmi &    0.75 &    1 &  inner\_product \\
    kron &           50\,000 &  0.72 &  0.51 &       0.89 &           0.91 &  logn &  scpmi &    0.75 &    1 &  inner\_product \\
\bottomrule
\end{tabular}


  \caption{Compositional (combined KS13, GS11 and PhraseRel) Max selection.}
  \label{tab:compositional-max-selection}
\end{table}

\begin{sidewaystable}
  \centering
  \scriptsize
  \begin{tabular}{lrrrrrlllll}
\toprule
operator &  dimensionality &  KS14 &  GS11 &  PhraseRel &  compositional &  freq &  discr &     cds &  neg &     similarity \\
\midrule
    head &            1\,000 &  0.68 &  0.33 &       0.64 &           0.70 &  logn &   spmi &  global &    1 &    correlation \\
    head &            2\,000 &  0.69 &  0.34 &       0.64 &           0.71 &  logn &   spmi &  global &    1 &    correlation \\
    head &            3\,000 &  0.68 &  0.33 &       0.61 &           0.69 &  logn &   spmi &  global &    1 &    correlation \\
    head &            5\,000 &  0.68 &  0.34 &       0.61 &           0.69 &  logn &   spmi &  global &    1 &            cos \\
    head &           10\,000 &  0.69 &  0.34 &       0.64 &           0.71 &  logn &   spmi &  global &  1.4 &            cos \\
    head &           20\,000 &  0.67 &  0.36 &       0.68 &           \textbf{0.73} &  logn &   spmi &  global &  1.4 &            cos \\
    head &           30\,000 &  0.67 &  0.36 &       0.68 &           \textbf{0.73} &  logn &   spmi &  global &  1.4 &            cos \\
    head &           40\,000 &  0.66 &  0.36 &       0.68 &           0.72 &  logn &   spmi &  global &  1.4 &            cos \\
    head &           50\,000 &  0.66 &  0.37 &       0.68 &           \textbf{0.73} &  logn &   spmi &  global &  1.4 &            cos \\ \addlinespace
     add &            1\,000 &  0.77 &  0.24 &       0.79 &           \textbf{0.73} &  logn &   spmi &  global &    1 &    correlation \\
     add &            2\,000 &  0.79 &  0.22 &       0.82 &           \textbf{0.73} &  logn &   spmi &  global &    1 &    correlation \\
     add &            3\,000 &  0.79 &  0.20 &       0.82 &           \textbf{0.73} &  logn &   spmi &  global &    1 &    correlation \\
     add &            5\,000 &  0.79 &  0.19 &       0.79 &           0.70 &  logn &   spmi &  global &    1 &    correlation \\
     add &           10\,000 &  0.79 &  0.18 &       0.79 &           0.70 &  logn &   spmi &  global &    1 &    correlation \\
     add &           20\,000 &  0.78 &  0.18 &       0.79 &           0.70 &  logn &   spmi &  global &    1 &    correlation \\
     add &           30\,000 &  0.78 &  0.17 &       0.79 &           0.69 &  logn &   spmi &  global &    1 &    correlation \\
     add &           40\,000 &  0.78 &  0.17 &       0.79 &           0.69 &  logn &   spmi &  global &    1 &    correlation \\
     add &           50\,000 &  0.77 &  0.16 &       0.79 &           0.68 &  logn &   spmi &  global &    1 &    correlation \\ \addlinespace
    mult &            1\,000 &  0.75 &  0.45 &       0.89 &           0.89 &     1 &   spmi &  global &  0.5 &    correlation \\
    mult &            2\,000 &  0.76 &  0.45 &       0.86 &           0.88 &     1 &   spmi &  global &  0.5 &    correlation \\
    mult &            3\,000 &  0.75 &  0.47 &       0.89 &           0.90 &     1 &   spmi &  global &  0.5 &    correlation \\
    mult &            5\,000 &  0.76 &  0.46 &       0.89 &           0.90 &     1 &   spmi &  global &  0.5 &    correlation \\
    mult &           10\,000 &  0.77 &  0.49 &       0.93 &           0.93 &     1 &  scpmi &  global &  0.7 &    correlation \\
    mult &           20\,000 &  0.77 &  0.53 &       0.96 &           \textbf{0.97} &     1 &  scpmi &  global &  0.7 &    correlation \\
    mult &           30\,000 &  0.76 &  0.51 &       0.89 &           0.93 &     1 &  scpmi &  global &  0.7 &    correlation \\
    mult &           40\,000 &  0.76 &  0.51 &       0.89 &           0.93 &     1 &  scpmi &  global &  0.7 &    correlation \\
    mult &           50\,000 &  0.77 &  0.48 &       0.89 &           0.91 &     1 &  scpmi &  global &  0.7 &    correlation \\ \addlinespace
    kron &            1\,000 &  0.78 &  0.42 &       0.86 &           0.87 &     1 &  scpmi &  global &  0.7 &    correlation \\
    kron &            2\,000 &  0.80 &  0.44 &       0.89 &           0.90 &     1 &  scpmi &  global &  0.7 &    correlation \\
    kron &            3\,000 &  0.80 &  0.47 &       0.89 &           \textbf{0.92} &     1 &  scpmi &  global &  0.7 &    correlation \\
    kron &            5\,000 &  0.77 &  0.50 &       0.86 &           \textbf{0.92} &  logn &  scpmi &    0.75 &  0.7 &  inner\_product \\
    kron &           10\,000 &  0.65 &  0.48 &       0.86 &           0.86 &  logn &   spmi &    0.75 &    1 &  inner\_product \\
    kron &           20\,000 &  0.69 &  0.51 &       0.89 &           0.90 &  logn &   spmi &    0.75 &    1 &  inner\_product \\
    kron &           30\,000 &  0.70 &  0.51 &       0.89 &           0.91 &  logn &   spmi &    0.75 &    1 &  inner\_product \\
    kron &           40\,000 &  0.72 &  0.51 &       0.89 &           0.91 &  logn &   spmi &    0.75 &    1 &  inner\_product \\
    kron &           50\,000 &  0.72 &  0.52 &       0.82 &           0.89 &  logn &   spmi &    0.75 &    1 &  inner\_product \\
\bottomrule
\end{tabular}


  \caption{Compositional (combined KS13, GS11 and PhraseRel) selection based on heuristics.}
  \label{tab:compositional-heuristics-selection}
\end{sidewaystable}


\begin{sidewaystable}
  \centering
  \scriptsize
  \begin{tabular}{lrrrrrrrlllll}
\toprule
operator &    dimensionality &           SimLex999 &            men &           KS14 &           GS11 &           PhraseRel &           universal &  freq &  discr &     cds &  neg &     similarity \\
\midrule
    head &            1\,000 &               0.35  &          0.68  &  \textbf{0.68} &  \textbf{0.33} &       \textbf{0.64} &       \textbf{0.79} &     1 &  scpmi &  global &    1 &            cos \\
    head &            2\,000 &       \textbf{0.36} &          0.70  &  \textbf{0.68} &  \textbf{0.35} &       \textbf{0.64} &       \textbf{0.82} &     1 &   spmi &  global &    1 &            cos \\
    head &            3\,000 &       \textbf{0.36} &          0.73  &  \textbf{0.69} &  \textbf{0.33} &       \textbf{0.64} &       \textbf{0.82} &  logn &  scpmi &  global &    1 &            cos \\
    head &            5\,000 &               0.35  &          0.74  &  \textbf{0.69} &  \textbf{0.34} &       \textbf{0.68} &       \textbf{0.83} &  logn &   spmi &    0.75 &  0.7 &            cos \\
    head &           10\,000 &       \textbf{0.37} &  \textbf{0.75} &  \textbf{0.67} &  \textbe{0.36} &       \textbf{0.64} &       \textbf{0.84} &  logn &  scpmi &       1 &  0.7 &            cos \\
    head &           20\,000 &       \textbf{0.37} &  \textbe{0.76} &  \textbf{0.71} &  \textbf{0.35} &       \textbe{0.71} &       \textbf{0.86} &  logn &   spmi &    0.75 &    1 &            cos \\
    head &           30\,000 &       \textbf{0.37} &  \textbe{0.76} &  \textbf{0.71} &  \textbf{0.35} &       \textbe{0.71} &       \textbf{0.86} &  logn &   spmi &    0.75 &    1 &            cos \\
    head &           40\,000 &       \textbe{0.38} &  \textbe{0.76} &  \textbe{0.72} &  \textbf{0.35} &       \textbe{0.71} &       \textbe{0.87} &  logn &  scpmi &    0.75 &    1 &            cos \\
    head &           50\,000 &       \textbe{0.38} &  \textbe{0.76} &  \textbf{0.71} &  \textbe{0.36} &       \textbe{0.71} &       \textbe{0.87} &  logn &   spmi &    0.75 &    1 &            cos \\ \addlinespace

     add &            1\,000 &               0.35  &          0.68  &  \textbf{0.77} &  \textbf{0.28} &       \textbf{0.86} &       \textbf{0.84} &     1 &  scpmi &  global &    1 &            cos \\
     add &            2\,000 &               0.33  &          0.68  &  \textbe{0.79} &  \textbe{0.29} &       \textbe{0.89} &       \textbf{0.84} &     1 &   cpmi &       1 &  N/A &    correlation \\
     add &            3\,000 &               0.34  &          0.72  &  \textbf{0.78} &  \textbf{0.26} &       \textbf{0.82} &       \textbf{0.84} &  logn &   cpmi &       1 &  N/A &    correlation \\
     add &            5\,000 &               0.35  &          0.73  &  \textbf{0.78} &  \textbf{0.25} &       \textbf{0.82} &       \textbf{0.84} &  logn &   cpmi &       1 &  N/A &    correlation \\
     add &           10\,000 &       \textbf{0.36} &          0.74  &  \textbf{0.78} &  \textbf{0.26} &       \textbf{0.82} &       \textbe{0.85} &  logn &   cpmi &       1 &  N/A &    correlation \\
     add &           20\,000 &       \textbf{0.37} &          0.74  &  \textbf{0.76} &  \textbf{0.25} &       \textbf{0.79} &       \textbf{0.84} &  logn &   cpmi &    0.75 &  N/A &    correlation \\
     add &           30\,000 &       \textbe{0.38} &  \textbe{0.76} &  \textbf{0.78} &          0.16  &               0.82  &       \textbf{0.84} &  logn &  scpmi &    0.75 &  0.7 &    correlation \\
     add &           40\,000 &       \textbe{0.38} &  \textbe{0.76} &  \textbf{0.78} &          0.17  &               0.79  &       \textbf{0.84} &  logn &  scpmi &    0.75 &  0.7 &    correlation \\
     add &           50\,000 &       \textbe{0.38} &  \textbe{0.76} &  \textbf{0.78} &          0.16  &               0.79  &       \textbf{0.84} &  logn &  scpmi &    0.75 &  0.7 &    correlation \\ \addlinespace

    mult &            1\,000 &               0.34  &          0.66  &  \textbf{0.71} &  \textbf{0.44} &       \textbf{0.89} &       \textbf{0.87} &  logn &   spmi &  global &  0.7 &            cos \\
    mult &            2\,000 &               0.35  &          0.69  &  \textbf{0.73} &  \textbf{0.46} &       \textbf{0.89} &       \textbf{0.89} &  logn &   spmi &       1 &  0.2 &            cos \\
    mult &            3\,000 &       \textbf{0.36} &          0.71  &  \textbf{0.73} &  \textbf{0.47} &       \textbf{0.89} &       \textbf{0.91} &  logn &   spmi &  global &  0.7 &            cos \\
    mult &            5\,000 &       \textbf{0.36} &          0.72  &  \textbf{0.73} &  \textbf{0.48} &       \textbf{0.86} &       \textbf{0.91} &  logn &   spmi &  global &  0.7 &            cos \\
    mult &           10\,000 &       \textbf{0.37} &  \textbf{0.75} &  \textbf{0.76} &  \textbf{0.45} &       \textbf{0.96} &       \textbf{0.94} &  logn &  scpmi &  global &    1 &            cos \\
    mult &           20\,000 &       \textbe{0.38} &  \textbe{0.76} &  \textbf{0.76} &  \textbf{0.48} &       \textbf{0.89} &       \textbf{0.94} &  logn &  scpmi &  global &    1 &            cos \\
    mult &           30\,000 &       \textbe{0.38} &  \textbe{0.76} &  \textbf{0.74} &  \textbf{0.48} &       \textbf{0.89} &       \textbf{0.94} &  logn &  scpmi &  global &    1 &            cos \\
    mult &           40\,000 &       \textbe{0.38} &  \textbe{0.76} &  \textbf{0.74} &  \textbe{0.49} &       \textbf{0.89} &       \textbe{0.95} &  logn &  scpmi &  global &    1 &            cos \\
    mult &           50\,000 &       \textbf{0.37} &  \textbe{0.76} &  \textbe{0.77} &  \textbf{0.45} &       \textbe{0.93} &       \textbf{0.94} &  logn &   spmi &  global &  1.4 &            cos \\ \addlinespace

    kron &            1\,000 &       \textbf{0.35} &          0.68  &  \textbf{0.79} &          0.39  &       \textbf{0.93} &       \textbf{0.88} &  logn &   spmi &  global &    1 &            cos \\
    kron &            2\,000 &       \textbe{0.36} &          0.72  &  \textbe{0.80} &          0.41  &       \textbf{0.93} &       \textbf{0.91} &  logn &  scpmi &  global &    1 &            cos \\
    kron &            3\,000 &       \textbe{0.36} &          0.71  &  \textbe{0.80} &          0.42  &       \textbe{0.96} &       \textbe{0.92} &     1 &  scpmi &  global &    1 &            cos \\
    kron &            5\,000 &               0.28  &          0.70  &  \textbf{0.77} &  \textbf{0.50} &       \textbf{0.86} &       \textbf{0.87} &  logn &  scpmi &    0.75 &  0.7 &  inner\_product \\
    kron &           10\,000 &               0.28  &          0.71  &  \textbf{0.73} &  \textbe{0.51} &       \textbf{0.86} &       \textbf{0.87} &  logn &   spmi &    0.75 &  0.7 &  inner\_product \\
    kron &           20\,000 &               0.28  &          0.72  &  \textbf{0.75} &  \textbe{0.51} &       \textbf{0.86} &       \textbf{0.88} &  logn &   spmi &    0.75 &  0.7 &  inner\_product \\
    kron &           30\,000 &               0.29  &  \textbf{0.73} &  \textbf{0.76} &  \textbf{0.50} &       \textbf{0.86} &       \textbf{0.88} &  logn &   spmi &    0.75 &  0.7 &  inner\_product \\
    kron &           40\,000 &               0.29  &  \textbf{0.73} &          0.72  &  \textbe{0.51} &       \textbf{0.89} &       \textbf{0.88} &  logn &   spmi &    0.75 &    1 &  inner\_product \\
    kron &           50\,000 &               0.29  &  \textbe{0.74} &          0.72  &  \textbe{0.51} &       \textbf{0.89} &       \textbf{0.88} &  logn &  scpmi &    0.75 &    1 &  inner\_product \\
\bottomrule
\end{tabular}


  \caption{Universal, Max selection.}
  \label{tab:universal-max-selection}
\end{sidewaystable}

\begin{sidewaystable}
  \centering
  \scriptsize
  \begin{tabular}{lrrrrrrrlllll}
\toprule
operator &  dimensionality &             SimLex999 &            men &           KS14 &           GS11 &           PhraseRel &           universal &  freq &  discr &     cds &  neg &     similarity \\
\midrule
    head &            1\,000 &               0.33  &          0.68  &  \textbf{0.67} &          0.29  &       \textbf{0.68} &       \textbf{0.78} &     1 &  scpmi &    0.75 &  0.7 &            cos \\
    head &            2\,000 &               0.35  &          0.72  &  \textbf{0.70} &  \textbf{0.31} &       \textbf{0.64} &       \textbf{0.81} &     1 &  scpmi &    0.75 &  0.7 &            cos \\
    head &            3\,000 &       \textbf{0.36} &          0.73  &  \textbf{0.69} &          0.30  &       \textbf{0.64} &       \textbf{0.81} &     1 &  scpmi &    0.75 &  0.7 &            cos \\
    head &            5\,000 &               0.35  &          0.73  &  \textbf{0.68} &  \textbf{0.33} &       \textbf{0.68} &       \textbf{0.82} &     1 &   spmi &    0.75 &  0.7 &            cos \\
    head &           10\,000 &       \textbf{0.36} &  \textbf{0.75} &  \textbe{0.71} &  \textbf{0.34} &       \textbf{0.68} &       \textbf{0.84} &  logn &   spmi &    0.75 &    1 &            cos \\
    head &           20\,000 &       \textbf{0.37} &  \textbe{0.76} &  \textbe{0.71} &  \textbf{0.35} &       \textbe{0.71} &       \textbf{0.86} &  logn &   spmi &    0.75 &    1 &            cos \\
    head &           30\,000 &       \textbf{0.37} &  \textbe{0.76} &  \textbe{0.71} &  \textbf{0.35} &       \textbe{0.71} &       \textbf{0.86} &  logn &   spmi &    0.75 &    1 &            cos \\
    head &           40\,000 &       \textbe{0.38} &  \textbe{0.76} &  \textbe{0.71} &  \textbf{0.35} &       \textbe{0.71} &       \textbe{0.87} &  logn &   spmi &    0.75 &    1 &            cos \\
    head &           50\,000 &       \textbe{0.38} &  \textbe{0.76} &  \textbe{0.71} &  \textbe{0.36} &       \textbe{0.71} &       \textbe{0.87} &  logn &   spmi &    0.75 &    1 &            cos \\ \addlinespace

     add &            1\,000 &               0.31  &          0.64  &  \textbf{0.76} &  \textbe{0.34} &       \textbf{0.86} &       \textbf{0.82} &     1 &  scpmi &  global &  0.7 &    correlation \\
     add &            2\,000 &               0.33  &          0.68  &  \textbe{0.78} &  \textbf{0.30} &       \textbe{0.89} &       \textbe{0.84} &     1 &  scpmi &  global &  0.7 &    correlation \\
     add &            3\,000 &               0.33  &          0.68  &  \textbe{0.78} &          0.28  &       \textbf{0.82} &       \textbf{0.82} &     1 &  scpmi &  global &  0.7 &    correlation \\
     add &            5\,000 &               0.34  &          0.69  &  \textbe{0.78} &          0.26  &       \textbf{0.82} &       \textbf{0.82} &     1 &  scpmi &  global &  0.7 &    correlation \\
     add &           10\,000 &       \textbf{0.36} &          0.73  &  \textbe{0.78} &          0.25  &       \textbf{0.79} &       \textbe{0.84} &  logn &  scpmi &  global &  0.7 &    correlation \\
     add &           20\,000 &       \textbe{0.38} &  \textbe{0.76} &  \textbe{0.78} &          0.16  &       \textbf{0.82} &       \textbe{0.84} &  logn &  scpmi &    0.75 &  0.7 &    correlation \\
     add &           30\,000 &       \textbe{0.38} &  \textbe{0.76} &  \textbe{0.78} &          0.16  &       \textbf{0.82} &       \textbe{0.84} &  logn &  scpmi &    0.75 &  0.7 &    correlation \\
     add &           40\,000 &       \textbe{0.38} &  \textbe{0.76} &  \textbe{0.78} &          0.17  &       \textbf{0.79} &       \textbe{0.84} &  logn &  scpmi &    0.75 &  0.7 &    correlation \\
     add &           50\,000 &       \textbe{0.38} &  \textbe{0.76} &  \textbe{0.78} &          0.16  &       \textbf{0.79} &       \textbe{0.84} &  logn &  scpmi &    0.75 &  0.7 &    correlation \\ \addlinespace

    mult &            1\,000 &               0.30  &          0.62  &  \textbf{0.75} &          0.45  &       \textbe{0.89} &       \textbf{0.84} &     1 &   spmi &  global &  0.5 &    correlation \\
    mult &            2\,000 &               0.32  &          0.66  &  \textbe{0.76} &          0.45  &       \textbf{0.86} &       \textbf{0.86} &     1 &   spmi &  global &  0.5 &    correlation \\
    mult &            3\,000 &               0.32  &          0.66  &  \textbf{0.75} &  \textbf{0.47} &       \textbe{0.89} &       \textbf{0.88} &     1 &   spmi &  global &  0.5 &    correlation \\
    mult &            5\,000 &               0.32  &          0.67  &  \textbe{0.76} &  \textbf{0.46} &       \textbe{0.89} &       \textbf{0.87} &     1 &   spmi &  global &  0.5 &    correlation \\
    mult &           10\,000 &       \textbf{0.36} &          0.73  &  \textbe{0.76} &  \textbf{0.50} &       \textbe{0.89} &       \textbf{0.93} &  logn &  scpmi &  global &  0.7 &    correlation \\
    mult &           20\,000 &       \textbe{0.37} &  \textbe{0.75} &  \textbf{0.75} &  \textbf{0.50} &       \textbe{0.89} &       \textbe{0.94} &  logn &  scpmi &  global &  0.7 &    correlation \\
    mult &           30\,000 &       \textbe{0.37} &  \textbe{0.75} &  \textbf{0.74} &  \textbf{0.50} &       \textbe{0.89} &       \textbe{0.94} &  logn &  scpmi &  global &  0.7 &    correlation \\
    mult &           40\,000 &       \textbe{0.37} &  \textbe{0.75} &  \textbf{0.74} &  \textbe{0.51} &       \textbe{0.89} &       \textbe{0.94} &  logn &  scpmi &  global &  0.7 &    correlation \\
    mult &           50\,000 &       \textbe{0.37} &  \textbe{0.75} &  \textbf{0.74} &  \textbf{0.50} &       \textbe{0.89} &       \textbe{0.94} &  logn &  scpmi &  global &  0.7 &    correlation \\ \addlinespace
  
    kron &            1\,000 &       \textbf{0.34} &          0.65  &  \textbf{0.78} &          0.42  &       \textbf{0.89} &       \textbf{0.87} &     1 &   spmi &  global &  0.7 &            cos \\
    kron &            2\,000 &       \textbe{0.35} &          0.68  &  \textbf{0.79} &          0.43  &       \textbf{0.89} &       \textbf{0.90} &     1 &   spmi &  global &  0.7 &            cos \\
    kron &            3\,000 &       \textbe{0.35} &          0.69  &  \textbe{0.80} &          0.45  &       \textbe{0.93} &       \textbe{0.91} &     1 &   spmi &  global &  0.7 &            cos \\
    kron &            5\,000 &               0.30  &          0.68  &  \textbf{0.76} &          0.44  &       \textbf{0.86} &       \textbf{0.86} &     1 &   spmi &    0.75 &  0.7 &  inner\_product \\
    kron &           10\,000 &               0.30  &          0.67  &  \textbf{0.78} &          0.43  &       \textbf{0.86} &       \textbf{0.85} &     1 &   spmi &    0.75 &  0.7 &  inner\_product \\
    kron &           20\,000 &               0.28  &  \textbe{0.73} &          0.69  &  \textbf{0.51} &       \textbf{0.89} &       \textbf{0.87} &  logn &   spmi &    0.75 &    1 &  inner\_product \\
    kron &           30\,000 &               0.29  &  \textbe{0.73} &          0.70  &  \textbf{0.51} &       \textbf{0.89} &       \textbf{0.88} &  logn &   spmi &    0.75 &    1 &  inner\_product \\
    kron &           40\,000 &               0.29  &  \textbe{0.73} &  \textbf{0.72} &  \textbf{0.51} &       \textbf{0.89} &       \textbf{0.88} &  logn &   spmi &    0.75 &    1 &  inner\_product \\
    kron &           50\,000 &               0.29  &  \textbe{0.73} &          0.72  &  \textbe{0.52} &       \textbf{0.82} &       \textbf{0.87} &  logn &   spmi &    0.75 &    1 &  inner\_product \\
\bottomrule
\end{tabular}


  \caption{Universal (operator dependent) Heuristics selection}
  \label{tab:universal-heuristics-selection}
\end{sidewaystable}


\begin{figure}
  \centering

  \begin{subfigure}[t]{\textwidth}
    \includegraphics[width=\textwidth]{supplement/figures/universal-results-simlex999}
    \caption{SimLex999.}
    \label{fig:universal-results-simlex}
  \end{subfigure}

  \begin{subfigure}[t]{\textwidth}
    \includegraphics[width=\textwidth]{supplement/figures/universal-results-men}
    \caption{MEN.}
    \label{fig:universal-results-men}
  \end{subfigure}


  \begin{subfigure}[t]{\textwidth}
    \includegraphics[width=\textwidth]{supplement/figures/universal-results-ks14}
    \caption{KS14.}
    \label{fig:universal-results-ks14}
  \end{subfigure}

  \begin{subfigure}[t]{\textwidth}
    \includegraphics[width=\textwidth]{supplement/figures/universal-results-gs11}
    \caption{GS11.}
    \label{fig:universal-results-gs11}
  \end{subfigure}

  \begin{subfigure}[t]{\textwidth}
    \includegraphics[width=\textwidth]{supplement/figures/universal-results-PhraseRel}
    \caption{PhraseRel.}
    \label{fig:universal-results-phraserel}
  \end{subfigure}


  \caption{Performance of models based on the selection over the average universal performance}
  \label{fig:universal-results}
\end{figure}

%%% Local Variables:
%%% mode: latex
%%% TeX-master: "../thesis"
%%% End:


\begin{figure}
% \begin{wrapfigure}{O}{0.5\textwidth}
  % \vspace{-30pt}
  \centering

  \begin{subfigure}[t]{\textwidth}
    \includegraphics[width=1.1\textwidth]{supplement/figures/universal-interaction-freq}

  \caption{\texttt{freq}}
  \label{fig:universal-freq}
  \end{subfigure}

  \begin{subfigure}[t]{\textwidth}
    \includegraphics[width=1.1\textwidth]{supplement/figures/universal-interaction-neg}

  \caption{\texttt{neg}}
  \label{fig:universal-neg}
  \end{subfigure}

  \begin{subfigure}[t]{\textwidth}
  \includegraphics[width=1.1\textwidth]{supplement/figures/universal-interaction-similarity}

  \caption{\texttt{similarity}}
  \label{fig:universal-similarity}
  \end{subfigure}

  \begin{subfigure}[t]{\textwidth}
    \includegraphics[width=1.1\textwidth]{supplement/figures/universal-interaction-cds}

  \caption{\texttt{cds}}
  \label{fig:universal-cds}
  \end{subfigure}

  \begin{subfigure}[t]{\textwidth}
  \includegraphics[width=1.1\textwidth]{supplement/figures/universal-interaction-discr}

  \caption{\texttt{discr}}
  \label{fig:universal-discr}
  \end{subfigure}

  \caption{Universal (parameter dependant) parameter influence}
  \label{fig:universal-parameters}
\end{figure}


\begin{sidewaystable}
  \centering
  \scriptsize
  \begin{tabular}{llllllrrrrrrrrrrr}
\toprule
      &      &   &   &     &             &      head &        &    add &        &           &   mult &        &           &   kron &        &           \\
      &      &   &   &     &             & SimLex999 &    men &   KS14 &   GS11 & PhraseRel &   KS14 &   GS11 & PhraseRel &   KS14 &   GS11 & PhraseRel \\
dimensionality & discr & cds & freq & neg & similarity &           &        &        &        &           &        &        &           &        &        &           \\
\midrule
1000  & scpmi & global & 1 & 0.7 & cos &     0.335 &  0.647 &  0.740 &  0.321 &     0.857 &  0.726 &  0.443 &     0.893 &  0.763 &  0.427 &     0.857 \\
2000  &      &   &   &     &             &     0.352 &  0.684 &  0.754 &  0.293 &     0.786 &  0.743 &  0.446 &     0.821 &  0.784 &  0.443 &     0.893 \\
3000  &      &   &   &     &             &     0.348 &  0.692 &  0.757 &  0.291 &     0.821 &  0.742 &  0.485 &     0.857 &  0.784 &  0.467 &     0.929 \\
5000  & cpmi & 1 & logn & N/A & correlation &     0.350 &  0.725 &  0.782 &  0.255 &     0.821 &  0.753 &  0.427 &     0.893 &    NaN &    NaN &       NaN \\
10000 & scpmi & global &   & 0.7 &             &     0.360 &  0.733 &  0.779 &  0.249 &     0.786 &  0.756 &  0.495 &     0.893 &    NaN &    NaN &       NaN \\
20000 & cpmi & 1 &   & N/A &             &     0.372 &  0.749 &  0.775 &  0.243 &     0.714 &  0.738 &  0.440 &     1.000 &    NaN &    NaN &       NaN \\
30000 & scpmi &   &   & 0.7 &             &     0.373 &  0.758 &  0.785 &  0.174 &     0.786 &  0.747 &  0.481 &     0.893 &    NaN &    NaN &       NaN \\
40000 &      &   &   &     &             &     0.373 &  0.757 &  0.779 &  0.171 &     0.786 &  0.749 &  0.492 &     0.893 &    NaN &    NaN &       NaN \\
50000 &      & global &   &     &             &     0.370 &  0.748 &  0.762 &  0.201 &     0.714 &  0.737 &  0.499 &     0.893 &    NaN &    NaN &       NaN \\
\bottomrule
\end{tabular}


  \caption{Single, Max selection.}
  \label{tab:single-max-selection}
\end{sidewaystable}

\begin{sidewaystable}
  \centering
  \scriptsize
  \begin{tabular}{llllllrrrrrrrrrrr}
\toprule
      &       &        &   &     &             &      head &       &   add &       &           &  mult &       &           &  kron &       &           \\
      &       &        &   &     &             & SimLex999 &   men &  KS14 &  GS11 & PhraseRel &  KS14 &  GS11 & PhraseRel &  KS14 &  GS11 & PhraseRel \\
dimensionality & discr & cds & freq & neg & similarity &           &       &       &       &           &       &       &           &       &       &           \\
\midrule
1\,000  & scpmi & global & 1    & 0.7 & cos         &              0.33  &  0.65 &  0.74 &  \textbf{0.32} &      \textbf{0.86} &  0.73 &  0.44 &      0.89 &  0.76 &  0.43 &      0.86 \\
2\,000  & scpmi & global & 1    & 0.7 & cos         &              0.35  &  0.68 &  0.75 &  0.29 &      0.79 &  0.74 &  0.45 &      0.82 &  \textbf{0.78} &  0.44 &      0.89 \\
3\,000  & scpmi & global & 1    & 0.7 & cos         &              0.35  &  0.69 &  0.76 &  0.29 &      0.82 &  0.74 &  0.48 &      0.86 &  \textbf{0.78} &  \textbf{0.47} &      \textbf{0.93} \\
5\,000  & scpmi & global & 1    & 0.7 & cos         &              0.34  &  0.70 &  0.75 &  0.27 &      0.82 &  0.74 &  0.49 &      0.89 &       &       &           \\
10\,000 & scpmi & global & 1    & 0.7 & cos         &              0.33  &  0.70 &  0.74 &  0.24 &      0.75 &  \textbf{0.75} &  0.49 &      \textbf{0.93} &       &       &           \\
20\,000 & scpmi & global & logn & 0.7 & correlation &      \textbf{0.37} &  \textbf{0.75} &  \textbf{0.77} &  0.23 &      0.75 &  \textbf{0.75} &  0.50 &      0.89 &       &       &           \\
30\,000 & scpmi & global & logn & 0.7 & correlation &      \textbf{0.37} &  \textbf{0.75} &  \textbf{0.77} &  0.22 &      0.71 &  0.74 &  0.50 &      0.89 &       &       &           \\
40\,000 & scpmi & global & logn & 0.7 & correlation &      \textbf{0.37} &  \textbf{0.75} &  \textbf{0.77} &  0.21 &      0.71 &  0.74 &  \textbf{0.51} &      0.89 &       &       &           \\
50\,000 & scpmi & global & logn & 0.7 & correlation &      \textbf{0.37} &  \textbf{0.75} &  0.76 &  0.20 &      0.71 &  0.74 &  0.50 &      0.89 &       &       &           \\
\bottomrule
\end{tabular}


  \caption{Single (operator-independent) heuristics selection}
  \label{tab:single-heuristics-selection}
\end{sidewaystable}


% \begin{wrapfigure}{O}{0.5\textwidth}
\begin{figure}[b]
  % \vspace{-30pt}
  \centering

  \begin{subfigure}[t]{0.49\textwidth}
  \includegraphics[width=\textwidth]{supplement/figures/single-interaction-freq}

  \caption{\texttt{freq}}
  \label{fig:single-similarity}
  \end{subfigure}
  \begin{subfigure}[t]{0.49\textwidth}

  \includegraphics[width=1.1\textwidth]{supplement/figures/single-interaction-similarity}

  \caption{Similarity measure}
  \label{fig:single-similarity}
  \end{subfigure}

  \begin{subfigure}[t]{0.49\textwidth}

  \includegraphics[width=\textwidth]{supplement/figures/single-interaction-neg}

  \caption{\texttt{neg}}
  \label{fig:single-neg}
  \end{subfigure}
  \begin{subfigure}[t]{0.49\textwidth}

  \includegraphics[width=\textwidth]{supplement/figures/single-interaction-discr}

  \caption{\texttt{discr}}
  \label{fig:single-discr}
  \end{subfigure}

  \begin{subfigure}[t]{0.49\textwidth}

  \includegraphics[width=\textwidth]{supplement/figures/single-interaction-cds}

  \caption{\texttt{cds}}
  \label{fig:single-cds}
  \end{subfigure}

  \caption{Single.}
  \label{fig:single-params}
\end{figure}

\clearpage

\begin{sidewaystable}
  \centering
  \scriptsize
  \begin{tabular}{lllllllrrrrrr}
\toprule
       & {} &      &   &      &     &   &                                         copy-object &           copy-subject &          frobenius-add &            frobenius-mult &           frobenius-outer &  relational \\
selection & operator & dimensionality & freq & discr & neg & cds &              &               &                &                 &                  &             \\
\midrule
single                 & {}   & 1\,000 & 1     & scpmi & 0.7 & global &         \textbf{0.60} &          \textbf{0.70} &           \textbf{0.71} &            \textbf{0.69}  &             \textbf{0.73}  &        \textbf{0.73}  \\
single                 & {}   & 2\,000 & 1     & scpmi & 0.7 & global &         \textbf{0.62} &          \textbf{0.70} &           \textbf{0.72} &            \textbf{0.72}  &             \textbf{0.74}  &        \textbf{0.74}  \\
single                 & {}   & 3\,000 & 1     & scpmi & 0.7 & global &         \textbf{0.61} &          \textbf{0.71} &           \textbf{0.73} &            \textbf{0.72}  &             \textbf{0.75}  &        \textbf{0.75}  \\ \addlinespace
universal (max)        & add  & 1\,000 & 1     & scpmi & 1.0 & global &         \textbf{0.60} &          \textbe{0.74} &           \textbf{0.75} &            \textbf{0.67}  &             \textbf{0.74}  &        \textbf{0.74}  \\
universal (max)        & add  & 2\,000 & 1     & cpmi  & N/A & 1      &         \textbf{0.61} &          \textbf{0.73} &           \textbf{0.75} &            \textbe{0.75}  &             \textbf{0.76}  &        \textbf{0.76}  \\
universal (max)        & add  & 3\,000 & logn  & cpmi  & N/A & 1      &         \textbf{0.62} &          \textbf{0.73} &           \textbf{0.75} &            \textbf{0.71}  &             \textbf{0.76}  &        \textbf{0.76}  \\ \addlinespace
universal (max)        & mult & 1\,000 & logn  & spmi  & 0.7 & global &         \textbf{0.60} &          \textbf{0.71} &           \textbf{0.72} &            \textbf{0.68}  &             \textbf{0.73}  &        \textbf{0.74}  \\
universal (max)        & mult & 2\,000 & logn  & spmi  & 0.2 & 1      &         \textbf{0.61} &          \textbf{0.70} &           \textbf{0.72} &            \textbf{0.69}  &             \textbf{0.74}  &        \textbf{0.74}  \\
universal (max)        & mult & 3\,000 & logn  & spmi  & 0.7 & global &         \textbf{0.62} &          \textbf{0.72} &           \textbf{0.74} &            \textbf{0.71}  &             \textbf{0.74}  &        \textbf{0.74}  \\ \addlinespace
universal (max)        & kron & 1\,000 & logn  & spmi  & 1.0 & global &         \textbf{0.60} &          \textbf{0.73} &           \textbf{0.74} &            \textbf{0.66}  &             \textbf{0.74}  &        \textbf{0.74}  \\
universal (max)        & kron & 2\,000 & logn  & scpmi & 1.0 & global &         \textbf{0.62} &          \textbe{0.74} &           \textbf{0.75} &            \textbf{0.68}  &             \textbf{0.75}  &        \textbf{0.75}  \\
universal (max)        & kron & 3\,000 & 1     & scpmi & 1.0 & global &         \textbe{0.63} &          \textbf{0.73} &           \textbe{0.76} &            \textbf{0.69}  &             \textbe{0.77}  &        \textbe{0.77}  \\ \addlinespace
universal (heuristics) & add  & 1\,000 & 1     & scpmi & 0.7 & global &         \textbf{0.59} &          \textbf{0.72} &           \textbf{0.73} &            \textbf{0.73}  &             \textbf{0.74}  &        \textbf{0.74}  \\
universal (heuristics) & add  & 2\,000 & 1     & scpmi & 0.7 & global &         \textbf{0.61} &          \textbf{0.72} &           \textbf{0.74} &            \textbf{0.73}  &             \textbf{0.75}  &        \textbf{0.75}  \\
universal (heuristics) & add  & 3\,000 & 1     & scpmi & 0.7 & global &         \textbf{0.61} &          \textbf{0.72} &           \textbf{0.75} &            \textbf{0.73}  &             \textbf{0.75}  &        \textbf{0.75}  \\ \addlinespace
universal (heuristics) & mult & 1\,000 & 1     & spmi  & 0.5 & global &         \textbf{0.59} &          \textbf{0.71} &           \textbf{0.72} &            \textbf{0.72}  &             \textbf{0.73}  &        \textbf{0.73}  \\
universal (heuristics) & mult & 2\,000 & 1     & spmi  & 0.5 & global &         \textbf{0.60} &          \textbf{0.72} &           \textbf{0.73} &            \textbf{0.73}  &             \textbf{0.74}  &        \textbf{0.74}  \\
universal (heuristics) & mult & 3\,000 & 1     & spmi  & 0.5 & global &         \textbf{0.61} &          \textbf{0.71} &           \textbf{0.74} &            \textbf{0.73}  &             \textbf{0.74}  &        \textbf{0.74}  \\ \addlinespace
universal (heuristics) & kron & 1\,000 & 1     & spmi  & 0.7 & global &         \textbf{0.60} &          \textbf{0.71} &           \textbf{0.72} &            \textbf{0.69}  &             \textbf{0.73}  &        \textbf{0.74}  \\
universal (heuristics) & kron & 2\,000 & 1     & spmi  & 0.7 & global &         \textbf{0.62} &          \textbf{0.72} &           \textbf{0.74} &            \textbf{0.71}  &             \textbf{0.74}  &        \textbf{0.75}  \\
universal (heuristics) & kron & 3\,000 & 1     & spmi  & 0.7 & global &         \textbf{0.62} &          \textbf{0.71} &           \textbf{0.74} &            \textbf{0.72}  &             \textbf{0.74}  &        \textbf{0.75}  \\
\bottomrule
\end{tabular}


  \caption{Frobenius operators on KS14}
  \label{tab:frobenius-ks14-results}
\end{sidewaystable}

\begin{sidewaystable}
  \centering
  \scriptsize
  \begin{tabular}{lllllllrrrrrr}
\toprule
       & {} &      &   &      &     &   &  copy-object &  copy-subject &  frobenious-add &  frobenious-mult &  frobenious-outer &  relational \\
selection & selection\_operator & dimensionality & freq & discr & neg & cds &              &               &                 &                  &                   &             \\
\midrule
single & {} & 1000 & 1 & scpmi & 0.7 & global &        0.157 &         0.318 &           0.283 &            0.230 &             0.289 &       0.305 \\
       & {} & 2000 &   &      &     &   &        0.180 &         0.271 &           0.247 &            0.211 &             0.259 &       0.275 \\
       & {} & 3000 &   &      &     &   &        0.192 &         0.259 &           0.249 &            0.223 &             0.261 &       0.282 \\
universal\_max & add & 1000 &   &      & 1.0 &   &        0.209 &         0.380 &           0.360 &            0.251 &             0.365 &       0.372 \\
       & {} & 2000 &   & cpmi & N/A & 1 &        0.132 &         0.238 &           0.205 &            0.158 &             0.218 &       0.238 \\
       & {} & 3000 & logn &      &     &   &        0.145 &         0.237 &           0.211 &            0.178 &             0.230 &       0.249 \\
       & mult & 1000 &   & spmi & 0.7 & global &        0.169 &         0.377 &           0.333 &            0.245 &             0.341 &       0.353 \\
       & {} & 2000 &   &      & 0.2 & 1 &        0.181 &         0.315 &           0.278 &            0.225 &             0.287 &       0.302 \\
       & {} & 3000 &   &      & 0.7 & global &        0.201 &         0.323 &           0.298 &            0.247 &             0.312 &       0.331 \\
       & kron & 1000 &   &      & 1.0 &   &        0.199 &         0.402 &           0.374 &            0.267 &             0.385 &       0.393 \\
       & {} & 2000 &   & scpmi &     &   &        0.239 &         0.360 &           0.335 &            0.268 &             0.362 &       0.376 \\
       & {} & 3000 & 1 &      &     &   &        0.278 &         0.344 &           0.338 &            0.295 &             0.363 &       0.375 \\
universal\_heuristics & add & 1000 &   &      & 0.7 &   &        0.154 &         0.317 &           0.284 &            0.231 &             0.292 &       0.310 \\
       & {} & 2000 &   &      &     &   &        0.173 &         0.279 &           0.250 &            0.217 &             0.261 &       0.278 \\
       & {} & 3000 &   &      &     &   &        0.180 &         0.270 &           0.251 &            0.225 &             0.261 &       0.282 \\
       & mult & 1000 &   & spmi & 0.5 &   &        0.140 &         0.321 &           0.270 &            0.240 &             0.274 &       0.289 \\
       & {} & 2000 &   &      &     &   &        0.158 &         0.281 &           0.244 &            0.204 &             0.257 &       0.269 \\
       & {} & 3000 &   &      &     &   &        0.166 &         0.273 &           0.245 &            0.219 &             0.261 &       0.276 \\
       & kron & 1000 &   &      & 0.7 &   &        0.175 &         0.348 &           0.310 &            0.232 &             0.325 &       0.334 \\
       & {} & 2000 &   &      &     &   &        0.203 &         0.304 &           0.267 &            0.221 &             0.296 &       0.312 \\
       & {} & 3000 &   &      &     &   &        0.204 &         0.303 &           0.276 &            0.250 &             0.303 &       0.321 \\
\bottomrule
\end{tabular}


  \caption{Frobenius operators on GS11}
  \label{tab:frobenius-gs11-results}
\end{sidewaystable}

\begin{sidewaystable}
  \centering
  \scriptsize
  \begin{tabular}{lllllllrrrrrr}
\toprule
       & {} &      &   &      &     &   &  copy-object &  copy-subject &  frobenius-add &  frobenius-mult &  frobenius-outer &  relational \\
selection & selection\_operator & dimensionality & freq & discr & neg & cds &              &               &                &                 &                  &             \\
\midrule
single                 & {}   & 1\,000 & 1    & scpmi & 0.7 & global &         0.61 &          0.61 &           0.61 &            0.57 &             0.54 &        0.54 \\
single                 & {}   & 2\,000 & 1    & scpmi & 0.7 & global &         0.54 &          0.57 &           0.57 &            0.54 &             0.61 &        0.61 \\
single                 & {}   & 3\,000 & 1    & scpmi & 0.7 & global &         \textbf{0.82} &          \textbf{0.82} &           0.71 &            0.82 &             0.82 &        0.82 \\ \addlinespace
universal (max)        & add  & 1\,000 & 1    & scpmi & 1.0 & global &         0.57 &          0.64 &           0.54 &            0.64 &             0.57 &        0.57 \\
universal (max)        & add  & 2\,000 & 1    & cpmi  & N/A & 1      &         0.54 &          0.64 &           0.57 &            0.57 &             0.57 &        0.57 \\
universal (max)        & add  & 3\,000 & logn & cpmi  & N/A & 1      &         0.79 &          \textbf{0.82} &           0.79 &            0.\textbf{86} &             0.82 &        0.82 \\ \addlinespace
universal (max)        & mult & 1\,000 & logn & spmi  & 0.7 & global &         0.57 &          0.68 &           0.61 &            0.61 &             0.54 &        0.54 \\
universal (max)        & mult & 2\,000 & logn & spmi  & 0.2 & 1      &         0.54 &          0.68 &           0.54 &            0.54 &             0.54 &        0.54 \\
universal (max)        & mult & 3\,000 & logn & spmi  & 0.7 & global &         0.71 &          0.64 &           0.61 &            0.68 &             0.68 &        0.68 \\ \addlinespace
universal (max)        & kron & 1\,000 & logn & spmi  & 1.0 & global &         0.68 &          0.71 &           0.57 &            0.64 &             0.54 &        0.54 \\
universal (max)        & kron & 2\,000 & logn & scpmi & 1.0 & global &         0.61 &          0.75 &           0.54 &            0.57 &             0.54 &        0.54 \\
universal (max)        & kron & 3\,000 & 1    & scpmi & 1.0 & global &         \textbf{0.82} &          0.79 &           \textbf{0.82} &            0.\textbf{86} &             \textbf{0.89} &        \textbf{0.89} \\ \addlinespace
universal (heuristics) & add  & 1\,000 & 1    & scpmi & 0.7 & global &         0.61 &          0.61 &           0.61 &            0.57 &             0.54 &        0.54 \\
universal (heuristics) & add  & 2\,000 & 1    & scpmi & 0.7 & global &         0.54 &          0.64 &           0.57 &            0.57 &             0.64 &        0.64 \\
universal (heuristics) & add  & 3\,000 & 1    & scpmi & 0.7 & global &         \textbf{0.82} &          0.79 &           0.75 &            0.82 &             0.82 &        0.82 \\ \addlinespace
universal (heuristics) & mult & 1\,000 & 1    & spmi  & 0.5 & global &         0.64 &          0.61 &           0.61 &            0.61 &             0.54 &        0.54 \\
universal (heuristics) & mult & 2\,000 & 1    & spmi  & 0.5 & global &         0.57 &          0.68 &           0.61 &            0.57 &             0.57 &        0.57 \\
universal (heuristics) & mult & 3\,000 & 1    & spmi  & 0.5 & global &         \textbf{0.82} &          0.79 &           0.71 &            0.82 &             0.82 &        0.82 \\ \addlinespace
universal (heuristics) & kron & 1\,000 & 1    & spmi  & 0.7 & global &         0.57 &          0.68 &           0.54 &            0.57 &             0.50 &        0.50 \\
universal (heuristics) & kron & 2\,000 & 1    & spmi  & 0.7 & global &         0.57 &          0.64 &           0.64 &            0.50 &             0.54 &        0.54 \\
universal (heuristics) & kron & 3\,000 & 1    & spmi  & 0.7 & global &         \textbf{0.82} &          0.79 &           0.79 &            0.\textbf{86} &             0.79 &        0.79 \\
\bottomrule
\end{tabular}


  \caption{Frobenius operators on PhraseRel}
  \label{tab:frobenius-phraserel-results}
\end{sidewaystable}


% \begin{wrapfigure}{O}{0.5\textwidth}
\begin{figure}[b]
  % \vspace{-30pt}
  \centering

  \begin{subfigure}[t]{0.9\textwidth}
  \includegraphics[width=\textwidth]{supplement/figures/frobenius-KS14-plot}

  \caption{KS14}
  \label{fig:frobenius-ks14-plot}
  \end{subfigure}

  \begin{subfigure}[t]{0.9\textwidth}
  \includegraphics[width=\textwidth]{supplement/figures/frobenius-GS11-plot}

  \caption{GS11}
  \label{fig:frobenius-gs11-plot}
  \end{subfigure}

  \begin{subfigure}[t]{0.9\textwidth}
  \includegraphics[width=\textwidth]{supplement/figures/frobenius-PhraseRel-plot}

  \caption{PhraseRel}
  \label{fig:frobenius-phraserel-plot}
  \end{subfigure}

  \caption{Frobenius results.}
  \label{fig:frobenius-results}
\end{figure}


\end{document}

%%% Local Variables:
%%% coding: utf-8
%%% mode: latex visual-line
%%% TeX-master: t
%%% TeX-command-extra-options: "-shell-escape"
%%% TeX-engine: xetex
%%% End: