\begin{figure}[t]
\tiny
\begin{subfigure}[t]{0.30\textwidth}
  Jaunkundze ar sunīti \\

  Un Vecrīgas šķērsielā, šaurā \\
  kā vēstuļu kastītes sprauga, \\
  kur troksnim un burzmai tik atbalss, \\
  kur smaržo pēc darvas, \\ ${}$\qquad dzelzs un pēc āboliem pagrabos sausos, \\
  es satiku jaunkundzi -- \\
  glītu un veiklu kā mēle, \\
  kā spēlējot vijoles lociņš.
  \vspace{0.74cm}
  \caption{}
  \label{fig:lv}
\end{subfigure}
~
\begin{subfigure}[t]{0.36\textwidth}
  Барышня с собачкой \\

  В Старой Риге, на улице поперечной, узкой, \\
  как щель в почтовый ящик, \\
  в который проникают только отголоски шума, гама, \\
  где запах дёгтя, ржавчины и яблок в сухих  подвалах, \\
  я встретил барышню -- \\
  красива и ловка - она - язык, \\
  смычок, играющий на скрипке.
  \vspace{0.99cm}
  \caption{}
  \label{fig:ru}
\end{subfigure}
~
\begin{subfigure}[t]{0.26\textwidth}
  Young Woman with a Dog \\

  On a narrow side-street in Riga’s old quarter, \\
  as though in a mailbox slot \\
  where noise and hustle only echo, \\
  and it smells of tar and steel \\
  and apples kept in dry basements, \\

  I met a young woman \\
  attractive and active \\
  as a tongue, \\
  as a violin-bow playing.
  \caption{}
  \label{fig:en}
\end{subfigure}
\caption[Three pieces of written natural language.]{Three pieces of written natural language. The text on
  Figure~\ref{fig:lv} is the beginning of the poem ``Jaunkundze ar sunīti'' by
  Aleksandrs Čaks, Figure~\ref{fig:ru} is a translation to Russian by Lora Trin,
  and Figure~\ref{fig:en} is an English translation by Inara Cedrins.}
\end{figure}
