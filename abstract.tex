{\Large \headingfont \thetitle}

\vspace{1em}

{\large \headingfont \theauthor}

\vspace{1em}

{\headingfont Abstract}

Categorical compositional methods \cite{DBLP:journals/corr/abs-1003-4394} cover a large number of grammatical phenomena such as the composition of a verb with its arguments, an adjective with a noun and a relative clause with a main clause. The advances in the field of lexical semantics \cite{mikolov2013efficient,baroni-dinu-kruszewski:2014:P14-1,TACL570} are successfully applied in compositional tasks \cite{milajevs-EtAl:2014:EMNLP2014}, however, it is not known whether there are word representations that perform well in both lexical and compositional tasks.
%
This work systematically tests a large number of distributional word representations while varying the parameters of the co-occurrence weighting functions.
The goal of this work is to find the best behaving parameters taking overfitting \cite{lapesa2014large} into account.
%
Here, we show that careful model selection leads to improvements in several lexical and compositional tasks in comparison to the current state-of-the-art methods and avoids overfitting.

\vfill

%Submitted in partial fulfillment of the requirements of the Degree of Doctor of Philosophy

Submitted for the degree of Doctor of Philosophy

Queen Mary University of London

\today


%%% Local Variables:
%%% mode: latex
%%% TeX-master: "thesis"
%%% End:
