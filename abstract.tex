\chapter*{Abstract}
\label{cha:abstract}

%   % Categorical compositional methods \newcite{DBLP:journals/corr/abs-1003-4394} cover a rather large
%   % number of grammatical phenomena such as composition of a verb with its
%   % arguments, adjectives and relative clauses. However, existing literature tend
%   % to study each linguistic phenomena separately by developing a theoretical
%   % foundation, implementing a computational model and evaluating the model on a
%   % specific task--tailored dataset, for example similarity measure of transitive
%   % sentences where verb arguments are nouns. Competitive results achieved in
%   % previous work suggest that it might be useful in tasks that process language
%   % with uncontrolled syntactic structure: paraphrase detection
%   % \cite{dolan2005microsoft} or dialogue act tagging \cite{Stolcke.etal00} to
%   % name a few.
